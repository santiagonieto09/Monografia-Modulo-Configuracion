\mychapter{Capítulo 1. Introducción}

\section{Justificación y planteamiento del problema}

La Universidad del Cauca ofrece una variedad de programas académicos orientados a la formación de profesionales en diversas áreas de conocimiento. Entre ellos, el programa de Contaduría Pública que tiene como propósito formar profesionales con una sólida capacidad en el ámbito de los negocios, responsables de garantizar la transparencia y veracidad de la información económica y financiera de las organizaciones. De igual manera, se busca formar contadores con las competencias necesarias para gestionar y administrar los recursos de em-
presas y entidades gubernamentales \cite{ContaduriaPublica1Justi}.

Sin embargo, se han presentado desafíos tecnológicos que impactan en la calidad de la enseñanza y el aprendizaje. Ante la carencia de un software contable propio y con el objetivo de fortalecer los procesos educativos dentro del programa de Contaduría Pública, se ha recurrido a la compra de licencias del software ERP Millenium Enterprise. No obstante, este software presenta la limitante de ser una versión académica que no da acceso a todos los módulos disponibles. Además, al ser un software externo, toda la configuración de las bases de datos debe ser gestionada por el proveedor, lo que genera retrasos y limita la capacidad de los profesores para personalizar el sistema de acuerdo a las necesidades específicas del programa.

Este tipo de dificultades subraya la importancia de integrar herramientas más eficientes y adaptadas, tal como lo sugieren varios estudios, los cuales demuestran que el uso de software en el aula apoya de manera significativa el alcance de los resultados de aprendizaje \cite{Kanapathippillai2012}, al
tiempo que contribuye al desarrollo y mejora de habilidades en el manejo de las Tecnologías de la Información y la Comunicación (TIC) \cite{WanMohdNori2016}. Además, se ha evidenciado que la integración de software en los procesos de enseñanza de contabilidad ofrece beneficios tangibles, lo que permite a los estudiantes aplicar sus conocimientos en el entorno empresarial \cite{Boulianne2014}.

El uso del software contable en el proceso de enseñanza y aprendizaje de la contabilidad ha sido ampliamente propuesto por muchos académicos destacados  \cite{WanMohdNori2016}, al punto que hoy en día se ha convertido en una necesidad su incorporación en las unidades de desarrollo académico (UDA) que promuevan el estudio y práctica del ciclo contable. Se argumenta que con la utilización de un software contable en clase se logra representar y simular las prácticas que se llevarían en la mayoría de las organizaciones, al mismo tiempo que se ofrece un recurso pedagógico alternativo capaz de promover conocimientos relevantes y fundamentales de
la contabilidad, proporcionando una mejor preparación de los estudiantes para el cambiante mundo de los negocios y la profesión contable \cite{Boulianne2014}.

La educación contable debe tener en cuenta los requisitos del mercado relacionados con las habilidades y conocimientos necesarios para los profesionales de la contaduría.
Esto incluye ajustar el plan de estudios, métodos y herramientas utilizados en el proceso de enseñanza, incorporando el uso de tecnologías de la información (TI), lo cual fomenta la formación de profesionales con habilidades relevantes para el mercado laboral \cite{Novak2021}. La integración de las TI fortalece dos dimensiones, en primer lugar mejorar las habilidades tecnológicas de los estudiantes y en segundo lugar, utilizar estas tecnologías como herramientas estratégicas de enseñanza que mejoran las experiencias de aprendizaje \cite{Sangster1992}.

En este contexto, para que los distintos modulos del sistema contable funcionen adecudamente, es necesario configurar y gestionar parámetros como paso inicial a la contabilidad de una empresa.  Por lo tanto, se propone implementar la propuesta del módulo de configuración como parte del ”Sistema Contable para Uso Educativo en el Programa de Contaduría Pública”. Este módulo permitirá configurar y administrar los elementos esenciales del sistema, como cuentas contables, terceros, unidades de medida,  productos, categorías, tipos de documentos, números de documentos, resolución de facturación e impuestos, entre otros. Con esta herramienta, los estudiantes podrán interactuar con un entorno contable configurable permitiendo el desarrollo de habilidades prácticas en la administración de sistemas contables, simulando escenarios empresariales. 

Al integrar estas tecnologías, el programa de Contaduría Pública no solo mejorará las competencias tecnológicas de los estudiantes, sino que también reforzará su capacidad para enfrentar los desafíos del mercado laboral y aplicar sus conocimientos en contextos reales. El desarrollo del módulo de configuración implica la implementación tanto del frontend como del backend, permitiendo la integración de una interfaz intuitiva para los usuarios y una lógica de negocio robusta que permita el manejo eficiente de la información.


\section{Objetivos}

\subsection{Objetivo General}
Desarrollar la propuesta del módulo de configuración\footnote{El módulo de configuración incluye parámetros importantes del sistema que serán usadas en otros módulos, como cuentas contables, terceros, unidades de medida,  productos, categorías, tipos de documentos, números de documentos, resolución de facturación e impuestos.} del ``Software contable para uso educativo en el programa de Contaduría Pública de la Universidad del Cauca'', con el propósito de ofrecer un entorno para la personalización y administración del sistema para adaptarse a las necesidades académicas y prácticas de estudiantes y docentes.

\subsection{Objetivos Específicos}
\begin{enumerate}
    \item Definir los requisitos funcionales y no funcionales del módulo de configuración como componente del “Software contable para uso educativo en el programa de Contaduría Pública de la Universidad del Cauca”, conforme a las necesidades contables y académicas del programa.

    \item Diseñar la arquitectura del módulo de configuración utilizando el modelo C4, aplicando principios de diseño modular, escalable y eficiente que permitan soportar adecuadamente los requisitos funcionales y no funcionales.

    \item Implementar el módulo de configuración, abarcando tanto el front-end como el back-end, considerando los requisitos funcionales, no funcionales y la arquitectura definida.

    \item Evaluar las funcionalidades implementadas del módulo de configuración, mediante pruebas de integración y aceptación, verificando su correcto funcionamiento y adecuada integración con el resto del sistema.
    
\end{enumerate}


\section{Metodología de trabajo}

Para el desarrollo de la práctica profesional, se adoptará un enfoque de entregas iterativas e incrementales, basado en un entorno similar al trabajo freelance, caracterizado por la autonomía, la flexibilidad horaria y la gestión de proyectos de manera independiente. En este modelo, el profesional actúa de forma autónoma, ofreciendo sus servicios a sus clientes según sus necesidades específicas \cite{mividalaboralFreelanceGua}. 

Sin embargo, dada la naturaleza modular del software contable a desarrollar, esta autonomía freelance se complementará con una comunicación estructurada y colaborativa con otros desarrolladores involucrados en el desarrollo del mismo. Esta interacción permite una mejor integración de los diferentes módulos del sistema, requiriendo coordinación constante en aspectos como la definición de interfaces, estándares de desarrollo y protocolos de comunicación entre módulos. La modalidad freelance autónoma se adapta así a un contexto colaborativo donde, aunque cada desarrollador mantiene su independencia operativa, se establece un marco de comunicación regular que contribuye a la coherencia técnica y funcional del producto final.

Para complementar esta modalidad de trabajo híbrida, se implementará Kanban, una herramienta que permite organizar y optimizar servicios profesionales adoptando una visión holística del trabajo, enfocándose en la mejora continua desde la perspectiva del cliente. Este método permite visualizar el flujo de trabajo y la carga de tareas mediante tableros que representan las distintas etapas del proceso organizadas en columnas como “pendiente”, “en progreso”, “en prueba” y “finalizado”, lo que facilita el seguimiento del avance hasta su culminación \cite{guiaKanban}.


\subsection{Trabajo Freelance}

El trabajo freelance representa una modalidad laboral donde el profesional ejerce su actividad de manera autónoma, estableciendo vínculos contractuales temporales con diversos clientes según las demandas específicas de cada proyecto; esta forma de trabajo se ha consolidado como una alternativa significativa al empleo tradicional en el panorama laboral contemporáneo, caracterizándose por su naturaleza flexible y su capacidad de adaptación a las dinámicas del mercado actual. La esencia del trabajo freelance radica en la libertad del profesional para estructurar su actividad laboral según sus propias necesidades y objetivos, sin estar sujeto a las restricciones organizacionales típicas del empleo convencional; el freelancer mantiene la autonomía para seleccionar sus proyectos, establecer sus tarifas y gestionar su tiempo de manera independiente, lo que le permite desarrollar una carrera profesional personalizada y diversificada. Esta modalidad laboral presenta múltiples dimensiones que la distinguen del empleo tradicional y que contribuyen a su creciente popularidad entre profesionales de diversas áreas \cite{mividalaboralFreelanceGua}:

\begin{itemize}
    \item \textbf{Flexibilidad Temporal y Geográfica:} Constituye uno de los aspectos más valorados de esta modalidad laboral, ya que permite al trabajador freelancer estructurar su jornada según sus preferencias personales y las exigencias específicas de cada proyecto, facilitando un equilibrio entre la vida profesional y personal. Esta flexibilidad temporal se complementa con la independencia geográfica, que elimina las limitaciones territoriales tradicionales del empleo convencional y posibilita el trabajo remoto desde cualquier ubicación.
    \item \textbf{Autonomía:} La autonomía se manifiesta en la capacidad del freelancer para tomar decisiones estratégicas sobre su negocio, desde la selección de clientes hasta la definición de metodologías de trabajo, lo que fomenta el desarrollo de habilidades empresariales y la construcción de una identidad profesional sólida.
    \item \textbf{Diversificación de Experiencias:} La diversificación de proyectos representa otra dimensión significativa, ya que el freelancer puede involucrase en iniciativas de diferentes sectores y naturalezas, enriqueciendo su experiencia profesional y ampliando su red de contactos, esta variedad no solo contribuye al crecimiento profesional, sino que también reduce los riesgos asociados a la dependencia de un único empleador o sector económico.
\end{itemize}


\subsection{Kanban}

El método Kanban representa un enfoque integral de gestión del trabajo que se fundamenta en la visualización y optimización de los flujos de trabajo, particularmente efectivo para servicios profesionales y trabajo del conocimiento. Este método adopta una perspectiva holística que prioriza la mejora continua de los servicios desde la óptica del cliente, permitiendo a las organizaciones desarrollar una comprensión profunda de cómo se desplaza el trabajo a través de sus procesos. La implementación de Kanban facilita una gestión empresarial más eficiente al proporcionar visibilidad sobre los riesgos asociados a la entrega de servicios, mientras que simultáneamente desarrolla la capacidad adaptativa organizacional necesaria para responder de manera ágil a las cambiantes necesidades del mercado y las expectativas de los clientes. Aunque inicialmente reconocido por su aplicación en equipos para mitigar la sobrecarga laboral y recuperar el control operativo, el verdadero potencial de Kanban se manifiesta cuando se implementa a escala organizacional, abarcando múltiples equipos y departamentos. En este amplio contexto, Kanban trasciende su función como herramienta de gestión de tareas para convertirse en un potente instrumento de desarrollo organizacional orientado al servicio, generando oportunidades significativas de mejora estructural y operativa \cite{guiaKanban}.

Para materializar estos principios conceptuales, los tableros Kanban constituyen el mecanismo de visualización más ampliamente utilizado en la implementación de sistemas Kanban. Estos tableros operan bajo una lógica direccional uniforme que facilita el seguimiento del progreso del trabajo: los elementos ingresan por el lado izquierdo del tablero y avanzan hacia la derecha hasta completar su ciclo de entrega de valor al cliente. La arquitectura de un sistema Kanban establece puntos claramente definidos de compromiso y entrega. Los elementos de trabajo pueden variar considerablemente en naturaleza y magnitud, abarcando desde tareas individuales y requisitos específicos hasta proyectos completos y paquetes de productos en tableros de nivel estratégico, adaptándose así a diversos contextos organizacionales como historias de usuario en desarrollo de software, procesos de recursos humanos o desarrollo de productos. Fundamentado en el principio de ``comenzar con lo que se hace actualmente'', el tablero Kanban refleja fielmente el flujo operativo real en lugar de una configuración idealizada, utilizando columnas para representar las etapas del proceso y carriles para distribuir la capacidad según diferentes tipos de trabajo o proyectos \cite{guiaKanban}.

\subsection{Desarrollo iterativo e incremental}

En un desarrollo iterativo e incremental el proyecto se organiza en varios bloques de tiempo conocidos como iteraciones. Las iteraciones funcionan como pequeños proyectos, en cada una se repite un flujo de trabajo similar (de ahí el término ``iterativo'') para ofrecer un resultado terminado sobre el producto final, permitiendo que el cliente reciba los beneficios del proyecto de manera progresiva, cada requisito debe finalizarse dentro de una sola iteración, donde se ejecutan todas las tareas necesarias para completarlo (incluyendo pruebas y documentación) asegurando que esté listo para entregarse al cliente con el mínimo esfuerzo adicional, así se evita dejar actividades críticas relacionadas con la entrega de requisitos para el final del proyecto. Durante cada iteración el equipo mejora el producto (realizando una entrega incremental) basándose en los resultados alcanzados en iteraciones previas, incorporando nuevos objetivos o requisitos o refinando los ya completados. Un elemento importante para dirigir el desarrollo iterativo e incremental es la priorización de los objetivos o requisitos según el valor que generan para el cliente \cite{iterativo}.

\subsection{Iteraciones}

A continuación se muestran las diferentes iteraciones con las actividades desarrolladas en cada una de ellas.

\begin{table}[h]
\centering
\caption{Descripción de la Fase 1}
\label{tab:sprint1}
\renewcommand{\arraystretch}{1.3} % mejora separación general
\begin{tabular}{|p{3cm}|p{10cm}|}
\hline
\multicolumn{2}{|c|}{\textbf{Fase 1: Análisis de requisitos}} \\
\hline
\textbf{Descripción} & Identificación y documentación de los requisitos funcionales y no funcionales del módulo de configuración. Priorización de funcionalidades según su valor para los interesados.\\
\hline
\textbf{Roles involucrados} & 
\begin{minipage}[t]{\linewidth}
\vspace{3pt} % espacio arriba
\begin{itemize}
\item Analista
\item Diseñador
\item Cliente
\end{itemize}
\vspace{4pt} % espacio abajo
\end{minipage} \\
\hline
\textbf{Actividades} & 
\begin{minipage}[t]{\linewidth}
\vspace{3pt}
\begin{itemize}
\item Reuniones presenciales y virtuales
\item Contrucción de historias de usuario
\item Construcción de prototipos
\end{itemize}
\vspace{3pt}
\end{minipage} \\
\hline
\textbf{Rol del practicante} & 
\begin{minipage}[t]{\linewidth}
\vspace{3pt}
\begin{itemize}
\item Analista
\item Diseñador
\end{itemize}
\vspace{3pt}
\end{minipage} \\
\hline
\end{tabular}
\end{table}

\begin{table}[h]
\centering
\caption{Descripción Iteración 2}
\label{tab:sprint2}
\begin{tabular}{|p{3cm}|p{10cm}|}
\hline
\multicolumn{2}{|c|}{\textbf{Iteración 2}} \\
\hline
\textbf{Fase} & Diseño de arquitectura \\
\hline
\textbf{Descripción} & Concepción de la arquitectura del sistema y lineamientos de desarrollo. \\
\hline
\textbf{Roles involucrados} & 
\begin{minipage}[t]{\linewidth}
\vspace{3pt}
\begin{itemize}
\item Arquitecto.
\item Desarrollador.
\item Director de grado.
\end{itemize}
\vspace{3pt}
\end{minipage} \\
\hline
\textbf{Actividades} & 
\begin{minipage}[t]{\linewidth}
\vspace{3pt}
\begin{itemize}
\item Revisión de la documentación.
\item Construcción de diagramas del sistema haciendo uso del modelo C4.
\end{itemize}
\vspace{3pt}
\end{minipage} \\
\hline
\textbf{Rol del practicante} & 
\begin{minipage}[t]{\linewidth}
\vspace{3pt}
\begin{itemize}
\item Arquitecto.
\item Desarrollador.
\end{itemize}
\vspace{3pt}
\end{minipage} \\
\hline
\end{tabular}
\end{table}
\begin{table}[h]
\centering
\caption{Descripción de la Fase 3}
\label{tab:sprint3}
\begin{tabular}{|p{3cm}|p{10cm}|}
\hline
\multicolumn{2}{|c|}{\textbf{Fase 3: Desarrollo}} \\
\hline
\textbf{Descripción} & Desarrollo de funcionalidades priorizadas. \\
\hline
\textbf{Roles involucrados} & 
\begin{minipage}[t]{\linewidth}
\vspace{3pt}
\begin{itemize}
\item Desarrollador.
\item Cliente.
\end{itemize}
\vspace{3pt}
\end{minipage} \\
\hline
\textbf{Actividades} & 
\begin{minipage}[t]{\linewidth}
\vspace{3pt}
\begin{itemize}
\item Implementación de funcionalidades.
\end{itemize}
\vspace{3pt}
\end{minipage} \\
\hline
\textbf{Rol del practicante} & 
\begin{minipage}[t]{\linewidth}
\vspace{3pt}
\begin{itemize}
\item Desarrollador
\end{itemize}
\vspace{3pt}
\end{minipage} \\
\hline
\end{tabular}
\end{table}
\begin{table}[h]
\centering
\caption{Descripción de la Fase 4}
\label{tab:sprint4}
\begin{tabular}{|p{3cm}|p{10cm}|}
\hline
\multicolumn{2}{|c|}{\textbf{Fase 4: Pruebas unitarias y de integración}} \\
\hline
\textbf{Descripción} & Validación del correcto funcionamiento de las funcionalidades implementadas mediante pruebas unitarias individuales y pruebas de integración. \\
\hline
\textbf{Roles involucrados} & 
\begin{minipage}[t]{\linewidth}
\vspace{3pt}
\begin{itemize}
\item Desarrollador
\item QA
\end{itemize}
\vspace{3pt}
\end{minipage} \\
\hline
\textbf{Actividades} & 
\begin{minipage}[t]{\linewidth}
\vspace{3pt}
\begin{itemize}
\item Pruebas unitarias.
\item Pruebas de integración.
\end{itemize}
\vspace{3pt}
\end{minipage} \\
\hline
\textbf{Rol del practicante} & 
\begin{minipage}[t]{\linewidth}
\vspace{3pt}
\begin{itemize}
\item Desarrollador
\item Tester
\end{itemize}
\vspace{3pt}
\end{minipage} \\
\hline
\end{tabular}
\end{table}
\begin{table}[h]
\centering
\caption{Descripción Iteración 5}
\label{tab:sprint5}
\begin{tabular}{|p{3cm}|p{10cm}|}
\hline
\multicolumn{2}{|c|}{\textbf{Iteración 5}} \\
\hline
\textbf{Fase} & Pruebas de aceptación con usuarios clave \\
\hline
\textbf{Descripción} & Verificación de la usabilidad, funcionalidad y adecuación del sistema para el entorno educativo contable. \\
\hline
\textbf{Roles involucrados} & 
\begin{minipage}[t]{\linewidth}
\vspace{3pt}
\begin{itemize}
\item Desarrollador
\item QA
\item Cliente
\end{itemize}
\vspace{3pt}
\end{minipage} \\
\hline
\textbf{Actividades} & 
\begin{minipage}[t]{\linewidth}
\vspace{3pt}
\begin{itemize}
\item Pruebas funcionales.
\item Pruebas de aceptación.
\end{itemize}
\vspace{3pt}
\end{minipage} \\
\hline
\textbf{Rol del practicante} & 
\begin{minipage}[t]{\linewidth}
\vspace{3pt}
\begin{itemize}
\item Desarrollador
\item Tester
\end{itemize}
\vspace{3pt}
\end{minipage} \\
\hline
\end{tabular}
\end{table}
\begin{table}[h]
\centering
\caption{Descripción de la Fase 6}
\label{tab:sprint6}
\begin{tabular}{|p{3cm}|p{10cm}|}
\hline
\multicolumn{2}{|c|}{\textbf{Fase 6: Revisión de entregas}} \\
\hline
\textbf{Descripción} & Evaluación integral del módulo de configuración desarrollado, analizando el cumplimiento de objetivos y requisitos establecidos. Se identifican oportunidades de mejora basadas en el feedback de usuarios y resultados de pruebas. \\
\hline
\textbf{Roles involucrados} & 
\begin{minipage}[t]{\linewidth}
\vspace{3pt}
\begin{itemize}
\item Desarrollador
\item Tester
\item Stakeholders
\item Usuario final
\end{itemize}
\vspace{3pt}
\end{minipage} \\
\hline
\textbf{Actividades} & 
\begin{minipage}[t]{\linewidth}
\vspace{3pt}
\begin{itemize}
\item Revisión de entregables y documentación
\item Análisis de feedback de usuarios
\item Identificación de mejoras potenciales
\end{itemize}
\vspace{3pt}
\end{minipage} \\
\hline
\textbf{Rol del practicante} & 
\begin{minipage}[t]{\linewidth}
\vspace{3pt}
\begin{itemize}
\item Analista
\item Desarrollador
\end{itemize}
\vspace{3pt}
\end{minipage} \\
\hline
\end{tabular}
\end{table}