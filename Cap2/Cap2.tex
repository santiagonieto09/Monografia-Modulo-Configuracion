
\mychapter{Capítulo 2. Marco Teórico}

La enseñanza de contabilidad en el siglo XXI requiere de profesionales con conocimientos sólidos y habilidades que permitan motivar a los estudiantes. Gracias a la tecnología, se ha facilitado la creación de herramientas para el aprendizaje individualizado \cite{Humphrey2014}, impulsando
modelos como Computer Assisted Learning (CAL) y otros basados en Tecnologías de la Infor-
mación y la Comunicación (TIC) \cite{Wali2021}.

El modelo CAL es el que se ha venido implementando en la educación contable al integrar
el software como parte esencial en el proceso de enseñanza y aprendizaje \cite{Huczynski2005}. Este modelo
permite aplicar diversas situaciones que ayudan al estudiante a comprender conceptos y gene-
rar experiencias prácticas \cite{Kanapathippillai2012}, ofreciendo un estilo de aprendizaje enriquecedor que aumenta
la motivación y facilita el aprendizaje activo \cite{Huczynski2005}. Sin embargo, es fundamental que no solo se
aprenda el uso del software, sino que este sea empleado como herramienta pedagógica \cite{Blount2016}.
Además, CAL beneficia a las universidades al optimizar costos y mejorar la calidad educativa
[9], reforzando su imagen como instituciones modernas \cite{Kanapathippillai2012}.

Debido a lo anterior, el dominio de las TIC se ha identificado como un factor diferenciador
en contabilidad y finanzas \cite{Osmani2020}. La International Federation of Accountants (IFAC), organismo
internacional que regula la profesión contable y promueve estándares de formación, enfatiza
en el desarrollo de habilidades técnicas, organizacionales y sociales en los contadores \cite{Wadi2021}.
Asímismo, organizaciones como CPA Australia, la American Accounting Association (AAA) y
el American Institute of Certified Public Accountants (AICPA) promueven la incorporación de
tecnologías emergentes en la educación contable \cite{Willis2016}. La adopción de estas tecnologías debe
formar parte de una estrategia educativa integral \cite{Sloan1995}.

El concepto de contador híbrido surge como respuesta a las nuevas exigencias del mercado laboral, donde se requiere que los profesionales combinen conocimientos contables con habilidades en tecnologías de la información \cite{Osmani2020}. Esto ha impulsado a las universidades a
mejorar la formación académica para responder a la demanda empresarial \cite{Sloan1995}. En este contexto, la IFAC, a través del International Accounting Education Standards Board (IAESB), ha desarrollado normativas internacionales como la International Education Standard 2 (IES 2),
la cual establece competencias esenciales en TIC para contadores \cite{Daff2021}. Estas competencias incluyen el análisis de datos, la gestión de riesgos y la optimización de sistemas organizacio nales, elementos clave para la formación y desempeño profesional en contabilidad \cite{Daff2021}.

Finalmente, el rápido avance tecnológico en el ámbito contable ha llevado a una evolución
en las herramientas utilizadas por los profesionales del sector. Tecnologías como la digitaliza-
ción de datos, la computación en la nube, la inteligencia artificial (IA), las tecnologías finan-
cieras (FinTech), el blockchain y el análisis de datos masivos (Big Data) han transformado los
procesos contables y financieros \cite{Daff2021}. La adopción y dominio de estas herramientas no solo
mejora la eficiencia de las organizaciones, sino que también fortalece la competitividad de los
contadores en el mercado laboral global \cite{Gould2017}. En este sentido, la formación académica debe
evolucionar constantemente para preparar a los futuros contadores ante los desafíos de la era
digital \cite{WanMohdNori2016}.

\section{Conceptos fundamentales}
A continuación se definen los conceptos y tecnologías empleadas durante el desarrollo de la práctica profesional, con el propósito de establecer un marco de referencia común para la comprensión del presente documento.

\subsection{Configuración}

En contabilidad, la configuración de un software contable se refiere a la implementación y adaptación de los ajustes, procesos, datos y reglas de negocio dentro del sistema para que se alineen precisamente con las necesidades operativas y fiscales de una empresa \cite{mygestion}. Esto implica tomar una herramienta estándar y personalizar sus módulos y funcionalidades, como la gestión de facturación, la contabilidad, el manejo de almacén y la configuración de datos fiscales o bancarios, para reflejar los flujos de trabajo y los requerimientos específicos del negocio \cite{mygestion}.

Una configuración adecuada permite gestionar las finanzas y operaciones de manera eficiente, ofreciendo simplicidad y control, similar a cómo un motor bien ajustado impulsa un automóvil \cite{cloudgestion}. Sin una estrategia clara y una configuración correcta, las empresas corren el riesgo de sufrir interrupciones costosas, ineficiencias operativas y no alcanzar los objetivos de la implementación del sistema \cite{aptyImplementationSteps}.

Además, permite centralizar la información y obtener una visión global en tiempo real, automatizar procesos, mejorar la toma de decisiones, facilitar el cumplimiento normativo y definir roles de usuario y entornos de trabajo \cite{diariosigloxxiCEESASA}. Es vital para gestionar procesos, procedimientos y proteger la información, asegurando que las transacciones se registren, administren y reporten de manera precisa dentro del contexto específico de la empresa \cite{managerCmoConfigurar}.

\subsection{Framework}
Un framework es una colección integrada de herramientas, aplicaciones y librerías que implementa estándares y patrones de diseño establecidos. Su objetivo principal es proporcionar una estructura cohesiva que facilite el desarrollo de software para fines específicos, permitiendo que los desarrolladores se enfoquen en la lógica de negocio fundamental sin preocuparse por tareas repetitivas o por la implementación de aspectos técnicos complejos y especializados \cite{FrameworkWikipedia}.

\subsection{Modelo C4}
El Modelo C4 está estructurado mediante niveles de abstracción organizados jerárquicamente, abarcando desde sistemas de software hasta contenedores, componentes y código fuente. Esta arquitectura se visualiza a través de diagramas dispuestos en diferentes niveles que muestran el contexto del sistema, la organización de contenedores, la estructura de componentes y los detalles del código. Una ventaja significativa de este modelo es que no depende de ninguna notación particular ni requiere herramientas específicas para su aplicación práctica \cite{c4modelHome}.

\subsection{Pruebas de software}
Las pruebas de software constituyen un proceso sistemático destinado a evaluar y confirmar que una aplicación o sistema funciona correctamente según las especificaciones establecidas. Sus principales ventajas incluyen la detección temprana de fallos y el incremento del rendimiento general del sistema.
Las pruebas resultan más eficientes cuando se ejecutan de manera continua, integrándose desde las fases iniciales de diseño y manteniéndose activas durante todo el ciclo de desarrollo hasta llegar al entorno de producción. Este enfoque continuo elimina la necesidad de esperar a que el producto esté completamente terminado para iniciar las validaciones.
Existen diferentes tipos pruebas de software, algunas de ellas son \cite{ibmPruebas}:

\begin{itemize}
    \item \textbf{Pruebas de aceptación:} Verifican si todo el sistema funciona según lo previsto.
    \item \textbf{Pruebas de integración:} Aseguran que los diferentes elementos o funcionalidades del software operan de manera coordinada e integrada.
    \item \textbf{Pruebas unitarias:} Validan que cada módulo individual del software opera conforme a lo previsto. Un módulo representa la parte más pequeña y verificable de una aplicación.
    \item \textbf{Pruebas funcionales:} Verifican las funcionalidades del sistema simulando situaciones empresariales reales basadas en los requerimientos funcionales establecidos. Las pruebas de caja negra constituyen un método común para validar estas funcionalidades.
    \item \textbf{Pruebas de rendimiento:} Evalúan el comportamiento del software bajo diversas condiciones de carga operativa. Las pruebas de carga, por ejemplo, se emplean para medir el desempeño del sistema en escenarios de uso real.
    \item \textbf{Pruebas de usabilidad:} Validan la facilidad con que un usuario puede interactuar con un sistema o aplicación web para llevar a cabo una tarea específica.
\end{itemize}

\subsection{Patrones arquitectónicos}
\subsubsection{Arquitectura hexagonal}
La arquitectura hexagonal, también conocida como arquitectura de puertos y adaptadores, es un patrón diseñado para crear componentes de aplicación débilmente acoplados, facilitando la conexión con el entorno externo. Su objetivo principal es aislar la lógica de negocio del núcleo de la aplicación de factores externos como bases de datos, interfaces de usuario o sistemas de mensajería, lo que mejora la modularidad, mantenibilidad y testabilidad. El sistema se divide en componentes intercambiables conectados a través de puertos (interfaces que representan puntos de entrada y salida) e implementados por adaptadores, que traducen las interacciones entre el mundo exterior y el núcleo \cite{HexagonalArchitecture}.

Como se ilustra en la Figura \ref{fig:hexagonal}, la arquitectura se organiza en capas concéntricas: el núcleo del dominio contiene entidades y lógica de negocio; la capa de aplicación incluye servicios que orquestan casos de uso; y la capa de infraestructura alberga adaptadores. Estos se clasifican en primarios (como controladores REST o interfaces gráficas, que inician la comunicación) y secundarios (como repositorios de bases de datos o servicios de mensajería, que responden a la aplicación). Los adaptadores primarios incluyen la API REST y la interfaz de línea de comandos, mientras que los secundarios comprenden PostgreSQL para persistencia y RabbitMQ para comunicación asíncrona entre microservicios.


\begin{figure}[h!]    
    \centering%
    \includegraphics[width=0.9\textwidth, height=0.8\textheight, keepaspectratio]{Cap2/Figuras/hexagonal.png}    
    \caption{Arquitectura con patrón hexagonal.}
    \label{fig:hexagonal}
\end{figure}

 Esta separación garantiza que la lógica de negocio permanezca independiente de los detalles de implementación tecnológicos, permitiendo reemplazar cualquier adaptador sin afectar el núcleo del dominio. El principio de inversión de control se aplica a nivel arquitectónico, donde las dependencias se dirigen hacia el centro del sistema, haciendo que la lógica de negocio solo dependa de los puertos (interfaces) diseñados para satisfacer sus necesidades y no de herramientas o adaptadores específicos.
 
\subsubsection{Arquitectura en capas}

La arquitectura por capas es una de las más empleadas, debido a su sencillez y al hecho de que se adopta automáticamente cuando no se tiene certeza sobre qué patrón arquitectónico elegir para el desarrollo de la aplicación. Este enfoque implica segmentar la aplicación en diferentes niveles, con el propósito de asignar a cada uno una función específica, tales como una capa de interfaz de usuario (UI), una capa de lógica de negocio (servicios) y una capa de acceso a datos (DAO). No obstante, este patrón no establece un número fijo de capas para la aplicación, sino que enfatiza la división de la aplicación en niveles (aplicando el principio de Separación de Responsabilidades (SoC)) \cite{ArquitecturaCapas}.

En la implementación práctica, generalmente se utiliza un esquema de 4 capas: presentación, negocio, persistencia y base de datos. Sin embargo, es común observar que las capas de negocio y persistencia se fusionan en una sola, especialmente cuando la lógica de almacenamiento se integra directamente en la capa de negocio \cite{ArquitecturaCapas}. 

En la \autoref{fig:capas} se observa la arquitectura organizada en cuatro capas horizontales que representan la separación de responsabilidades.

\begin{figure}[h!]    
    \centering%
    \includegraphics[width=1.0\textwidth, height=1.0\textheight, keepaspectratio]{Cap2/Figuras/capas.png}    
    \caption{Arquitectura con patrón en capas.}
    \label{fig:capas}
\end{figure}

\subsection{Patrones de diseño}

Los patrones de diseño representan estrategias recurrentes para abordar problemas comunes en el desarrollo de software, que funcionan como guías conceptuales que pueden adaptarse y personalizarse según las necesidades específicas de cada proyecto. A diferencia de una función o biblioteca lista para usar, un patrón de diseño no es un fragmento de código concreto, sino una solución abstracta que orienta la estructura y organización del programa. Su implementación requiere interpretar el concepto y ajustarlo a las particularidades del sistema en desarrollo \cite{patterndesign}.

\subsubsection{Patrón strategy}

Es una solución de diseño orientada al comportamiento que facilita la definición de múltiples algoritmos relacionados, encapsulando cada uno en una clase distinta y permitiendo que sus instancias se puedan intercambiar dinámicamente según las necesidades del sistema. El patrón Strategy se compone de tres elementos fundamentales: el contexto, que mantiene una referencia a una estrategia y delega en ella la ejecución de la tarea, la interfaz Strategy, que establece un contrato común para todas las estrategias concretas y las estrategias concretas, que implementan las distintas variantes del algoritmo. Esta organización permite separar la lógica de negocio de los detalles específicos de cada algoritmo, lo que facilita la realización de pruebas y la evolución del sistema \cite{refactoringStrategy}.

\subsubsection{Patrón publicador-suscriptor}

Es un patrón de diseño de comportamiento en el que un objeto, llamado sujeto, mantiene una lista de sus dependientes, llamados observadores, y les notifica automáticamente cualquier cambio de estado. Este patrón es útil cuando los cambios en el estado de un objeto pueden requerir cambios en otros objetos, y el grupo de objetos puede ser desconocido de antemano o cambiar dinámicamente. Además, el patrón publicador-suscriptor ayuda a desacoplar los componentes del sistema, lo que facilita su mantenimiento y escalabilidad \cite{Observer}. 

\section{Tecnologías para el desarrollo}

Para el desarrollo del módulo es necesario utilizar diversas tecnologías que permitan su correcto funcionamiento, permitiendo tanto la interacción con los usuarios como el procesamiento de la información. Estas herramientas han sido seleccionadas con el objetivo de mejorar la experiencia del usuario, la eficiencia en el manejo de datos y la estabilidad del sistema. A continuación, se presentan las principales tecnologías que serán empleadas, abarcando desde el diseño de la interfaz hasta la gestión de la información en el servidor y la base de datos. 

\subsection{Tecnologías frontend}

\textbf{Angular y Tailwind:} En el desarrollo frontend, la combinación de herramientas que atienden
tanto la lógica funcional como el diseño visual de las aplicaciones resulta fundamental para construir soluciones eficientes, escalables y mantenibles. En este contexto, Angular y Tailwind CSS se presentan como tecnologías que, aunque abordan áreas distintas, pueden integrarse de forma complementaria para optimizar el proceso de desarrollo.

Angular es un framework que proporciona una arquitectura robusta basada en componentes, permitiendo estructurar las aplicaciones en vistas jerárquicas que interactúan entre sí.
Estas vistas se comunican mediante la inyección de dependencias con servicios que encapsulan la lógica de negocio o funcionalidades reutilizables, sin necesidad de estar directamente ligados a la interfaz gráfica. Además de su enfoque estructural, Angular incorpora herramientas integradas para la realización de pruebas unitarias y de integración, utilizando frameworks como Jasmine y Karma, lo cual favorece el desarrollo confiable y orientado a pruebas desde las primeras fases del proyecto \cite{Angular1MT}.

Por su parte, Tailwind CSS es un framework de estilos que permite aplicar directamente clases predefinidas en el HTML, eliminando la necesidad de escribir archivos CSS personalizados. Este enfoque promueve la rapidez en la maquetación de interfaces, permitiendo ajustes visuales inmediatos sin salir del flujo de trabajo del código. Tailwind favorece un diseño consistente, reutilizable y altamente personalizable, facilitando la implementación de interfaces modernas y adaptables a distintas resoluciones y dispositivos \cite{Tailwind1MT}.

\subsection{Tecnologías backend}

El desarrollo backend moderno se fundamenta en la integración de diversas tecnologías que, en conjunto, permiten estructurar aplicaciones web eficientes, escalables y orientadas a servicios. Tecnologías como JSON, REST, Spring Framework, Java y herramientas de registro de servicios como Eureka no solo cumplen funciones individuales dentro del ecosistema de desarrollo, sino que trabajan de forma articulada para facilitar el procesamiento, la entrega y la organización de los datos en una aplicación.

En este flujo de trabajo, JSON actúa como el formato de intercambio de datos por excelencia, gracias a su ligereza, legibilidad y compatibilidad con múltiples lenguajes. Su estructura basada en pares clave-valor lo hace ideal para representar objetos y transmitir información
entre cliente y servidor de forma sencilla y estructurada \cite{Json1MT}. Este formato cobra especial relevancia dentro de arquitecturas REST, donde los datos se transportan principalmente en formato JSON. REST define un estilo de arquitectura que permite que los distintos sistemas se
comuniquen a través de HTTP mediante operaciones estándar (GET, POST, PUT, DELETE), y que los recursos se identifiquen a través de URLs claras. Al combinar REST con JSON, se obtiene una interfaz de comunicación eficiente y estandarizada para consumir y exponer servicios web \cite{Rest1MT}.

Para implementar esta lógica de negocio entra en juego Spring Framework, una herramienta robusta que proporciona todos los componentes necesarios para desarrollar APIs de manera modular y mantenible. Spring no solo ofrece soporte para la construcción de controladores que gestionen solicitudes HTTP y devuelvan respuestas en formato JSON, sino que también simplifica aspectos internos como la inyección de dependencias, la gestión de datos o la configuración de pruebas. Su compatibilidad nativa con REST y su estructura basada en componentes lo convierten en un puente ideal entre los principios arquitectónicos REST y el desarrollo práctico de servicios web \cite{Spring1MT}.

En arquitecturas distribuidas de microservicios, se hace necesario incorporar herramientas especializadas para el registro y descubrimiento de servicios. Eureka Server, desarrollado inicialmente por Netflix, cumple esta función al actuar como un servidor centralizado donde los microservicios pueden registrarse automáticamente para ser localizados por otros servicios. Esta herramienta elimina la necesidad de definir manualmente las direcciones de cada servicio, gestionando dinámicamente el proceso de descubrimiento y comunicación entre componentes distribuidos \cite{EurekaServer}.

Todo este ecosistema se construye sobre Java, el lenguaje base que aporta características
como portabilidad, orientación a objetos, robustez y seguridad. Java no solo permite imple-
mentar la lógica del negocio y manipular estructuras complejas de datos, sino que también se
adapta perfectamente al uso de frameworks como Spring, asegurando rendimiento y fiabilidad
en aplicaciones backend de cualquier escala \cite{JAVA1MT}.


\subsection{Tecnologías para base de datos}

La gestión eficiente de la persistencia de datos en una aplicación requiere herramientas
que reduzcan la complejidad del acceso a bases de datos relacionales y que, al mismo tiempo,
se integren de manera fluida con el lenguaje de programación utilizado. En este sentido, Java
Persistence API (JPA) e Hibernate forman una combinación poderosa que facilita el trabajo
con bases de datos desde entornos orientados a objetos como Java.

JPA proporciona una especificación estándar para mapear clases Java a tablas de una base de datos, permitiendo manipular datos como objetos sin tener que interactuar directamente con el lenguaje SQL. Esto se traduce en una capa de abstracción que permite al desarrollador realizar operaciones como insertar, actualizar, eliminar o consultar registros, utilizando una sintaxis más cercana al paradigma orientado a objetos \cite{JPA1MT}.

Por su parte, Hibernate es una implementación concreta y extendida de esa especificación.No solo cumple con los lineamientos definidos por JPA, sino que añade funcionalidades avanzadas que automatizan procesos comunes como la sincronización entre objetos y registros, el control de transacciones o el uso de caché para mejorar el rendimiento. Hibernate actúa como el motor que lleva a cabo las operaciones indicadas por JPA, reduciendo la necesidad de código repetitivo y facilitando el mantenimiento del sistema \cite{Hibernate1MT}.

El sistema relacional que sirve de soporte a esta arquitectura de persistencia es PostgreSQL. Este sistema de gestión de bases de datos objeto-relacional (SGBDOR) de código abierto es reconocido por su robustez, su fuerte cumplimiento del estándar SQL y sus avanzadas capacidades transaccionales, incluyendo el control de concurrencia multi-versión (MVCC), lo que lo hace ideal para aplicaciones empresariales que requieren alta integridad y consistencia de datos \cite{googleQuPostgreSQL}.

\section{Lineamientos división TIC }

\subsection{Lineamientos de diseño}

El documento ``Lineamientos de diseño UX-UI para proyectos de desarrollo de software en la División TIC de la Universidad del Cauca'', desarrollado por el área de soporte y desarrollo en su primera versión, define directrices para el diseño en proyectos de desarrollo de software. Este incluye consideraciones sobre tipografía, esquemas cromáticos y enfoques de diseño, ya que los ``proyectos deben mantener la identidad visual del Alma Mater, con la finalidad de reforzar la institucionalidad y el posicionamiento de la marca en diversos entornos digitales'' \cite{TICS}. Entre estas directrices se encuentran especificaciones tipográficas que determinan el uso de las fuentes Titillium Web y Open Sans, así como definiciones cromáticas que establecen diferentes gamas de colores y su aplicación. En las figuras \ref{fig:fig1}, \ref{fig:fig2} y \ref{fig:fig3} se presentan algunos de los colores fundamentales.

\begin{figure}[h!]
    \centering%
    \includegraphics[width=0.6\textwidth, height=0.3\textheight, keepaspectratio]{Cap2/Figuras/Figura1.png}    
    \caption{Color institucional primario}
    \label{fig:fig1}
\end{figure}

\begin{figure}[h!]    
    \centering%
    \includegraphics[width=0.6\textwidth, height=0.3\textheight, keepaspectratio]{Cap2/Figuras/Figura2.png}    
    \caption{Color institucional secundario}
    \label{fig:fig2}
\end{figure}

\begin{figure}[h!]
    \centering%
    \includegraphics[width=0.6\textwidth, height=0.3\textheight, keepaspectratio]{Cap2/Figuras/Figura3.png}    
    \caption{Otros colores destacados}
    \label{fig:fig3}
\end{figure}

\newpage
\subsection{Lineamientos de desarrollo}
La Universidad del Cauca, a través de su división TIC, establece directrices de desarrollo que determinan, entre otros aspectos, las tecnologías a utilizar para el desarrollo de aplicaciones institucionales, las cuales incluyen las siguientes \cite{TICS2}:

\begin{table}[h]
    \centering
    \caption{Tecnologías para proyectos software de la división TIC}
    \label{tab:tecnologias}
    \begin{tabular}{|p{4cm}|p{8cm}|}
        \hline
        \textbf{Tipo} & \textbf{Tecnología} \\
        \hline
        Bases de datos SQL & 
        \begin{itemize}
            \item Oracle.
            \item MySQL.
            \item PostgreSQL.
        \end{itemize} \\
        \hline
        Bases de datos no SQL & 
        \begin{itemize}
            \item MongoDB.
            \item Cassandra.
        \end{itemize} \\
        \hline
        Bases de datos Real Time & 
        \begin{itemize}
            \item Firebase
        \end{itemize} \\
        \hline
        Back-end & 
        \begin{itemize}
            \item Spring Boot
            \item Java
        \end{itemize} \\
        \hline
        Front-end & 
        \begin{itemize}
            \item JavaScript
            \item Vue
            \item React con Next
            \item Angular con TypeScript
        \end{itemize} \\
        \hline
    \end{tabular}
\end{table}

\subsection{Lineamientos de usabilidad}
En el documento ``Lineamientos normativos y pautas de usabilidad para el diseño de interfaces en los proyectos de desarrollo software en la División TIC de la Universidad del Cauca'' se definen criterios de usabilidad, entre los cuales se encuentran \cite{TICS3}:

\begin{itemize}
    \item \textbf{Tipografía legible:} Uso de fuentes con dimensiones y contraste apropiados que garanticen una lectura cómoda para los usuarios.
    \item \textbf{Colores accesibles:} Implementación de esquemas de color que mantengan la identidad visual institucional, asegurando un contraste óptimo entre el texto y el fondo para promover la legibilidad y reducir el cansancio visual.
    \item \textbf{Navegación intuitiva:} Desarrollo de una arquitectura gráfica transparente y de fácil comprensión mediante el uso de encabezados y divisiones apropiadas que orienten al usuario en la búsqueda de información.
    \item \textbf{Contenido accesible:} Presentación de contenido claro y directo, empleando un lenguaje accesible y evitando terminología técnica innecesaria.
\end{itemize}
