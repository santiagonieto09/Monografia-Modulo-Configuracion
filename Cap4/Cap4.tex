\mychapter{Capítulo 4. Ingeniería de requisitos}
\label{cap:IngRequisitos} 
La ingeniería de requisitos es una etapa fundamental en un proyecto de software, ya que permite identificar, analizar y documentar las necesidades de un sistema antes de su implementación. Un levantamiento adecuado de requisitos garantiza que el resultado cumpla las expectativas del cliente. Es por esto que en este capítulo se describe el proceso llevado a cabo para la especificación de los requerimientos funcionales y no funcionales del módulo de configuración del ``Software contable para uso educativo en el programa de Contaduría Pública de la Universidad del Cauca''.

Durante la ingeniería de requisitos llevada a cabo en el proyecto se realizaron diferentes actividades con el fin de obtener una lista de tareas congruente con los requisitos del software contable educativo. Para hacer posible este proceso fue fundamental la comunicación constante con los principales interesados, docentes del programa de Contaduría Pública, el asesor empresarial Mauro Andrés Sánchez Muñoz y el director de la práctica profesional, quienes muy amablemente reservaron espacios en sus agendas para atender las inquietudes resultantes. Esto permitió una retroalimentación constante y sumado a la flexibilidad de la metodología de trabajo híbrida adoptada, fue posible ir ajustando los requisitos en cada encuentro, incluso en etapas posteriores del desarrollo.

También se debe tener en cuenta que el análisis previo descrito en el ``Capítulo 3. Descripción de los módulos a implementar'', en el cual se identificó los submódulos que componen el módulo de configuración, permitió tener una visión clara sobre los procesos y actividades involucradas en la gestión contable educativa. Lo anterior claramente benefició la recolección y especificación de requerimientos, pues se pudo avanzar con mayor seguridad y velocidad en la definición de funcionalidades para cada submódulo.

Sin embargo, a pesar de la disposición de los interesados, se presentaron dificultades relacionadas con la coordinación de horarios y las múltiples ocupaciones de los participantes. En ocasiones, los docentes del programa de Contaduría Pública no disponían de tiempo suficiente para el proyecto debido a sus compromisos académicos, evaluaciones y actividades administrativas. A pesar de ello, se logró ajustar las actividades y el trabajo requerido para el levantamiento de requisitos, superando los inconvenientes mediante la flexibilidad horaria característica del enfoque freelance adoptado.

\section{Actividades desarrolladas para la ingeniería de requisitos}
Las actividades desarrolladas para llevar a cabo la ingeniería de requisitos para este proyecto son las siguientes.

\subsection{Reuniones con los interesados}
Como previamente se ha mencionado, durante el proyecto fue de suma importancia los encuentros presenciales y virtuales con los interesados, donde se resolvieron dudas sobre aspectos contables y pedagógicos, y se obtuvo retroalimentación constante la cual moldeaba el trabajo que se realizaba. También en ocasiones se compartió documentación sobre las actividades llevadas a cabo en el programa de Contaduría Pública, como planes de estudio, casos de uso contables y ejemplos de ejercicios prácticos útiles para el proceso. Se debe notar que, dada la naturaleza técnica del proyecto, la comunicación presencial y por videoconferencia resultó más efectiva que los medios digitales asíncronos para la discusión de conceptos contables complejos.

\subsection{Construcción de prototipos}
La construcción de prototipos fue fundamental para no dirigir el diálogo con los docentes de forma exclusivamente técnica, sino establecer una conversación apoyada visualmente por los prototipos de interfaz. Los interesados aportaron sus comentarios sobre cambios en las pantallas de gestión de cuentas contables, terceros, productos, impuestos y demás submódulos, evaluando su adecuación para el entorno educativo. A su vez se realizaron negociaciones sobre la complejidad de las funcionalidades según el nivel académico de los estudiantes. Los cambios producto de estas negociaciones eran presentados como apertura en la siguiente reunión para poder continuar el proceso iterativo.

\subsection{Definición de historias de usuario}
La conversación con los docentes se llevó a cabo de forma pedagógica y orientada a casos de uso contables reales, como se mencionó anteriormente. Sin embargo, para el proceso de desarrollo del proyecto es fundamental tener documentación estructurada que defina claramente las funcionalidades que requiere el módulo de configuración del software contable educativo. Es por esto que las historias de usuario como artefacto para el levantamiento de requerimientos fueron realizadas de forma paralela a la construcción de prototipos, considerando tanto los aspectos técnicos del desarrollo como los objetivos pedagógicos del programa de Contaduría Pública. Estas historias fueron ajustadas según la retroalimentación brindada por el cliente en cada encuentro, asegurando que cada submódulo cumpliera con los requisitos educativos específicos.

\subsection{Validación de prototipos}
Como ya fue mencionado, la validación de prototipos se realizó en la apertura de cada encuentro con el cliente, de modo que se tuviera seguridad de que las interfaces diseñadas fueran intuitivas para estudiantes de contabilidad y cumplieran con los lineamientos de usabilidad establecidos por la División TIC de la Universidad del Cauca. Se evaluó especialmente la claridad de los formularios de gestión de cada submódulo y la coherencia con los procesos contables estándar.

\subsection{Validación de historias de usuario}
La validación de las historias de usuario fue realizada en conjunto con la validación de prototipos, pues estas fueron ajustadas en cada encuentro considerando tanto los aspectos técnicos como los requisitos pedagógicos específicos del programa de Contaduría Pública. Se prestó especial atención a que cada funcionalidad contribuyera efectivamente al aprendizaje de conceptos contables fundamentales y permitiera la simulación de escenarios empresariales reales.

\section{Identificación de Stakeholders}
Para identificar los stakeholders del proyecto fue fundamental la colaboración con el asesor empresarial y el director de la práctica profesional. Es por esto que se logra la identificación de los interesados como a continuación se describe:

\begin{table}[h]
\centering
\caption{Stakeholders del proyecto}
\label{tab:stakeholders}
\begin{tabular}{|p{6cm}|p{8cm}|}
\hline
\textbf{Interesado} & \textbf{Cargo} \\
\hline
Julio Ariel Hurtado Alegria & Asesor de la Universidad del Cauca \\
\hline
Wilson Libardo Pantoja & Asesor de la Universidad del Cauca \\
\hline
Juan Ignacio Oviedo & Asesor de la organización \\
\hline
Viviana Narvaez & Asesor de la organización \\
\hline
Mauro Sanchez & Asesor de la organización \\
\hline
Omar Gomez Gomez & Usuario Final Experto \\
\hline
Brayan Neil Vargas & Usuario Final Experto \\
\hline
Comunidad universitaria & Estudiantes administrativos y docentes \\
\hline
\end{tabular}
\end{table}
\newpage

\section{Prototipos}

Como ya se ha mencionado, la construcción de prototipos fue esencial para una comunicación clara con el asesor empresarial, ya que al permitir visualizar un acercamiento de lo que serían las interfaces del módulo de configuración se obtenía retroalimentación constante y rápida sobre aspectos pedagógicos y de usabilidad. Esto ha permitido realizar un refinamiento de la especificación de requerimientos en el formato de historias de usuario, este último también ha contribuido en sentido contrario a detallar más cuidadosamente los prototipos, generando así un círculo virtuoso de mejora continua.

Los prototipos se enfocaron en representar las interfaces de los submódulos que componen el módulo de configuración. Cada prototipo consideró los lineamientos de diseño UX-UI establecidos por la División TIC de la Universidad del Cauca, incluyendo la paleta de colores institucional y las fuentes tipográficas especificadas.

Para el diseño de los prototipos se hizo necesario el uso de una herramienta flexible y de fácil manejo, en la que se pudiera plasmar rápidamente el concepto de las interfaces del módulo de configuración. La herramienta elegida fue Figma por su comodidad, flexibilidad y fácil manejo para el diseño de interfaces web, además de permitir la colaboración en tiempo real con los interesados del proyecto. Por tanto, si se desea explorar en su totalidad los prototipos desarrollados, es preciso dirigirse al proyecto Figma en el siguiente enlace [prototipos] o en el anexo A, donde se encuentran compilados todos los diseños de las interfaces del módulo de configuración.

Por lo tanto, se incluyen a continuación ejemplos seleccionados de los prototipos desarrollados, evitando así saturar el documento y optimizando la comprensión visual.