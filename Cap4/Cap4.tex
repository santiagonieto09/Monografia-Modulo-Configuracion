\mychapter{Capítulo 4. Ingeniería de requisitos}
\label{cap:IngRequisitos} 
La ingeniería de requisitos es una etapa fundamental en un proyecto de software, ya que permite identificar, analizar y documentar las necesidades de un sistema antes de su implementación. Un levantamiento adecuado de requisitos permite que el resultado cumpla las expectativas del cliente. Es por esto que en este capítulo se describe el proceso llevado a cabo para la especificación de los requerimientos funcionales y no funcionales del módulo de configuración del ``Software contable para uso educativo en el programa de Contaduría Pública de la Universidad del Cauca''.

Durante la ingeniería de requisitos llevada a cabo en el proyecto se realizaron diferentes actividades con el fin de obtener una lista de tareas congruente con los requisitos del software contable educativo. Para hacer posible este proceso fue fundamental la comunicación constante con los principales interesados, docentes del programa de Contaduría Pública, el asesor empresarial Mauro Andrés Sánchez Muñoz y el director de la práctica profesional PhD. Wilson Libardo Pantoja Yepez, quienes muy amablemente reservaron espacios en sus agendas para atender las inquietudes resultantes. Esto permitió una retroalimentación constante y sumado a la flexibilidad de la metodología de trabajo híbrida, fue posible ir ajustando los requisitos en cada encuentro, incluso en etapas posteriores del desarrollo.

También se debe tener en cuenta que el análisis previo descrito en el \autoref{chap:submodulos}, en el cual se identificaron los submódulos que componen el módulo de configuración, permitió tener un contexto más amplio sobre los procesos y actividades involucradas en la gestión contable educativa. Lo anterior claramente benefició la recolección y especificación de requerimientos, dado que fue posible avanzar con mayor seguridad y velocidad en la definición de funcionalidades para cada submódulo.

Sin embargo, a pesar de la disposición de los interesados, se presentaron dificultades relacionadas con la coordinación de horarios y las múltiples ocupaciones de los participantes. En ocasiones, los docentes del programa de Contaduría Pública no disponían de tiempo suficiente para el proyecto debido a sus compromisos académicos, evaluaciones y actividades administrativas. A pesar de ello, se logró ajustar las actividades y el trabajo requerido para el levantamiento de requisitos, superando los inconvenientes mediante la flexibilidad horaria característica del enfoque freelance adoptado.

A continuación, se explica detalladamente la etapa de elicitación de requisitos, en la cual se capturaron, analizaron y validaron las necesidades con los stakeholders.
\section{Elicitación de requisitos}

La fase de elicitación de requisitos se llevó a cabo mediante un proceso sistemático que integra la experiencia práctica acumulada durante el Aprendizaje Basado en Proyectos (PBL) y actividades estructuradas de ingeniería de requisitos. El conocimiento previo del dominio contable, adquirido en las fases iniciales del PBL, proporcionó una base sólida para identificar y priorizar necesidades, mientras que las técnicas formales de elicitación permitieron una especificación completa y validada.


El proceso se organizó en cinco actividades principales:

\begin{itemize}
\item \textbf{Captura:} Se realizaron entrevistas estructuradas con los stakeholders listados en la  \autoref{tab:stakeholders}, se analizó documentación institucional y se levantaron modelos de negocio para comprender los flujos contables educativos.

\item \textbf{Especificación:} Los requisitos se formalizaron mediante historias de usuario, criterios de aceptación detallados y prototipos, lo que facilitó su comprensión y validación temprana.

\item \textbf{Negociación:} Se identificaron y resolvieron discrepancias entre las expectativas de los interesados, los lineamientos institucionales y las restricciones técnicas a través de videoconferencias de discusión y ajuste como se evidencia en la \autoref{fig:evidencia1}.

\item \textbf{Verificación:} Se aplicó revisión por pares como se evidencia en la \autoref{fig:evidenciaHU}, en la cual se presentaron las especificaciones a los directores del proyecto para confirmar su corrección, integridad, claridad y coherencia con los objetivos del sistema.

\item \textbf{Validación:} Los requisitos fueron expuestos y validados con los stakeholders clave, incluyendo docentes del programa de Contaduría Pública y asesores técnicos, mediante demostraciones de prototipos y pruebas funcionales preliminares como se muestra en la \autoref{fig:evidenciaprototipos}, asegurando su alineación con las expectativas y necesidades educativas.
\end{itemize}


\begin{table}[h]
\centering
\caption{Stakeholders del proyecto}
\label{tab:stakeholders}
\begin{tabular}{|p{6cm}|p{8cm}|}
\hline
\textbf{Interesado} & \textbf{Cargo} \\
\hline
Julio Ariel Hurtado Alegria & Asesor de la Universidad del Cauca \\
\hline
Wilson Libardo Pantoja & Asesor de la Universidad del Cauca \\
\hline
Juan Ignacio Oviedo & Asesor de la organización \\
\hline
Viviana Narvaez & Asesor de la organización \\
\hline
Mauro Sanchez & Asesor de la organización \\
\hline
Omar Gomez Gomez & Usuario Final Experto \\
\hline
Brayan Neil Vargas & Usuario Final Experto \\
\hline
Comunidad universitaria & Estudiantes administrativos y docentes \\
\hline
\end{tabular}
\end{table}

\begin{figure}[h!]    
    \centering%
    \includegraphics[width=1.0\textwidth, height=1.0\textheight, keepaspectratio]{Cap4/Figuras/evidencia1.png}    
    \caption{Evidencia sesiones de negociación.}
    \label{fig:evidencia1}
\end{figure}

    
\begin{figure}[h!]    
    \centering%
    \includegraphics[width=1.0\textwidth, height=1.0\textheight, keepaspectratio]{Cap4/Figuras/evidenciaHU.png}    
    \caption{Evidencia sesión de revisión de especificaciones.}
    \label{fig:evidenciaHU}
\end{figure}

\begin{figure}[h!]    
    \centering%
    \includegraphics[width=1.0\textwidth, height=1.0\textheight, keepaspectratio]{Cap4/Figuras/evidenciaprototipos.jpeg}    
    \caption{Evidencia sesión de revisión de prototipos.}
    \label{fig:evidenciaprototipos}
\end{figure}


\section{Especificación de requisitos funcionales}

La \autoref{tab:epicas} refleja las necesidades identificadas y se estructura mediante historias épicas, las cuales a su vez se desglosan en historias de usuario. Las historias épicas agrupan funcionalidades o tareas complejas, mientras que las historias de usuario detallan estas épicas en descripciones manejables que permiten su desarrollo independiente.

\begin{longtable}{|p{2cm}|p{6cm}|p{6cm}|}
\caption{Historias de Épicas}
\label{tab:epicas} \\
\hline
\textbf{ID Épica} & \textbf{Nombre} & \textbf{Descripción} \\
\hline
\endfirsthead

\multicolumn{3}{c}{{\bfseries Tabla \thetable\ Continuación: Historias de Épicas}} \\
\hline
\textbf{ID Épica} & \textbf{Nombre} & \textbf{Descripción} \\
\hline
\endhead
\hline
\endfoot
\endlastfoot
HE-01 & Gestionar el catálogo de cuentas & Yo como administrador del sistema
\newline Quiero gestionar el catálogo de cuentas
\newline Para la correcta administración, actualización y disponibilidad de la información contable necesaria \\
\hline
HE-02 & Gestionar las tarifas de los impuestos & Yo como administrador del sistema
\newline Quiero gestionar las tarifas de los impuestos
\newline Para el cálculo correcto, registro contable y actualización de los impuestos asociados a las transacciones del sistema \\
\hline
HE-03 & Gestionar la información de terceros (clientes, proveedores y otras entidades) & Yo como administrador del sistema
\newline Quiero gestionar la información de terceros (clientes, proveedores y otras entidades)
\newline Para tener disponibilidad de los datos necesarios para las operaciones empresariales \\
\hline
HE-04 & Gestionar el catálogo de productos del inventario & Yo como administrador del sistema
\newline Quiero gestionar el catálogo de productos del inventario
\newline Para tener la información de productos actualizada y correctamente asociada a las cuentas contables e impuestos \\
\hline
HE-05 & Gestionar el centro de ayuda & Yo como administrador del sistema
\newline Quiero gestionar el centro de ayuda
\newline Para facilitar la comprensión y el uso del sistema \\
\hline
HE-06 & Gestionar los métodos de pago & Yo como administrador del sistema
\newline Quiero gestionar los métodos de pago
\newline Para automatizar los registros contables en comprobantes de egreso y recibos de caja \\
\hline
HE-07 & Gestionar los tipos de documento & Yo como administrador del sistema
\newline Quiero gestionar los tipos de documento
\newline Para una adecuada clasificación, flujo y registro contable de las operaciones \\
\hline
HE-08 & Gestionar los bancos y sus cuentas bancarias & Yo como administrador del sistema
\newline Quiero gestionar los bancos y sus cuentas bancarias
\newline Para permitir que las operaciones financieras reales se asocien correctamente con su representación contable \\
\hline
HE-09 & Gestionar los centros de costos & Yo como administrador del sistema
\newline Quiero gestionar los centros de costos
\newline Para segmentar las operaciones contables y presupuestales por áreas funcionales, unidades organizacionales o proyectos \\
\hline
HE-10 & Gestionar el calendario contable interactivo & Yo como administrador del sistema
\newline Quiero gestionar el calendario contable interactivo
\newline Para el registro adecuado de las operaciones en el periodo correspondiente \\
\hline
\end{longtable}



La \autoref{tab:historiasUsuario} muestra fragmentos de las historias de usuario correspondientes a los submódulos de Catálogo de cuentas, Impuestos, Terceros y Métodos de pago especificadas mediante un identificador único, un nombre, una descripción y sus respectivos criterios de aceptación. Para explorar la especificación completa, incluyendo las historias épicas y las historias de usuario detalladas, se debe consultar el siguiente \href{https://docs.google.com/spreadsheets/d/1rFsgFhqhBDgjrOPtcxpHFp3UVVghU8dv/edit?usp=sharing&ouid=114064353546583843217&rtpof=true&sd=true}{\underline{\textcolor{blue}{enlace}}}, en el cual cada criterio de aceptación tiene asociado su respectivo prototipo desarrollado en Figma.

\begin{longtable}{|p{1.5cm}|p{4cm}|p{4cm}|p{6cm}|}
\caption{Historias de Usuario}
\label{tab:historiasUsuario} \\
\hline
\textbf{ID} & \textbf{Nombre} & \textbf{Descripción} & \textbf{Criterios de aceptación} \\
\hline
\endfirsthead
\multicolumn{4}{c}{{\bfseries Tabla \thetable\ Continuación: Historias de Usuario}} \\
\hline
\textbf{ID} & \textbf{Nombre} & \textbf{Descripción} & \textbf{Criterios de aceptación} \\
\hline
\endhead
\hline
\endfoot
\endlastfoot
HE-01-HU07 & Activar/desactivar cuentas contables & Yo como administrador del sistema
\newline Quiero activar/desactivar las cuentas contables en el catálogo de cuentas
\newline Para controlar qué cuentas están disponibles para uso en las operaciones contables & CA-01: Dado que me encuentro en la interfaz configuración/catálogo de cuentas y he seleccionado una cuenta contable Cuando doy clic en activar
\newline Entonces el sistema deberá mostrar un mensaje claro indicando que la cuenta contable fue activada correctamente
\newline CA-02: Dado que me encuentro en la interfaz configuración/catálogo de cuentas y he seleccionado una cuenta contable Cuando doy clic en desactivar 
\newline Entonces el sistema deberá mostrar un mensaje claro indicando que la cuenta contable fue desactivada correctamente \\
\hline
HE-02-HU05 & Activar/desactivar tarifas de impuestos & Yo como administrador del sistema
\newline Quiero activar/desactivar las tarifas de impuestos
\newline Para controlar qué impuestos están disponibles para uso en las operaciones contables & CA-01: Dado que me encuentro en la interfaz configuración/impuestos y he seleccionado un impuesto 
\newline Cuando doy clic en activar 
\newline Entonces el sistema deberá mostrar un mensaje claro indicando que el impuesto fue activado correctamente
\newline CA-02: Dado que me encuentro en la interfaz configuración/impuestos y he seleccionado un impuesto 
\newline Cuando doy clic en desactivar 
\newline Entonces el sistema deberá mostrar un mensaje claro indicando que el impuesto fue desactivado correctamente \\
\hline
HE-03-HU04 & Activar/desactivar información de terceros & Yo como administrador del sistema
\newline Quiero activar/desactivar la información de terceros desde la interfaz del sistema
\newline Para conservar registros vigentes y relevantes & CA-01: Dado que me encuentro en la interfaz configuración / Terceros y he seleccionado un tercero existente de la lista 
\newline Cuando doy clic en Activar/Desactivar 
\newline Entonces el sistema deberá mostrar un mensaje indicando que el tercero fue activado/desactivado correctamente
\newline CA-02: Dado que me encuentro en la interfaz configuración / Terceros / Tipo de Identificación y he seleccionado un tipo de identificación existente de la lista \newline Cuando doy clic en Activar/Desactivar 
\newline Entonces el sistema deberá mostrar un mensaje indicando que el tipo de identificación fue activado/desactivado correctamente
\newline CA-03: Dado que me encuentro en la interfaz configuración / Terceros / Tipo de Tercero y he seleccionado un tipo de tercero existente de la lista
\newline Cuando doy clic en Activar/Desactivar 
\newline Entonces el sistema deberá mostrar un mensaje indicando que el tipo de tercero fue activado/desactivado correctamente \\
\hline
HE-06-HU05 & Activar/desactivar métodos de pago & Yo como administrador del sistema
\newline Quiero activar/desactivar métodos de pago
\newline Para conservar registros vigentes y relevantes & CA-01: Dado que me encuentro en la interfaz configuración / métodos de pago y he seleccionado un método de pago existente de la lista
\newline Cuando doy clic en Activar/Desactivar
\newline Entonces el sistema deberá mostrar un mensaje indicando que el metodo de pago fue activado/desactivado correctamente \\
\hline
\end{longtable}




\section{Definición de requisitos no funcionales}

Una vez definidos los requisitos funcionales y consignados en historias de usuario, es importante considerar también los requisitos no funcionales, los cuales van más allá de las funcionalidades y definen aspectos de calidad del sistema.

La selección y definición de estos requisitos es el resultado de un proceso de consenso colaborativo gestado durante la primera etapa del Aprendizaje Basado en Proyectos (PBL), correspondiente al desarrollo inicial de los módulos básicos. A través de mesas de trabajo conjuntas entre los docentes del programa de Contaduría Pública y los asesores de la División TIC de la Univerdad del Cauca, se identificaron las necesidades fundamentales para la viabilidad del software a largo plazo. En estas sesiones se determinó que, dada la naturaleza evolutiva de la herramienta académica y el crecimiento proyectado, el sistema requería una base estructural robusta. Esta visión estratégica fue el factor determinante para la elección de un estilo arquitectónico basado en microservicios, decisión que fundamenta y da origen a los siguientes atributos de calidad:

\begin{itemize}
\item \textbf{Usabilidad:} En el desarrollo del presente proyecto se tuvieron en cuenta los lineamientos de desarrollo, usabilidad y diseño establecidos por la División TIC de la Universidad del Cauca. Específicamente, los documentos de usabilidad y diseño incluyen parámetros que se reflejan en el uso de tipografía, colores, contrastes y enfoque de diseño adecuados para proporcionar una buena experiencia al usuario final. Por lo tanto, en el proyecto se recogen y aplican estas reglas mediante una interfaz que cumple, entre otras cosas, con las especificaciones de la División TIC y Material Design, sobre las cuales están basados dichos documentos.

\item \textbf{Escalabilidad:} Considerando que el número de usuarios puede crecer con el tiempo y que el volumen de información contable aumenta mes a mes, el sistema fue diseñado pensando en el futuro. La arquitectura de microservicios, seleccionada en las fases iniciales, permite que cuando haya más usuarios trabajando simultáneamente o cuando se procesen más transacciones, el sistema pueda ajustarse sin necesidad de hacer cambios drásticos en su estructura. Esto significa que si hay períodos de mayor actividad, la aplicación puede manejar esta carga adicional sin afectar su funcionamiento normal ni la experiencia de los usuarios.

\item \textbf{Mantenibilidad:}  Con el fin de facilitar la evolución y mejora continua del sistema, se estableció como requisito que el código fuera claro y bien organizado. Cada microservicio está documentado con su propósito y funcionamiento, lo que facilita la colaboración entre desarrolladores. Además, las pruebas automatizadas aseguran el correcto funcionamiento ante cambios reduciendo el riesgo de errores. Esta estructura permite corregir fallos, incorporar nuevas funcionalidades y mantener el sistema actualizado con las normativas contables vigentes sin complicaciones.

\end{itemize}

\section{Prototipos}

La construcción de prototipos fue esencial para una comunicación clara con el asesor empresarial, ya que al permitir visualizar un acercamiento de lo que serían las interfaces del módulo de Configuración se obtenía retroalimentación constante y rápida sobre aspectos pedagógicos y de usabilidad. Esto ha permitido realizar un refinamiento de la especificación de requerimientos en el formato de historias de usuario, este último también ha contribuido en sentido contrario a detallar más cuidadosamente los prototipos, generando así un círculo virtuoso de mejora continua.

Los prototipos se enfocaron en representar las interfaces de los submódulos que componen el módulo de Configuración, cada prototipo consideró los lineamientos de diseño UX-UI establecidos por la División TIC de la Universidad del Cauca, incluyendo la paleta de colores institucional y las fuentes tipográficas especificadas.

Para el diseño de los prototipos se hizo necesario el uso de una herramienta flexible y de fácil manejo, en la que se pudiera plasmar rápidamente el concepto de las interfaces del módulo de Configuración, la herramienta elegida fue Figma por su comodidad, flexibilidad y fácil manejo para el diseño de interfaces web, además de permitir la colaboración en tiempo real con los interesados del proyecto. Por tanto, si se desea explorar en su totalidad los prototipos desarrollados, es preciso dirigirse al proyecto Figma en el siguiente \href{https://www.figma.com/design/b28M2VlaXjz2JrX5TvlfwF/MODULO-CONFIGURACION?node-id=1092-13124}{\underline{\textcolor{blue}{enlace}}}, donde se encuentran compilados todos los diseños de las interfaces. Igualmente, se incluyen a continuación ejemplos seleccionados de los prototipos desarrollados.

\subsection{Vista de catálogo de cuentas}

La \autoref{fig:CatalogoCuentas} muestra el prototipo asociado al catálogo de cuentas, que establece la estructura maestra de clasificación contable donde se definen, organizan y gestionan todas las cuentas que registrarán las transacciones financieras del sistema. Configura el esquema contable maestro del sistema a través de una interfaz que permite gestionar jerárquicamente la estructura de clasificación financiera. Desde este módulo central, el administrador puede crear, importar o exportar cuentas codificadas numéricamente del 1 al 9, organizadas según naturaleza contable (Activo, Pasivos, Patrimonio, Ingresos, Gastos, Costos de transformación y Cuentas de orden), con control de estado activo/inactivo para mantener vigente el plan de cuentas.

\begin{figure}[h!]    
    \centering%
    \includegraphics[width=1.0\textwidth, height=1.0\textheight, keepaspectratio]{Cap4/Figuras/catalogo de cuentas.png}    
    \caption{Prototipo vista catálogo de cuentas.}
    \label{fig:CatalogoCuentas}
\end{figure}

\subsection{Vista de inventario de productos}

La \autoref{fig:InventarioProductos} muestra el prototipo asociado al inventario de productos establece el catálogo maestro de artículos donde se centraliza la gestión de insumos, productos terminados y mercancías a través de una interfaz que permite crear, importar, exportar y administrar registros con atributos clave como referencia, nombre, descripción, costo, cantidad, unidad de medida, categoría, tipo de producto y estado activo.

\begin{figure}[h!]    
    \centering%
    \includegraphics[width=1.0\textwidth, height=1.0\textheight, keepaspectratio]{Cap4/Figuras/inventario de productos.png}    
    \caption{Prototipo vista inventario de productos.}
    \label{fig:InventarioProductos}
\end{figure}

\newpage
\subsection{Vista de terceros}

La \autoref{fig:tercero} muestra el prototipo asociado a terceros,  configura el registro maestro de entidades externas a través de una interfaz que permite crear, importar, exportar y administrar personas naturales y jurídicas, capturando atributos esenciales como identificación, tipo de entidad, correo electrónico y estado activo. Incorpora la funcionalidad específica de procesar archivos PDF del RUT para automatizar la creación de registros, validando duplicados y formatos de datos requeridos. Este módulo opera como repositorio centralizado que estructura y mantiene actualizada la base de datos de clientes, proveedores y otros actores, habilitando búsquedas dinámicas por nombre o numero de identificación.

\begin{figure}[h!]    
    \centering%
    \includegraphics[width=1.0\textwidth, height=1.0\textheight, keepaspectratio]{Cap4/Figuras/terceros.png}    
    \caption{Prototipo vista de terceros.}
    \label{fig:tercero}
\end{figure}


\newpage
\subsection{Vista de impuestos }

La \autoref{fig:impuestos} muestra el prototipo asociado a impuestos encargado de la configuracion maestra de tarifas mediante una interfaz que permite crear, modificar y eliminar registros, asignando descripciones, porcentajes y las cuentas contables específicas para compras y ventas. Centraliza la parametrización de tributos como IVA y retenciones, validando la existencia de las cuentas asociadas.

\begin{figure}[h!]    
    \centering%
    \includegraphics[width=1.0\textwidth, height=1.0\textheight, keepaspectratio]{Cap4/Figuras/impuestos.png}    
    \caption{Prototipo vista de impuestos.}
    \label{fig:impuestos}
\end{figure}