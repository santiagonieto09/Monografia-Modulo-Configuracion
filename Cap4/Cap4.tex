\mychapter{Capítulo 4. Ingeniería de requisitos}
\label{cap:IngRequisitos} 
La ingeniería de requisitos es una etapa fundamental en un proyecto de software, ya que permite identificar, analizar y documentar las necesidades de un sistema antes de su implementación. Un levantamiento adecuado de requisitos permite que el resultado cumpla las expectativas del cliente. Es por esto que en este capítulo se describe el proceso llevado a cabo para la especificación de los requerimientos funcionales y no funcionales del módulo de configuración del ``Software contable para uso educativo en el programa de Contaduría Pública de la Universidad del Cauca''.

Durante la ingeniería de requisitos llevada a cabo en el proyecto se realizaron diferentes actividades con el fin de obtener una lista de tareas congruente con los requisitos del software contable educativo. Para hacer posible este proceso fue fundamental la comunicación constante con los principales interesados, docentes del programa de Contaduría Pública, el asesor empresarial Mauro Andrés Sánchez Muñoz y el director de la práctica profesional PhD. Wilson Libardo Pantoja Yepez, quienes muy amablemente reservaron espacios en sus agendas para atender las inquietudes resultantes. Esto permitió una retroalimentación constante y sumado a la flexibilidad de la metodología de trabajo híbrida adoptada, fue posible ir ajustando los requisitos en cada encuentro, incluso en etapas posteriores del desarrollo.

También se debe tener en cuenta que el análisis previo descrito en el ``Capítulo 3. Descripción de los submódulos a implementar'', en el cual se identificó los submódulos que componen el módulo de configuración, permitió tener un contexto más amplio sobre los procesos y actividades involucradas en la gestión contable educativa. Lo anterior claramente benefició la recolección y especificación de requerimientos, dado que fue posible avanzar con mayor seguridad y velocidad en la definición de funcionalidades para cada submódulo.

Sin embargo, a pesar de la disposición de los interesados, se presentaron dificultades relacionadas con la coordinación de horarios y las múltiples ocupaciones de los participantes. En ocasiones, los docentes del programa de Contaduría Pública no disponían de tiempo suficiente para el proyecto debido a sus compromisos académicos, evaluaciones y actividades administrativas. A pesar de ello, se logró ajustar las actividades y el trabajo requerido para el levantamiento de requisitos, superando los inconvenientes mediante la flexibilidad horaria característica del enfoque freelance adoptado.

\section{Actividades desarrolladas para la ingeniería de requisitos}
Las actividades desarrolladas para llevar a cabo la ingeniería de requisitos para este proyecto son las siguientes.

\subsection{Reuniones con los interesados}
Como previamente se ha mencionado, durante el proyecto fue de suma importancia los encuentros presenciales y virtuales con los interesados, donde se resolvieron dudas sobre aspectos contables y pedagógicos y se obtuvo retroalimentación constante la cual moldeaba el trabajo que se realizaba. También en ocasiones se compartió documentación sobre las actividades llevadas a cabo en el programa de Contaduría Pública, como planes de estudio, casos de uso contables y ejemplos de ejercicios prácticos útiles para el proceso. Se debe notar que, dada la naturaleza técnica del proyecto, la comunicación presencial y por videoconferencia resultó más efectiva que los medios digitales asíncronos para la discusión de conceptos contables complejos.

\subsection{Construcción de prototipos}
La construcción de prototipos fue fundamental para no dirigir el diálogo con los docentes de forma exclusivamente técnica, sino establecer una conversación apoyada visualmente por los prototipos de interfaz. Los interesados aportaron sus comentarios sobre cambios en las pantallas de gestión de cuentas contables, terceros, productos, impuestos y demás submódulos identificados anteriormente, evaluando su adecuación para el entorno educativo. Asimismo se realizaron negociaciones sobre la complejidad de las funcionalidades según el nivel académico de los estudiantes. Los cambios producto de estas negociaciones eran presentados como apertura en la siguiente reunión para poder continuar el proceso iterativo.

\subsection{Definición de historias de usuario}
La conversación con los docentes se llevó a cabo de forma pedagógica y orientada a casos de uso contables reales, como se mencionó anteriormente. Sin embargo, para el proceso de desarrollo del proyecto es fundamental tener documentación estructurada que defina claramente las funcionalidades que requiere el módulo de configuración del software contable educativo. Es por esto que las historias de usuario como artefacto para el levantamiento de requerimientos fueron realizadas de forma paralela a la construcción de prototipos, considerando tanto los aspectos técnicos del desarrollo como los objetivos pedagógicos del programa de Contaduría Pública, estas historias fueron ajustadas según la retroalimentación brindada por el cliente en cada encuentro, asegurando que cada submódulo cumpliera con los requisitos educativos específicos.

\subsection{Validación de prototipos}
Como ya fue mencionado, la validación de prototipos se realizó en la apertura de cada encuentro con el cliente, de modo que se tuviera seguridad de que las interfaces diseñadas fueran intuitivas para estudiantes de contabilidad y cumplieran con los lineamientos de usabilidad establecidos por la División TIC de la Universidad del Cauca. Se evaluó especialmente la claridad de los formularios de gestión de cada submódulo y la coherencia con los procesos contables estándar.

\subsection{Validación de historias de usuario}
La validación de las historias de usuario se realizó en conjunto con la validación de prototipos, estas fueron ajustadas en cada encuentro, considerando tanto los aspectos técnicos como los requisitos pedagógicos específicos del programa de Contaduría Pública. En este proceso, se prestó especial atención a que cada funcionalidad contribuyera efectivamente al aprendizaje de conceptos contables fundamentales y permitiera la simulación de escenarios empresariales reales.

\section{Identificación de stakeholders}
Para identificar los stakeholders del proyecto fue fundamental la colaboración con el asesor empresarial y el director de la práctica profesional. Es por esto que se logra la identificación de los interesados como a continuación se describe:

\begin{table}[h]
\centering
\caption{Stakeholders del proyecto}
\label{tab:stakeholders}
\begin{tabular}{|p{6cm}|p{8cm}|}
\hline
\textbf{Interesado} & \textbf{Cargo} \\
\hline
Julio Ariel Hurtado Alegria & Asesor de la Universidad del Cauca \\
\hline
Wilson Libardo Pantoja & Asesor de la Universidad del Cauca \\
\hline
Juan Ignacio Oviedo & Asesor de la organización \\
\hline
Viviana Narvaez & Asesor de la organización \\
\hline
Mauro Sanchez & Asesor de la organización \\
\hline
Omar Gomez Gomez & Usuario Final Experto \\
\hline
Brayan Neil Vargas & Usuario Final Experto \\
\hline
Comunidad universitaria & Estudiantes administrativos y docentes \\
\hline
\end{tabular}
\end{table}
\newpage

\section{Prototipos}

La construcción de prototipos fue esencial para una comunicación clara con el asesor empresarial, ya que al permitir visualizar un acercamiento de lo que serían las interfaces del módulo de configuración se obtenía retroalimentación constante y rápida sobre aspectos pedagógicos y de usabilidad. Esto ha permitido realizar un refinamiento de la especificación de requerimientos en el formato de historias de usuario, este último también ha contribuido en sentido contrario a detallar más cuidadosamente los prototipos, generando así un círculo virtuoso de mejora continua.

Los prototipos se enfocaron en representar las interfaces de los submódulos que componen el módulo de configuración, cada prototipo consideró los lineamientos de diseño UX-UI establecidos por la División TIC de la Universidad del Cauca, incluyendo la paleta de colores institucional y las fuentes tipográficas especificadas.

Para el diseño de los prototipos se hizo necesario el uso de una herramienta flexible y de fácil manejo, en la que se pudiera plasmar rápidamente el concepto de las interfaces del módulo de configuración, la herramienta elegida fue Figma por su comodidad, flexibilidad y fácil manejo para el diseño de interfaces web, además de permitir la colaboración en tiempo real con los interesados del proyecto. Por tanto, si se desea explorar en su totalidad los prototipos desarrollados, es preciso dirigirse al proyecto Figma en el siguiente \href{https://www.figma.com/design/b28M2VlaXjz2JrX5TvlfwF/MODULO-CONFIGURACION?node-id=1092-13124}{\underline{\textcolor{blue}{enlace}}}, donde se encuentran compilados todos los diseños de las interfaces del módulo de configuración.

Igualmente, se incluyen a continuación ejemplos seleccionados de los prototipos desarrollados, evitando así saturar el documento y optimizando la comprensión visual.

\subsection{Vista de catálogo de cuentas}

La figura \ref{fig:CatalogoCuentas} muestra el prototipo asociado al catálogo de cuentas, que establece la estructura maestra de clasificación contable donde se definen, organizan y gestionan todas las cuentas que registrarán las transacciones financieras del sistema. Configura el esquema contable maestro del sistema a través de una interfaz que permite gestionar jerárquicamente la estructura de clasificación financiera. Desde este módulo central, el administrador puede crear, importar o exportar cuentas codificadas numéricamente del 1 al 9, organizadas según naturaleza contable (Activo, Pasivos, Patrimonio, Ingresos, Gastos, Costos de transformación y Cuentas de orden), con control de estado activo/inactivo para mantener vigente el plan de cuentas.

\begin{figure}[h!]    
    \centering%
    \includegraphics[width=1.0\textwidth, height=1.0\textheight, keepaspectratio]{Cap4/Figuras/catalogo de cuentas.png}    
    \caption{Prototipo vista catálogo de cuentas.}
    \label{fig:CatalogoCuentas}
\end{figure}

\newpage
\subsection{Vista de inventario de productos}

La figura \ref{fig:InventarioProductos} muestra el prototipo asociado al inventario de productos establece el catálogo maestro de artículos donde se centraliza la gestión de insumos, productos terminados y mercancías a través de una interfaz que permite crear, importar, exportar y administrar registros con atributos clave como referencia, nombre, descripción, costo, cantidad, unidad de medida, categoría, tipo de producto y estado activo.

\begin{figure}[h!]    
    \centering%
    \includegraphics[width=1.0\textwidth, height=1.0\textheight, keepaspectratio]{Cap4/Figuras/inventario de productos.png}    
    \caption{Prototipo vista inventario de productos.}
    \label{fig:InventarioProductos}
\end{figure}

\newpage
\subsection{Vista de terceros}

La figura \ref{fig:tercero} muestra el prototipo asociado a terceros,  configura el registro maestro de entidades externas a través de una interfaz que permite crear, importar, exportar y administrar personas naturales y jurídicas, capturando atributos esenciales como identificación, tipo de entidad, correo electrónico y estado activo. Incorpora la funcionalidad específica de procesar archivos PDF del RUT para automatizar la creación de registros, validando duplicados y formatos de datos requeridos. Este módulo opera como repositorio centralizado que estructura y mantiene actualizada la base de datos de clientes, proveedores y otros actores, habilitando búsquedas dinámicas por nombre o numero de identificación.

\begin{figure}[h!]    
    \centering%
    \includegraphics[width=1.0\textwidth, height=1.0\textheight, keepaspectratio]{Cap4/Figuras/terceros.png}    
    \caption{Prototipo vista de terceros.}
    \label{fig:tercero}
\end{figure}


\newpage
\subsection{Vista de impuestos }

La figura \ref{fig:impuestos} muestra el prototipo asociado a impuestos encargado de la configuracion maestra de tarifas mediante una interfaz que permite crear, modificar y eliminar registros, asignando descripciones, porcentajes y las cuentas contables específicas para compras y ventas. Centraliza la parametrización de tributos como IVA y retenciones, validando la existencia de las cuentas asociadas.

\begin{figure}[h!]    
    \centering%
    \includegraphics[width=1.0\textwidth, height=1.0\textheight, keepaspectratio]{Cap4/Figuras/impuestos.png}    
    \caption{Prototipo vista de impuestos.}
    \label{fig:impuestos}
\end{figure}


\section{Requisitos funcionales}

La tabla \ref{tab:listaTareas} refleja las necesidades identificadas y se estructura mediante historias épicas, las cuales a su vez se desglosan en historias de usuario. Las historias épicas agrupan funcionalidades o tareas complejas que no pueden abordarse en una sola iteración, mientras que las historias de usuario detallan estas épicas en descripciones manejables que permiten su desarrollo independiente en una iteración.

\begin{longtable}{|p{2cm}|p{5cm}|p{3cm}|p{6cm}|}
\caption{Lista de tareas del proyecto: historias épicas y de usuario}
\label{tab:listaTareas} \\
\hline
\multicolumn{2}{|c|}{Épica} & \multicolumn{2}{c|}{Historia de Usuario Asociada} \\
\hline
\endfirsthead
\multicolumn{4}{c}{{\bfseries Tabla \thetable\ Continuación: Lista de tareas del proyecto}} \\
\hline
\multicolumn{2}{|c|}{Épica} & \multicolumn{2}{c|}{Historia de Usuario Asociada} \\
\hline
\endhead
\hline
\endfoot
\endlastfoot
ID Épica & Nombre Épica & ID Historia & Nombre Historia \\
\hline
\multirow{7}{2cm}{HE-01} & \multirow{7}{5cm}{Gestionar el catálogo de cuentas} & HE-01-HU01 & Crear cuentas contables \\
 &  & HE-01-HU02 & Modificar cuentas contables \\
 &  & HE-01-HU03 & Eliminar cuentas contables \\
 &  & HE-01-HU04 & Importar catálogo de cuentas desde archivo plano \\
 &  & HE-01-HU05 & Exportar catálogo de cuentas en un archivo plano \\
 &  & HE-01-HU06 & Consultar cuentas contables \\
 &  & HE-01-HU07 & Activar/Desactivar cuentas contables \\
\hline
\multirow{5}{2cm}{HE-02} & \multirow{5}{5cm}{Gestionar las tarifas de los impuestos} & HE-02-HU01 & Crear nuevas tarifas de impuestos \\
 &  & HE-02-HU02 & Modificar tarifas de impuestos \\
 &  & HE-02-HU03 & Eliminar tarifas de impuestos \\
 &  & HE-02-HU04 & Consultar tarifas de impuestos \\
 &  & HE-02-HU05 & Activar/Desactivar tarifas de impuestos \\
\hline
\multirow{7}{2cm}{HE-03} & \multirow{7}{5cm}{Gestionar la información de terceros} & HE-03-HU01 & Crear terceros \\
 &  & HE-03-HU02 & Crear terceros desde PDF (RUT) \\
 &  & HE-03-HU03 & Modificación de terceros \\
 &  & HE-03-HU04 & Activar/Desactivar terceros \\
 &  & HE-03-HU05 & Importar terceros desde archivo plano \\
 &  & HE-03-HU06 & Exportar terceros en un archivo plano \\
 &  & HE-03-HU07 & Consultar terceros \\
\hline
\multirow{6}{2cm}{HE-04} & \multirow{6}{5cm}{Gestionar el catálogo de productos del inventario} & HE-04-HU01 & Crear productos en el inventario \\
 &  & HE-04-HU02 & Modificar productos en el inventario \\
 &  & HE-04-HU03 & Eliminar productos en el inventario \\
 &  & HE-04-HU04 & Importar productos desde archivo plano \\
 &  & HE-04-HU05 & Exportar productos en un archivo plano \\
 &  & HE-04-HU06 & Consultar productos en el inventario \\
\hline
\multirow{5}{2cm}{HE-05} & \multirow{5}{5cm}{Gestionar el centro de ayuda} & HE-05-HU01 & Crear centros de ayuda explicativos \\
 &  & HE-05-HU02 & Modificar contenido en centros de ayuda \\
 &  & HE-05-HU03 & Eliminar centros de ayuda \\
 &  & HE-05-HU04 & Consultar centros de ayuda \\
 &  & HE-05-HU05 & Activar/Desactivar centros de ayuda \\
\hline
\multirow{5}{2cm}{HE-06} & \multirow{5}{5cm}{Gestionar los métodos de pago} & HE-06-HU01 & Crear métodos de pago \\
 &  & HE-06-HU02 & Modificar métodos de pago \\
 &  & HE-06-HU03 & Eliminar métodos de pago \\
 &  & HE-06-HU04 & Consultar métodos de pago \\
 &  & HE-06-HU05 & Activar/Desactivar métodos de pago \\
\hline
\multirow{5}{2cm}{HE-07} & \multirow{5}{5cm}{Gestionar los tipos de documento} & HE-07-HU01 & Crear tipos de documento \\
 &  & HE-07-HU02 & Modificar tipos de documento \\
 &  & HE-07-HU03 & Eliminar tipos de documento \\
 &  & HE-07-HU04 & Consultar tipos de documentos \\
 &  & HE-07-HU05 & Activar/Desactivar tipos de documentos \\
\hline
\multirow{5}{2cm}{HE-08} & \multirow{5}{5cm}{Gestionar los bancos y sus cuentas bancarias} & HE-08-HU01 & Crear bancos \\
 &  & HE-08-HU02 & Modificar bancos \\
 &  & HE-08-HU03 & Eliminar bancos \\
 &  & HE-08-HU04 & Consultar bancos \\
 &  & HE-08-HU05 & Activar/Desactivar bancos \\
\hline
\multirow{5}{2cm}{HE-09} & \multirow{5}{5cm}{Gestionar los centros de costos} & HE-09-HU01 & Crear centros de costo \\
 &  & HE-09-HU02 & Modificar centros de costo \\
 &  & HE-09-HU03 & Eliminar centros de costo \\
 &  & HE-09-HU04 & Consultar centros de costo \\
 &  & HE-09-HU05 & Activar/Desactivar centros de costo \\
\hline
\multirow{4}{2cm}{HE-10} & \multirow{4}{5cm}{Gestionar el calendario contable interactivo} & HE-10-HU01 & Abrir/cerrar año contable \\
 &  & HE-10-HU02 & Abrir/cerrar mes contable \\
 &  & HE-10-HU03 & Abrir/cerrar fecha específica \\
 &  & HE-10-HU04 & Consultar periodos contables \\
\hline
\end{longtable}


La tabla \ref{tab:epicas} presenta las historias épicas, detallando su especificación. Cada épica incluye un identificador único, un nombre descriptivo y su descripción completa en formato de historia de usuario.

\begin{longtable}{|p{2cm}|p{6cm}|p{6cm}|}
\caption{Historias de Épicas}
\label{tab:epicas} \\
\hline
\textbf{ID Épica} & \textbf{Nombre} & \textbf{Descripción} \\
\hline
\endfirsthead

\multicolumn{3}{c}{{\bfseries Tabla \thetable\ Continuación: Historias de Épicas}} \\
\hline
\textbf{ID Épica} & \textbf{Nombre} & \textbf{Descripción} \\
\hline
\endhead
\hline
\endfoot
\endlastfoot
HE-01 & Gestionar el catálogo de cuentas & Yo como administrador del sistema
\newline Quiero gestionar el catálogo de cuentas
\newline Para la correcta administración, actualización y disponibilidad de la información contable necesaria \\
\hline
HE-02 & Gestionar las tarifas de los impuestos & Yo como administrador del sistema
\newline Quiero gestionar las tarifas de los impuestos
\newline Para el cálculo correcto, registro contable y actualización de los impuestos asociados a las transacciones del sistema \\
\hline
HE-03 & Gestionar la información de terceros (clientes, proveedores y otras entidades) & Yo como administrador del sistema
\newline Quiero gestionar la información de terceros (clientes, proveedores y otras entidades)
\newline Para tener disponibilidad de los datos necesarios para las operaciones empresariales \\
\hline
HE-04 & Gestionar el catálogo de productos del inventario & Yo como administrador del sistema
\newline Quiero gestionar el catálogo de productos del inventario
\newline Para tener la información de productos actualizada y correctamente asociada a las cuentas contables e impuestos \\
\hline
HE-05 & Gestionar el centro de ayuda & Yo como administrador del sistema
\newline Quiero gestionar el centro de ayuda
\newline Para facilitar la comprensión y el uso del sistema \\
\hline
HE-06 & Gestionar los métodos de pago & Yo como administrador del sistema
\newline Quiero gestionar los métodos de pago
\newline Para automatizar los registros contables en comprobantes de egreso y recibos de caja \\
\hline
HE-07 & Gestionar los tipos de documento & Yo como administrador del sistema
\newline Quiero gestionar los tipos de documento
\newline Para una adecuada clasificación, flujo y registro contable de las operaciones \\
\hline
HE-08 & Gestionar los bancos y sus cuentas bancarias & Yo como administrador del sistema
\newline Quiero gestionar los bancos y sus cuentas bancarias
\newline Para permitir que las operaciones financieras reales se asocien correctamente con su representación contable \\
\hline
HE-09 & Gestionar los centros de costos & Yo como administrador del sistema
\newline Quiero gestionar los centros de costos
\newline Para segmentar las operaciones contables y presupuestales por áreas funcionales, unidades organizacionales o proyectos \\
\hline
HE-10 & Gestionar el calendario contable interactivo & Yo como administrador del sistema
\newline Quiero gestionar el calendario contable interactivo
\newline Para el registro adecuado de las operaciones en el periodo correspondiente \\
\hline
\end{longtable}


La tabla \ref{tab:historiasUsuario} muestra un fragmento de las historias de usuario donde su especificación está dada por un código, este código está compuesto con el ID de la historia épica a la que pertenece, posee un nombre, una descripción y los criterios de aceptación para la historia de usuario.

\begin{longtable}{|p{1.5cm}|p{4cm}|p{4cm}|p{6cm}|}
\caption{Historias de Usuario}
\label{tab:historiasUsuario} \\
\hline
\textbf{ID} & \textbf{Nombre} & \textbf{Descripción} & \textbf{Criterios de aceptación} \\
\hline
\endfirsthead
\multicolumn{4}{c}{{\bfseries Tabla \thetable\ Continuación: Historias de Usuario}} \\
\hline
\textbf{ID} & \textbf{Nombre} & \textbf{Descripción} & \textbf{Criterios de aceptación} \\
\hline
\endhead
\hline
\endfoot
\endlastfoot
HE-01-HU07 & Activar/desactivar cuentas contables & Yo como administrador del sistema
\newline Quiero activar/desactivar las cuentas contables en el catálogo de cuentas
\newline Para controlar qué cuentas están disponibles para uso en las operaciones contables & CA-01: Dado que me encuentro en la interfaz configuración/catálogo de cuentas y he seleccionado una cuenta contable Cuando doy clic en activar
\newline Entonces el sistema deberá mostrar un mensaje claro indicando que la cuenta contable fue activada correctamente
\newline CA-02: Dado que me encuentro en la interfaz configuración/catálogo de cuentas y he seleccionado una cuenta contable Cuando doy clic en desactivar 
\newline Entonces el sistema deberá mostrar un mensaje claro indicando que la cuenta contable fue desactivada correctamente \\
\hline
HE-02-HU05 & Activar/desactivar tarifas de impuestos & Yo como administrador del sistema
\newline Quiero activar/desactivar las tarifas de impuestos
\newline Para controlar qué impuestos están disponibles para uso en las operaciones contables & CA-01: Dado que me encuentro en la interfaz configuración/impuestos y he seleccionado un impuesto 
\newline Cuando doy clic en activar 
\newline Entonces el sistema deberá mostrar un mensaje claro indicando que el impuesto fue activado correctamente
\newline CA-02: Dado que me encuentro en la interfaz configuración/impuestos y he seleccionado un impuesto 
\newline Cuando doy clic en desactivar 
\newline Entonces el sistema deberá mostrar un mensaje claro indicando que el impuesto fue desactivado correctamente \\
\hline
HE-03-HU04 & Activar/desactivar información de terceros & Yo como administrador del sistema
\newline Quiero activar/desactivar la información de terceros desde la interfaz del sistema
\newline Para conservar registros vigentes y relevantes & CA-01: Dado que me encuentro en la interfaz configuración / Terceros y he seleccionado un tercero existente de la lista 
\newline Cuando doy clic en Activar/Desactivar 
\newline Entonces el sistema deberá mostrar un mensaje indicando que el tercero fue activado/desactivado correctamente
\newline CA-02: Dado que me encuentro en la interfaz configuración / Terceros / Tipo de Identificación y he seleccionado un tipo de identificación existente de la lista \newline Cuando doy clic en Activar/Desactivar 
\newline Entonces el sistema deberá mostrar un mensaje indicando que el tipo de identificación fue activado/desactivado correctamente
\newline CA-03: Dado que me encuentro en la interfaz configuración / Terceros / Tipo de Tercero y he seleccionado un tipo de tercero existente de la lista
\newline Cuando doy clic en Activar/Desactivar 
\newline Entonces el sistema deberá mostrar un mensaje indicando que el tipo de tercero fue activado/desactivado correctamente \\
\hline
HE-06-HU05 & Activar/desactivar métodos de pago & Yo como administrador del sistema
\newline Quiero activar/desactivar métodos de pago
\newline Para conservar registros vigentes y relevantes & CA-01: Dado que me encuentro en la interfaz configuración / métodos de pago y he seleccionado un método de pago existente de la lista
\newline Cuando doy clic en Activar/Desactivar
\newline Entonces el sistema deberá mostrar un mensaje indicando que el metodo de pago fue activado/desactivado correctamente \\
\hline
\end{longtable}


Para explorar con mayor detalle, es preciso dirigirse al siguiente \href{https://docs.google.com/spreadsheets/d/1rFsgFhqhBDgjrOPtcxpHFp3UVVghU8dv/edit?usp=sharing&ouid=114064353546583843217&rtpof=true&sd=true}{\underline{\textcolor{blue}{enlace}}}, el cual presenta una estructura organizada donde se detallan tanto las historias épicas como las historias de usuario, donde cada criterio de aceptación de las historias de usuario tiene asociado su respectivo prototipo en Figma.

\section{Requisitos no funcionales}

Una vez definidos los requisitos funcionales y consignados en historias de usuario que conforman la lista de tareas, es importante considerar también los requisitos no funcionales, los cuales van más allá de las funcionalidades y definen aspectos de calidad del sistema.

Para la identificación de estos requisitos fue esencial la comunicación con el principal interesado del sistema y los aspectos normativos que rigen la construcción de software desde la División TIC de la Universidad. Teniendo en cuenta esto, se han identificado los siguientes requisitos no funcionales:

\begin{itemize}
\item \textbf{Usabilidad:} En el desarrollo del presente proyecto se tuvieron en cuenta los lineamientos de desarrollo, usabilidad y diseño establecidos por la División TIC de la Universidad del Cauca. Específicamente, los documentos de usabilidad y diseño incluyen parámetros que se reflejan en el uso de tipografía, colores, contrastes y enfoque de diseño adecuados para proporcionar una buena experiencia al usuario final. Por lo tanto, en el proyecto se recogen y aplican estas reglas mediante una interfaz que cumple, entre otras cosas, con las especificaciones de la División TIC y Material Design, sobre las cuales están basados dichos documentos.

\item \textbf{Flexibilidad:} El módulo de configuración está diseñado como un microservicio independiente dentro del ecosistema del software contable educativo, registrado en Eureka Server para descubrimiento dinámico de servicios. Esta arquitectura permite escalar horizontalmente mediante el despliegue de múltiples instancias según la demanda, actualizar el módulo de forma independiente sin afectar otros componentes del sistema y facilita el aislamiento de fallos evitando que errores en este microservicio comprometan todo el sistema contable. Además, la arquitectura hexagonal en los módulos de terceros, catálogo de cuentas y productos, proporciona flexibilidad tecnológica al permitir cambiar implementaciones de persistencia o incorporar nuevos adaptadores sin modificar el dominio.
\end{itemize}