\mychapter{Capítulo 5. Arquitectura del sistema}

En el presente capítulo se explora la arquitectura del sistema, el cual es un plan que define aspectos y decisiones importantes para el software, teniendo en cuenta los requisitos del sistema, su organización y la comunicación entre los diferentes componentes.

---AQUI FALTA---

Para describir la arquitectura del sistema se utiliza el modelo C4, el cual establece el uso de cuatro diagramas principales: Contexto, Contenedores, Componentes y Código. Por lo tanto, a continuación se describe la arquitectura del sistema mediante estos diagramas.

\section{Modelo C4: Descripción de la arquitectura}

\subsection{Contexto}

En la figura se puede observar el diagrama de contexto del sistema del software contable educativo, siendo la representación visual más general del sistema, busca describir la relación con los usuarios y sistemas externos.

\subsubsection{Usuarios}
\begin{itemize}
\item \textbf{Usuario aplicación:} Hace referencia a aquellos vinculados al programa de Contaduría Pública de la Universidad del Cauca de una u otra forma. Incluye estudiantes que toman cursos de contabilidad, docentes que imparten clases y el administrador del sistema. Estos usuarios no solo pueden visualizar la información de la oferta académica del programa, sino que tienen acceso a las funcionalidades del software contable educativo según su rol. Los roles principales en el sistema son:
  \begin{itemize}
  \item \textbf{Administrador:} Se encarga de la administración del sistema, con privilegios elevados para crear, eliminar y modificar información y elementos del módulo de configuración.
  \item \textbf{Profesor:} Se encarga de dirigir los grupos de estudiantes, con privilegios para manipular las actividades y evaluaciones de sus alumnos en el contexto educativo.
  \item \textbf{Estudiante:} Estudiantes matriculados en el programa de Contaduría Pública que se han registrado en el sistema para practicar conceptos contables.
  \end{itemize}
\end{itemize}
\subsubsection{Sistemas externos}