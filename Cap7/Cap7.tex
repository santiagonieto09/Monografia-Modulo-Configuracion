\mychapter{Capítulo 7. Evaluación de Funcionalidades}

En este capítulo se describen las pruebas realizadas al módulo de configuración, con el objetivo de evaluar sus funcionalidades y verificar su correcto funcionamiento e integración con el resto del sistema.


\section{Pruebas del Backend}

\subsection{Pruebas unitarias}

Las pruebas unitarias desempeñan un papel fundamental en el desarrollo de software, permitiendo validar el comportamiento correcto de componentes individuales de manera independiente. En este proyecto, se aplicaron pruebas unitarias mediante JUnit 5 y Mockito, que juntas simplifican la elaboración de tests automatizados. JUnit 5 ofrece la estructura necesaria para diseñar, ordenar y llevar a cabo las pruebas, en tanto que Mockito simula las interacciones con dependencias externas a través de mocks, asegurando un aislamiento total de cada elemento en su verificación. Esta sinergia tecnológica garantiza que cada módulo cumpla con las especificaciones requeridas antes de su combinación con otras partes del sistema. Con el fin de lograr pruebas veloces y efectivas, se descartó el empleo de entornos complejos como Spring Boot, centrándose en examinar las piezas de código de manera autónoma. Esta estrategia proporciona una respuesta inmediata durante la fase de desarrollo y ayuda a detectar fallos de forma precoz.

