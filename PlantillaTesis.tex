\documentclass[11pt,spanish,fleqn,openany,letterpaper,pagesize]{scrbook}
\usepackage[utf8]{inputenc}
\usepackage[spanish]{babel}%escribir con acentos sin necesidad de comandos \'{} .
\usepackage[margin=2.5cm]{geometry} % Márgenes de 2.5cm en todos los lados
\usepackage{fancyhdr}
\usepackage{epsfig}
\usepackage{epic}
\usepackage{eepic}
\usepackage{amsmath}
\usepackage{threeparttable}
\usepackage{amscd}
\usepackage{here}
\usepackage{graphicx}
\usepackage{lscape}
\usepackage{tabularx}
\usepackage{subcaption}
\usepackage{longtable}
\usepackage{multirow}
\usepackage{calligra} 
\usepackage[T1]{fontenc}
\usepackage{helvet} % Fuente Helvetica (muy similar a Arial)
\renewcommand{\familydefault}{\sfdefault} % Usa sans-serif (Helvetica) como fuente principal
\usepackage{listings} % Para mostrar codigo fuente en C, R, java
%\usepackage{natbib} %para las referencias plain
\usepackage[backend=bibtex,style=numeric]{biblatex}

\addbibresource{Bibliografia.bib}
\usepackage{rotating} %Para rotar texto, objetos y tablas seite. No se ve en DVI solo en PS. Seite 328 Hundebuch
                        %se usa junto con \rotate, \sidewidestable ....

% CONFIGURACIÓN DE HYPERREF (debe ir casi al final del preámbulo)
\usepackage{hyperref}
\hypersetup{
    colorlinks=true,
    linkcolor=black,
    citecolor=black,
    filecolor=black,
    urlcolor=black,
    pdftitle={Tu título aquí},
    pdfauthor={Tu nombre aquí},
}

% REDEFINICIONES DE NUMERACIÓN (después de hyperref)
\renewcommand{\theequation}{\thechapter-\arabic{equation}}
\renewcommand{\thefigure}{\thechapter-\arabic{figure}}
\renewcommand{\thetable}{\thechapter-\arabic{table}}

% CONFIGURACIÃ"N DE PÃGINA
\pagestyle{fancyplain}%\addtolength{\headwidth}{\marginparwidth}
\renewcommand{\headrulewidth}{0pt}
\renewcommand{\chaptermark}[1]{\markboth{\thechapter\; #1}{}}
\renewcommand{\sectionmark}[1]{\markright{\thesection\; #1}}
\lhead[\fancyplain{}{}]{\fancyplain{}{}}
\rhead[\fancyplain{}{}]{\fancyplain{}{}}
\fancyfoot[C]{\thepage} % Número de página centrado en el pie
\thispagestyle{fancy}%
\addtolength{\headwidth}{0cm}
\unitlength1mm %Define la unidad LE para Figuras
\mathindent0cm %Define la distancia de las formulas al texto,  fleqn las descentra
\marginparwidth0cm
\parindent0cm %Define la distancia de la primera linea de un parrafo a la margen
\parskip 0.6em %Espacio entre párrafos

%Para tablas,  redefine el backschlash en tablas donde se define la posici\'{o}n del texto en las
%casillas (con \centering \raggedright o \raggedleft)
\newcommand{\PreserveBackslash}[1]{\let\temp=\\#1\let\\=\temp}
\let\PBS=\PreserveBackslash
%Espacio entre lineas
\renewcommand{\baselinestretch}{1.1}
%Neuer Befehl f\"{u}r die Tabelle Eigenschaften der Aktivkohlen
\newcommand{\arr}[1]{\raisebox{1.5ex}[0cm][0cm]{#1}}
%Neue Kommandos
\usepackage{Befehle}

% Variable para controlar numeración de capítulos
\newif\ifnumberchapters
\numberchaptersfalse  % Por defecto, NO numerar capítulos

% Comando que se usa en cada archivo de capítulo
\newcommand{\mychapter}[1]{%
    \ifnumberchapters%
        \chapter{\centering #1}%
    \else%
        \chapter*{\centering #1}%
        \addcontentsline{toc}{chapter}{#1}%
        \stepcounter{chapter}%
        \phantomsection
    \fi%
}

%Inicio del documento. Tener en cuenta que hay archivos auxiliares
\begin{document}
\pagenumbering{roman}
%\newpage
%\setcounter{page}{1}

%----PORTADA----
\begin{center}

\thispagestyle{empty} \textbf{\Large
Propuesta del módulo de configuración del ``software contable
para uso educativo en el programa de contaduría pública de la
Universidad del Cauca''}\\[1.0cm]

%\vskip 1cm

\begin{figure}[H]
	\centering
	\includegraphics[scale = 0.3]{HojaTitulo/EscudoUnicauca.png}
\end{figure}

%\vskip 1cm

\Large{Monografía para optar al título de ingeniero de sistemas}\\[0.1cm]
\Large{Modalidad: Práctica Profesional}\\[0.1cm]
\Large\textbf{Santiago Nieto Guaca}\\[1.0cm]

\Large{Asesor de la empresa: Mauro Andrés Sánchez Muñoz}\\

\Large{Director: PhD. Wilson Libardo Pantoja Yepez}\\

\Large{Co-Director: PhD. Julio Ariel Hurtado Alegria}\\[1.0cm]

\textit{\huge\calligra{Universidad del Cauca}} \\
\textbf{Facultad de Ingeniería Electrónica y Telecomunicaciones \\
	Departamento de Sistemas \\
	Linea de investigación: Ingeniería de Software \\
Popayán, Diciembre de 2025}

\end{center}

% REDEFINICIONES DE NOMBRES (después de \begin{document})
\renewcommand{\tablename}{\textbf{Tabla}}
\renewcommand{\figurename}{\textbf{Figura}}
\renewcommand{\listtablename}{Lista de Tablas}
\renewcommand{\listfigurename}{Lista de Figuras}
\renewcommand{\contentsname}{Contenido}

% TABLA DE CONTENIDOS Y LISTAS
\tableofcontents
\clearpage

\phantomsection
\addcontentsline{toc}{chapter}{Lista de figuras} % para que aparezca en el indice de contenidos
\listoffigures % indice de figuras
\clearpage

\phantomsection
\addcontentsline{toc}{chapter}{Lista de tablas} % para que aparezca en el indice de contenidos
\listoftables % indice de tablas
\clearpage

%\include{Tab_Simbolos/TabSimbolosMSc}
%\include{Resumen}%\newcommand{\clearemptydoublepage}{\newpage{\pagestyle{empty}\cleardoublepage}}

\pagenumbering{arabic}

% Capítulos sin numeración (usa \mychapter{} en los archivos)
\mychapter{Capítulo 1. Introducción}

\section{Justificación y planteamiento del problema}

La Universidad del Cauca ofrece una variedad de programas académicos orientados a la formación de profesionales en diversas áreas de conocimiento. Entre ellos, el programa de Contaduría Pública que tiene como propósito formar profesionales con una sólida capacidad en el ámbito de los negocios, responsables de garantizar la transparencia y veracidad de la información económica y financiera de las organizaciones. De igual manera, se busca formar contadores con las competencias necesarias para gestionar y administrar los recursos de em-
presas y entidades gubernamentales \cite{ContaduriaPublica1Justi}.

Sin embargo, se han presentado desafíos tecnológicos que impactan en la calidad de la enseñanza y el aprendizaje. Ante la carencia de un software contable propio y con el objetivo de fortalecer los procesos educativos dentro del programa de Contaduría Pública, se ha recurrido a la compra de licencias del software ERP Millenium Enterprise. No obstante, este software presenta la limitante de ser una versión académica que no da acceso a todos los módulos disponibles. Además, al ser un software externo, toda la configuración de las bases de datos debe ser gestionada por el proveedor, lo que genera retrasos y limita la capacidad de los profesores para personalizar el sistema de acuerdo a las necesidades específicas del programa.

Este tipo de dificultades subraya la importancia de integrar herramientas más eficientes y adaptadas, tal como lo sugieren varios estudios, los cuales demuestran que el uso de software en el aula apoya de manera significativa el alcance de los resultados de aprendizaje \cite{Kanapathippillai2012}, al
tiempo que contribuye al desarrollo y mejora de habilidades en el manejo de las Tecnologías de la Información y la Comunicación (TIC) \cite{WanMohdNori2016}. Además, se ha evidenciado que la integración de software en los procesos de enseñanza de contabilidad ofrece beneficios tangibles, lo que permite a los estudiantes aplicar sus conocimientos en el entorno empresarial \cite{Boulianne2014}.

El uso del software contable en el proceso de enseñanza y aprendizaje de la contabilidad ha sido ampliamente propuesto por muchos académicos destacados  \cite{WanMohdNori2016}, al punto que hoy en día se ha convertido en una necesidad su incorporación en las unidades de desarrollo académico (UDA) que promuevan el estudio y práctica del ciclo contable. Se argumenta que con la utilización de un software contable en clase se logra representar y simular las prácticas que se llevarían en la mayoría de las organizaciones, al mismo tiempo que se ofrece un recurso pedagógico alternativo capaz de promover conocimientos relevantes y fundamentales de
la contabilidad, proporcionando una mejor preparación de los estudiantes para el cambiante mundo de los negocios y la profesión contable \cite{Boulianne2014}.

La educación contable debe tener en cuenta los requisitos del mercado relacionados con las habilidades y conocimientos necesarios para los profesionales de la contaduría.
Esto incluye ajustar el plan de estudios, métodos y herramientas utilizados en el proceso de enseñanza, incorporando el uso de tecnologías de la información (TI), lo cual fomenta la formación de profesionales con habilidades relevantes para el mercado laboral \cite{Novak2021}. La integración de las TI fortalece dos dimensiones, en primer lugar mejorar las habilidades tecnológicas de los estudiantes y en segundo lugar, utilizar estas tecnologías como herramientas estratégicas de enseñanza que mejoran las experiencias de aprendizaje \cite{Sangster1992}.

En este contexto, para que los distintos modulos del sistema contable funcionen adecudamente, es necesario configurar y gestionar parámetros como paso inicial a la contabilidad de una empresa.  Por lo tanto, se propone implementar la propuesta del módulo de configuración como parte del ”Sistema Contable para Uso Educativo en el Programa de Contaduría Pública”. Este módulo permitirá configurar y administrar los elementos esenciales del sistema, como cuentas contables, terceros, unidades de medida,  productos, categorías, tipos de documentos, números de documentos, resolución de facturación e impuestos, entre otros. Con esta herramienta, los estudiantes podrán interactuar con un entorno contable configurable permitiendo el desarrollo de habilidades prácticas en la administración de sistemas contables, simulando escenarios empresariales. 

Al integrar estas tecnologías, el programa de Contaduría Pública no solo mejorará las competencias tecnológicas de los estudiantes, sino que también reforzará su capacidad para enfrentar los desafíos del mercado laboral y aplicar sus conocimientos en contextos reales. El desarrollo del módulo de configuración implica la implementación tanto del frontend como del backend, permitiendo la integración de una interfaz intuitiva para los usuarios y una lógica de negocio robusta que permita el manejo eficiente de la información.


\section{Objetivos}

\subsection{Objetivo General}
Desarrollar la propuesta del módulo de configuración\footnote{El módulo de configuración incluye parámetros importantes del sistema que serán usados en otros módulos, como cuentas contables, terceros, inventario de productos, metodos de pago, centros de costo, bancos, tipos de documentos e impuestos.} del ``Software contable para uso educativo en el programa de Contaduría Pública de la Universidad del Cauca'', con el propósito de ofrecer un entorno para la personalización y administración del sistema para adaptarse a las necesidades académicas y prácticas de estudiantes y docentes.

\subsection{Objetivos Específicos}
\begin{enumerate}
    \item Definir los requisitos funcionales y no funcionales del módulo de configuración como componente del “Software contable para uso educativo en el programa de Contaduría Pública de la Universidad del Cauca”, conforme a las necesidades contables y académicas del programa.

    \item Diseñar la arquitectura del módulo de configuración utilizando el modelo C4, aplicando principios de diseño modular, escalable y eficiente que permitan soportar adecuadamente los requisitos funcionales y no funcionales.

    \item Implementar el módulo de configuración, abarcando tanto el front-end como el back-end, considerando los requisitos funcionales, no funcionales y la arquitectura definida.

    \item Evaluar las funcionalidades implementadas del módulo de configuración, mediante pruebas de integración y aceptación, verificando su correcto funcionamiento y adecuada integración con el resto del sistema.
    
\end{enumerate}


 \section{Aprendizaje Basado en Proyectos (PBL)}

El Aprendizaje Basado en Proyectos (PBL) es una estrategia pedagógica que sitúa a los participantes en el centro de un proceso de investigación y creación. Esta se distingue por el desarrollo de proyectos complejos y reales que exigen la aplicación de conocimientos y habilidades para resolver un problema o producir un resultado específico \cite{redemAprendizajeBasado}.

El desarrollo inicial del sistema contable se llevó a cabo mediante esta estrategia, lo que facilitó una comprensión más profunda del dominio del negocio y de los requisitos del sistema a través de la experiencia práctica. La implementación de esta estrategia se organizó en dos etapas bien diferenciadas, tal como se ilustra en la \autoref{fig:PBL}.

 
\begin{figure}[h!]    
    \centering%
    \includegraphics[width=1.0\textwidth, height=1.0\textheight, keepaspectratio]{Cap1/Figuras/PBL.png}    
    \caption{Etapas del enfoque de aprendizaje basado en proyectos.}
    \label{fig:PBL}
\end{figure}


 \begin{itemize}
\item \textbf{Etapa 1 (2024-1 a 2024-2):} En esta etapa inicial, se implementaron los módulos básicos de ContApp, incluyendo Terceros, Impuestos, Productos y Catálogo de cuentas, los cuales actualmente forman parte del módulo de Configuración. Los grupos conformados por estudiantes y profesores colaboraron en el desarrollo de las características fundamentales del sistema, lo que facilitó la obtención de conocimientos esenciales acerca de los procedimientos contables, la organización estructural y las demandas operativas de la empresa. Esta etapa resultó esencial para crear una base firme de entendimiento del área y para determinar los requerimientos preliminares del sistema.

\item \textbf{Etapa 2 (2025-1 a 2025-2):} En esta segunda etapa, correspondiente a la práctica profesional, se desarrolló la versión funcional de ContAppUC. El conocimiento y experiencia adquiridos en la primera fase facilitaron una elicitación de requisitos más precisa y completa. La comprensión profunda del dominio del negocio permitió identificar requisitos que no habían sido evidentes inicialmente, mejorar la comunicación con los stakeholders y proponer soluciones más alineadas con las necesidades de la organización.
\end{itemize}

\newpage
\section{Metodología de trabajo}

Para el desarrollo de la práctica profesional correspondiente a la etapa 2 del PBL, se realizaron entregas por fases con la modalidad de trabajo freelance. Esta modalidad se distingue por conferir al profesional la flexibilidad horaria y la capacidad de gestionar proyectos de manera independiente, permitiéndole ofrecer sus servicios y actuar conforme a los requerimientos específicos del cliente. \cite{mividalaboralFreelanceGua}. 

Sin embargo, dada la naturaleza modular del software contable, esta autonomía freelance se articuló con una comunicación constante con los otros desarrolladores responsables de los demás componentes del sistema. Esta interacción permite una mejor integración de los diferentes módulos, requiriendo una coordinación constante en aspectos como la definición de interfaces, estándares de desarrollo y protocolos de comunicación entre módulos. De esta forma, la modalidad freelance autónoma se ajusta así a un contexto colaborativo en el que, si bien cada desarrollador mantiene su independencia operativa, el marco de comunicación regular permite la coherencia técnica y funcional del producto final.

Para gestionar el flujo de trabajo híbrido se utilizó Kanban, una herramienta que permite organizar y optimizar servicios profesionales adoptando una visión holística del trabajo, enfocándose en la mejora continua desde la perspectiva del cliente. Este método permite visualizar el flujo de trabajo y la carga de tareas mediante tableros que representan las distintas etapas del proceso organizadas en columnas como “pendiente”, “en progreso”, “en prueba” y “finalizado”, lo que facilita el seguimiento del avance hasta su culminación \cite{guiaKanban}.


\subsection{Trabajo freelance}

El trabajo freelance representa una modalidad laboral donde el profesional ejerce su actividad de manera autónoma, estableciendo vínculos contractuales temporales con diversos clientes según las demandas específicas de cada proyecto, esta forma de trabajo se ha consolidado como una alternativa significativa al empleo tradicional en el panorama laboral contemporáneo, caracterizándose por su naturaleza flexible y su capacidad de adaptación a las dinámicas del mercado actual. La esencia del trabajo freelance radica en la libertad del profesional para estructurar su actividad laboral según sus propias necesidades y objetivos, sin estar sujeto a las restricciones organizacionales típicas del empleo convencional; el freelancer mantiene la autonomía para seleccionar sus proyectos, gestionar su tiempo de manera independiente, lo que le permite desarrollar una carrera profesional personalizada y diversificada. Esta modalidad laboral presenta múltiples dimensiones que la distinguen del empleo tradicional y que contribuyen a su creciente popularidad entre profesionales de diversas áreas \cite{mividalaboralFreelanceGua}:

\begin{itemize}
    \item \textbf{Flexibilidad temporal y geográfica:} Constituye uno de los aspectos más valorados de esta modalidad laboral, ya que permite al trabajador freelancer estructurar su jornada según sus preferencias personales y las exigencias específicas de cada proyecto, facilitando un equilibrio entre la vida profesional y personal. Esta flexibilidad temporal se complementa con la independencia geográfica, que elimina las limitaciones territoriales tradicionales del empleo convencional y posibilita el trabajo remoto desde cualquier ubicación.
    \item \textbf{Autonomía:} La autonomía se manifiesta en la capacidad del freelancer para tomar decisiones estratégicas sobre su negocio, desde la selección de clientes hasta la definición de metodologías de trabajo, lo que fomenta el desarrollo de habilidades empresariales y la construcción de una identidad profesional sólida.
    \item \textbf{Diversificación de experiencias:} La diversificación de proyectos representa otra dimensión significativa, ya que el freelancer puede involucrase en iniciativas de diferentes sectores y naturalezas, enriqueciendo su experiencia profesional y ampliando su red de contactos, esta variedad no solo contribuye al crecimiento profesional, sino que también reduce los riesgos asociados a la dependencia de un único empleador o sector económico.
\end{itemize}

\subsection{Kanban}

El método Kanban representa un enfoque integral de gestión del trabajo que se fundamenta en la visualización y optimización de los flujos de trabajo, particularmente efectivo para servicios profesionales y trabajo del conocimiento. Este método adopta una perspectiva holística que prioriza la mejora continua de los servicios desde la óptica del cliente, permitiendo a las organizaciones desarrollar una comprensión profunda de cómo se desplaza el trabajo a través de sus procesos. La implementación de Kanban facilita una gestión empresarial más eficiente al proporcionar visibilidad sobre los riesgos asociados a la entrega de servicios, mientras que simultáneamente desarrolla la capacidad adaptativa organizacional necesaria para responder de manera ágil a las cambiantes necesidades del mercado y las expectativas de los clientes. Aunque inicialmente reconocido por su aplicación en equipos para mitigar la sobrecarga laboral y recuperar el control operativo, el verdadero potencial de Kanban se manifiesta cuando se implementa a escala organizacional, abarcando múltiples equipos y departamentos. En este amplio contexto, Kanban trasciende su función como herramienta de gestión de tareas para convertirse en un potente instrumento de desarrollo organizacional orientado al servicio, generando oportunidades significativas de mejora estructural y operativa \cite{guiaKanban}.

Para materializar estos principios conceptuales, los tableros Kanban constituyen el mecanismo de visualización más ampliamente utilizado en la implementación de sistemas Kanban. Estos tableros operan bajo una lógica direccional uniforme que facilita el seguimiento del progreso del trabajo, los elementos ingresan por el lado izquierdo del tablero y avanzan hacia la derecha hasta completar su ciclo de entrega de valor al cliente. La arquitectura de un sistema Kanban establece puntos claramente definidos de compromiso y entrega. Los elementos de trabajo pueden variar considerablemente en naturaleza y magnitud, abarcando desde tareas individuales y requisitos específicos hasta proyectos completos y paquetes de productos en tableros de nivel estratégico, adaptándose así a diversos contextos organizacionales como historias de usuario en desarrollo de software, procesos de recursos humanos o desarrollo de productos. Fundamentado en el principio de ``comenzar con lo que se hace actualmente'', el tablero Kanban refleja fielmente el flujo operativo real en lugar de una configuración idealizada, utilizando columnas para representar las etapas del proceso y carriles para distribuir la capacidad según diferentes tipos de trabajo o proyectos \cite{guiaKanban}.

\subsection{Fases de desarrollo}

A continuación, se presentan las diferentes fases, junto con las actividades específicas desarrolladas en cada una de ellas.

\begin{table}[h]
\centering
\caption{Descripción de la Fase 1}
\label{tab:sprint1}
\renewcommand{\arraystretch}{1.3} % mejora separación general
\begin{tabular}{|p{3cm}|p{10cm}|}
\hline
\multicolumn{2}{|c|}{\textbf{Fase 1: Análisis de requisitos}} \\
\hline
\textbf{Descripción} & Identificación y documentación de los requisitos funcionales y no funcionales del módulo de configuración. Priorización de funcionalidades según su valor para los interesados.\\
\hline
\textbf{Roles involucrados} & 
\begin{minipage}[t]{\linewidth}
\vspace{3pt} % espacio arriba
\begin{itemize}
\item Analista
\item Diseñador
\item Cliente
\end{itemize}
\vspace{4pt} % espacio abajo
\end{minipage} \\
\hline
\textbf{Actividades} & 
\begin{minipage}[t]{\linewidth}
\vspace{3pt}
\begin{itemize}
\item Reuniones presenciales y virtuales
\item Contrucción de historias de usuario
\item Construcción de prototipos
\end{itemize}
\vspace{3pt}
\end{minipage} \\
\hline
\textbf{Rol del practicante} & 
\begin{minipage}[t]{\linewidth}
\vspace{3pt}
\begin{itemize}
\item Analista
\item Diseñador
\end{itemize}
\vspace{3pt}
\end{minipage} \\
\hline
\end{tabular}
\end{table}

\begin{table}[h]
\centering
\caption{Descripción Iteración 2}
\label{tab:sprint2}
\begin{tabular}{|p{3cm}|p{10cm}|}
\hline
\multicolumn{2}{|c|}{\textbf{Iteración 2}} \\
\hline
\textbf{Fase} & Diseño de arquitectura \\
\hline
\textbf{Descripción} & Concepción de la arquitectura del sistema y lineamientos de desarrollo. \\
\hline
\textbf{Roles involucrados} & 
\begin{minipage}[t]{\linewidth}
\vspace{3pt}
\begin{itemize}
\item Arquitecto.
\item Desarrollador.
\item Director de grado.
\end{itemize}
\vspace{3pt}
\end{minipage} \\
\hline
\textbf{Actividades} & 
\begin{minipage}[t]{\linewidth}
\vspace{3pt}
\begin{itemize}
\item Revisión de la documentación.
\item Construcción de diagramas del sistema haciendo uso del modelo C4.
\end{itemize}
\vspace{3pt}
\end{minipage} \\
\hline
\textbf{Rol del practicante} & 
\begin{minipage}[t]{\linewidth}
\vspace{3pt}
\begin{itemize}
\item Arquitecto.
\item Desarrollador.
\end{itemize}
\vspace{3pt}
\end{minipage} \\
\hline
\end{tabular}
\end{table}
\begin{table}[h]
\centering
\caption{Descripción de la Fase 3}
\label{tab:sprint3}
\begin{tabular}{|p{3cm}|p{10cm}|}
\hline
\multicolumn{2}{|c|}{\textbf{Fase 3: Desarrollo}} \\
\hline
\textbf{Descripción} & Desarrollo de funcionalidades priorizadas. \\
\hline
\textbf{Roles involucrados} & 
\begin{minipage}[t]{\linewidth}
\vspace{3pt}
\begin{itemize}
\item Desarrollador.
\item Cliente.
\end{itemize}
\vspace{3pt}
\end{minipage} \\
\hline
\textbf{Actividades} & 
\begin{minipage}[t]{\linewidth}
\vspace{3pt}
\begin{itemize}
\item Implementación de funcionalidades.
\end{itemize}
\vspace{3pt}
\end{minipage} \\
\hline
\textbf{Rol del practicante} & 
\begin{minipage}[t]{\linewidth}
\vspace{3pt}
\begin{itemize}
\item Desarrollador
\end{itemize}
\vspace{3pt}
\end{minipage} \\
\hline
\end{tabular}
\end{table}
\begin{table}[h]
\centering
\caption{Descripción de la Fase 4}
\label{tab:sprint4}
\begin{tabular}{|p{3cm}|p{10cm}|}
\hline
\multicolumn{2}{|c|}{\textbf{Fase 4: Pruebas unitarias y de integración}} \\
\hline
\textbf{Descripción} & Validación del correcto funcionamiento de las funcionalidades implementadas mediante pruebas unitarias individuales y pruebas de integración. \\
\hline
\textbf{Roles involucrados} & 
\begin{minipage}[t]{\linewidth}
\vspace{3pt}
\begin{itemize}
\item Desarrollador
\item QA
\end{itemize}
\vspace{3pt}
\end{minipage} \\
\hline
\textbf{Actividades} & 
\begin{minipage}[t]{\linewidth}
\vspace{3pt}
\begin{itemize}
\item Pruebas unitarias.
\item Pruebas de integración.
\end{itemize}
\vspace{3pt}
\end{minipage} \\
\hline
\textbf{Rol del practicante} & 
\begin{minipage}[t]{\linewidth}
\vspace{3pt}
\begin{itemize}
\item Desarrollador
\item Tester
\end{itemize}
\vspace{3pt}
\end{minipage} \\
\hline
\end{tabular}
\end{table}
\begin{table}[h]
\centering
\caption{Descripción Iteración 5}
\label{tab:sprint5}
\begin{tabular}{|p{3cm}|p{10cm}|}
\hline
\multicolumn{2}{|c|}{\textbf{Iteración 5}} \\
\hline
\textbf{Fase} & Pruebas de aceptación con usuarios clave \\
\hline
\textbf{Descripción} & Verificación de la usabilidad, funcionalidad y adecuación del sistema para el entorno educativo contable. \\
\hline
\textbf{Roles involucrados} & 
\begin{minipage}[t]{\linewidth}
\vspace{3pt}
\begin{itemize}
\item Desarrollador
\item QA
\item Cliente
\end{itemize}
\vspace{3pt}
\end{minipage} \\
\hline
\textbf{Actividades} & 
\begin{minipage}[t]{\linewidth}
\vspace{3pt}
\begin{itemize}
\item Pruebas funcionales.
\item Pruebas de aceptación.
\end{itemize}
\vspace{3pt}
\end{minipage} \\
\hline
\textbf{Rol del practicante} & 
\begin{minipage}[t]{\linewidth}
\vspace{3pt}
\begin{itemize}
\item Desarrollador
\item Tester
\end{itemize}
\vspace{3pt}
\end{minipage} \\
\hline
\end{tabular}
\end{table}
\begin{table}[h]
\centering
\caption{Descripción de la Fase 6}
\label{tab:sprint6}
\begin{tabular}{|p{3cm}|p{10cm}|}
\hline
\multicolumn{2}{|c|}{\textbf{Fase 6: Revisión de entregas}} \\
\hline
\textbf{Descripción} & Evaluación integral del módulo de configuración desarrollado, analizando el cumplimiento de objetivos y requisitos establecidos. Se identifican oportunidades de mejora basadas en el feedback de usuarios y resultados de pruebas. \\
\hline
\textbf{Roles involucrados} & 
\begin{minipage}[t]{\linewidth}
\vspace{3pt}
\begin{itemize}
\item Desarrollador
\item Tester
\item Stakeholders
\item Usuario final
\end{itemize}
\vspace{3pt}
\end{minipage} \\
\hline
\textbf{Actividades} & 
\begin{minipage}[t]{\linewidth}
\vspace{3pt}
\begin{itemize}
\item Revisión de entregables y documentación
\item Análisis de feedback de usuarios
\item Identificación de mejoras potenciales
\end{itemize}
\vspace{3pt}
\end{minipage} \\
\hline
\textbf{Rol del practicante} & 
\begin{minipage}[t]{\linewidth}
\vspace{3pt}
\begin{itemize}
\item Analista
\item Desarrollador
\end{itemize}
\vspace{3pt}
\end{minipage} \\
\hline
\end{tabular}
\end{table}




\mychapter{Capítulo 2. Marco Teórico}

La enseñanza de contabilidad en el siglo XXI requiere de profesionales con conocimientos sólidos y habilidades que permitan motivar a los estudiantes. Gracias a la tecnología, se ha facilitado la creación de herramientas para el aprendizaje individualizado \cite{Humphrey2014}, impulsando
modelos como Computer Assisted Learning (CAL) y otros basados en Tecnologías de la Infor-
mación y la Comunicación (TIC) \cite{Wali2021}.

El modelo CAL es el que se ha venido implementando en la educación contable al integrar
el software como parte esencial en el proceso de enseñanza y aprendizaje \cite{Huczynski2005}. Este modelo
permite aplicar diversas situaciones que ayudan al estudiante a comprender conceptos y gene-
rar experiencias prácticas \cite{Kanapathippillai2012}, ofreciendo un estilo de aprendizaje enriquecedor que aumenta
la motivación y facilita el aprendizaje activo \cite{Huczynski2005}. Sin embargo, es fundamental que no solo se
aprenda el uso del software, sino que este sea empleado como herramienta pedagógica \cite{Blount2016}.
Además, CAL beneficia a las universidades al optimizar costos y mejorar la calidad educativa
[9], reforzando su imagen como instituciones modernas \cite{Kanapathippillai2012}.

Debido a lo anterior, el dominio de las TIC se ha identificado como un factor diferenciador
en contabilidad y finanzas \cite{Osmani2020}. La International Federation of Accountants (IFAC), organismo
internacional que regula la profesión contable y promueve estándares de formación, enfatiza
en el desarrollo de habilidades técnicas, organizacionales y sociales en los contadores \cite{Wadi2021}.
Asímismo, organizaciones como CPA Australia, la American Accounting Association (AAA) y
el American Institute of Certified Public Accountants (AICPA) promueven la incorporación de
tecnologías emergentes en la educación contable \cite{Willis2016}. La adopción de estas tecnologías debe
formar parte de una estrategia educativa integral \cite{Sloan1995}.

El concepto de contador híbrido surge como respuesta a las nuevas exigencias del mercado laboral, donde se requiere que los profesionales combinen conocimientos contables con habilidades en tecnologías de la información \cite{Osmani2020}. Esto ha impulsado a las universidades a
mejorar la formación académica para responder a la demanda empresarial \cite{Sloan1995}. En este contexto, la IFAC, a través del International Accounting Education Standards Board (IAESB), ha desarrollado normativas internacionales como la International Education Standard 2 (IES 2),
la cual establece competencias esenciales en TIC para contadores \cite{Daff2021}. Estas competencias incluyen el análisis de datos, la gestión de riesgos y la optimización de sistemas organizacio nales, elementos clave para la formación y desempeño profesional en contabilidad \cite{Daff2021}.

Finalmente, el rápido avance tecnológico en el ámbito contable ha llevado a una evolución
en las herramientas utilizadas por los profesionales del sector. Tecnologías como la digitaliza-
ción de datos, la computación en la nube, la inteligencia artificial (IA), las tecnologías finan-
cieras (FinTech), el blockchain y el análisis de datos masivos (Big Data) han transformado los
procesos contables y financieros \cite{Daff2021}. La adopción y dominio de estas herramientas no solo
mejora la eficiencia de las organizaciones, sino que también fortalece la competitividad de los
contadores en el mercado laboral global \cite{Gould2017}. En este sentido, la formación académica debe
evolucionar constantemente para preparar a los futuros contadores ante los desafíos de la era
digital \cite{WanMohdNori2016}.

\section{Conceptos fundamentales}
A continuación se definen los conceptos y tecnologías empleadas durante el desarrollo de la práctica profesional, con el propósito de establecer un marco de referencia común para la comprensión del presente documento.

\subsection{Configuración}

En contabilidad, la configuración de un software contable se refiere a la implementación y adaptación de los ajustes, procesos, datos y reglas de negocio dentro del sistema para que se alineen precisamente con las necesidades operativas y fiscales de una empresa \cite{mygestion}. Esto implica tomar una herramienta estándar y personalizar sus módulos y funcionalidades, como la gestión de facturación, la contabilidad, el manejo de almacén y la configuración de datos fiscales o bancarios, para reflejar los flujos de trabajo y los requerimientos específicos del negocio \cite{mygestion}.

Una configuración adecuada permite gestionar las finanzas y operaciones de manera eficiente, ofreciendo simplicidad y control, similar a cómo un motor bien ajustado impulsa un automóvil \cite{cloudgestion}. Sin una estrategia clara y una configuración correcta, las empresas corren el riesgo de sufrir interrupciones costosas, ineficiencias operativas y no alcanzar los objetivos de la implementación del sistema \cite{aptyImplementationSteps}.

Además, permite centralizar la información y obtener una visión global en tiempo real, automatizar procesos, mejorar la toma de decisiones, facilitar el cumplimiento normativo y definir roles de usuario y entornos de trabajo \cite{diariosigloxxiCEESASA}. Es vital para gestionar procesos, procedimientos y proteger la información, asegurando que las transacciones se registren, administren y reporten de manera precisa dentro del contexto específico de la empresa \cite{managerCmoConfigurar}.

\subsection{Framework}
Un framework es una colección integrada de herramientas, aplicaciones y librerías que implementa estándares y patrones de diseño establecidos. Su objetivo principal es proporcionar una estructura cohesiva que facilite el desarrollo de software para fines específicos, permitiendo que los desarrolladores se enfoquen en la lógica de negocio fundamental sin preocuparse por tareas repetitivas o por la implementación de aspectos técnicos complejos y especializados \cite{FrameworkWikipedia}.

\subsection{Modelo C4}
El Modelo C4 está estructurado mediante niveles de abstracción organizados jerárquicamente, abarcando desde sistemas de software hasta contenedores, componentes y código fuente. Esta arquitectura se visualiza a través de diagramas dispuestos en diferentes niveles que muestran el contexto del sistema, la organización de contenedores, la estructura de componentes y los detalles del código. Una ventaja significativa de este modelo es que no depende de ninguna notación particular ni requiere herramientas específicas para su aplicación práctica \cite{c4modelHome}.

\subsection{Pruebas de software}
Las pruebas de software constituyen un proceso sistemático destinado a evaluar y confirmar que una aplicación o sistema funciona correctamente según las especificaciones establecidas. Sus principales ventajas incluyen la detección temprana de fallos y el incremento del rendimiento general del sistema.
Las pruebas resultan más eficientes cuando se ejecutan de manera continua, integrándose desde las fases iniciales de diseño y manteniéndose activas durante todo el ciclo de desarrollo hasta llegar al entorno de producción. Este enfoque continuo elimina la necesidad de esperar a que el producto esté completamente terminado para iniciar las validaciones.
Existen diferentes tipos pruebas de software, algunas de ellas son \cite{ibmPruebas}:

\begin{itemize}
    \item \textbf{Pruebas de aceptación:} Verifican si todo el sistema funciona según lo previsto.
    \item \textbf{Pruebas de integración:} Aseguran que los diferentes elementos o funcionalidades del software operan de manera coordinada e integrada.
    \item \textbf{Pruebas unitarias:} Validan que cada módulo individual del software opera conforme a lo previsto. Un módulo representa la parte más pequeña y verificable de una aplicación.
    \item \textbf{Pruebas funcionales:} Verifican las funcionalidades del sistema simulando situaciones empresariales reales basadas en los requerimientos funcionales establecidos. Las pruebas de caja negra constituyen un método común para validar estas funcionalidades.
    \item \textbf{Pruebas de rendimiento:} Evalúan el comportamiento del software bajo diversas condiciones de carga operativa. Las pruebas de carga, por ejemplo, se emplean para medir el desempeño del sistema en escenarios de uso real.
    \item \textbf{Pruebas de usabilidad:} Validan la facilidad con que un usuario puede interactuar con un sistema o aplicación web para llevar a cabo una tarea específica.
\end{itemize}

\subsection{Patrones arquitectónicos}
\subsubsection{Arquitectura hexagonal}
La arquitectura hexagonal, también conocida como arquitectura de puertos y adaptadores, es un patrón diseñado para crear componentes de aplicación débilmente acoplados, facilitando la conexión con el entorno externo. Su objetivo principal es aislar la lógica de negocio del núcleo de la aplicación de factores externos como bases de datos, interfaces de usuario o sistemas de mensajería, lo que mejora la modularidad, mantenibilidad y testabilidad. El sistema se divide en componentes intercambiables conectados a través de puertos (interfaces que representan puntos de entrada y salida) e implementados por adaptadores, que traducen las interacciones entre el mundo exterior y el núcleo \cite{HexagonalArchitecture}.

Como se ilustra en la Figura \ref{fig:hexagonal}, la arquitectura se organiza en capas concéntricas: el núcleo del dominio contiene entidades y lógica de negocio; la capa de aplicación incluye servicios que orquestan casos de uso; y la capa de infraestructura alberga adaptadores. Estos se clasifican en primarios (como controladores REST o interfaces gráficas, que inician la comunicación) y secundarios (como repositorios de bases de datos o servicios de mensajería, que responden a la aplicación). Los adaptadores primarios incluyen la API REST y la interfaz de línea de comandos, mientras que los secundarios comprenden PostgreSQL para persistencia y RabbitMQ para comunicación asíncrona entre microservicios.


\begin{figure}[h!]    
    \centering%
    \includegraphics[width=0.9\textwidth, height=0.8\textheight, keepaspectratio]{Cap2/Figuras/hexagonal.png}    
    \caption{Arquitectura con patrón hexagonal.}
    \label{fig:hexagonal}
\end{figure}

 Esta separación garantiza que la lógica de negocio permanezca independiente de los detalles de implementación tecnológicos, permitiendo reemplazar cualquier adaptador sin afectar el núcleo del dominio. El principio de inversión de control se aplica a nivel arquitectónico, donde las dependencias se dirigen hacia el centro del sistema, haciendo que la lógica de negocio solo dependa de los puertos (interfaces) diseñados para satisfacer sus necesidades y no de herramientas o adaptadores específicos.
 
\subsubsection{Arquitectura en capas}

La arquitectura por capas es una de las más empleadas, debido a su sencillez y al hecho de que se adopta automáticamente cuando no se tiene certeza sobre qué patrón arquitectónico elegir para el desarrollo de la aplicación. Este enfoque implica segmentar la aplicación en diferentes niveles, con el propósito de asignar a cada uno una función específica, tales como una capa de interfaz de usuario (UI), una capa de lógica de negocio (servicios) y una capa de acceso a datos (DAO). No obstante, este patrón no establece un número fijo de capas para la aplicación, sino que enfatiza la división de la aplicación en niveles (aplicando el principio de Separación de Responsabilidades (SoC)) \cite{ArquitecturaCapas}.

En la implementación práctica, generalmente se utiliza un esquema de 4 capas: presentación, negocio, persistencia y base de datos. Sin embargo, es común observar que las capas de negocio y persistencia se fusionan en una sola, especialmente cuando la lógica de almacenamiento se integra directamente en la capa de negocio \cite{ArquitecturaCapas}. 

En la \autoref{fig:capas} se observa la arquitectura organizada en cuatro capas horizontales que representan la separación de responsabilidades.

\begin{figure}[h!]    
    \centering%
    \includegraphics[width=1.0\textwidth, height=1.0\textheight, keepaspectratio]{Cap2/Figuras/capas.png}    
    \caption{Arquitectura con patrón en capas.}
    \label{fig:capas}
\end{figure}

\subsection{Patrones de diseño}

Los patrones de diseño representan estrategias recurrentes para abordar problemas comunes en el desarrollo de software, que funcionan como guías conceptuales que pueden adaptarse y personalizarse según las necesidades específicas de cada proyecto. A diferencia de una función o biblioteca lista para usar, un patrón de diseño no es un fragmento de código concreto, sino una solución abstracta que orienta la estructura y organización del programa. Su implementación requiere interpretar el concepto y ajustarlo a las particularidades del sistema en desarrollo \cite{patterndesign}.

\subsubsection{Patrón strategy}

Es una solución de diseño orientada al comportamiento que facilita la definición de múltiples algoritmos relacionados, encapsulando cada uno en una clase distinta y permitiendo que sus instancias se puedan intercambiar dinámicamente según las necesidades del sistema. El patrón Strategy se compone de tres elementos fundamentales: el contexto, que mantiene una referencia a una estrategia y delega en ella la ejecución de la tarea, la interfaz Strategy, que establece un contrato común para todas las estrategias concretas y las estrategias concretas, que implementan las distintas variantes del algoritmo. Esta organización permite separar la lógica de negocio de los detalles específicos de cada algoritmo, lo que facilita la realización de pruebas y la evolución del sistema \cite{refactoringStrategy}.

\subsubsection{Patrón publicador-suscriptor}

Es un patrón de diseño de comportamiento en el que un objeto, llamado sujeto, mantiene una lista de sus dependientes, llamados observadores, y les notifica automáticamente cualquier cambio de estado. Este patrón es útil cuando los cambios en el estado de un objeto pueden requerir cambios en otros objetos, y el grupo de objetos puede ser desconocido de antemano o cambiar dinámicamente. Además, el patrón publicador-suscriptor ayuda a desacoplar los componentes del sistema, lo que facilita su mantenimiento y escalabilidad \cite{Observer}. 

\section{Tecnologías para el desarrollo}

Para el desarrollo del módulo es necesario utilizar diversas tecnologías que permitan su correcto funcionamiento, permitiendo tanto la interacción con los usuarios como el procesamiento de la información. Estas herramientas han sido seleccionadas con el objetivo de mejorar la experiencia del usuario, la eficiencia en el manejo de datos y la estabilidad del sistema. A continuación, se presentan las principales tecnologías que serán empleadas, abarcando desde el diseño de la interfaz hasta la gestión de la información en el servidor y la base de datos. 

\subsection{Tecnologías frontend}

\textbf{Angular y Tailwind:} En el desarrollo frontend, la combinación de herramientas que atienden
tanto la lógica funcional como el diseño visual de las aplicaciones resulta fundamental para construir soluciones eficientes, escalables y mantenibles. En este contexto, Angular y Tailwind CSS se presentan como tecnologías que, aunque abordan áreas distintas, pueden integrarse de forma complementaria para optimizar el proceso de desarrollo.

Angular es un framework que proporciona una arquitectura robusta basada en componentes, permitiendo estructurar las aplicaciones en vistas jerárquicas que interactúan entre sí.
Estas vistas se comunican mediante la inyección de dependencias con servicios que encapsulan la lógica de negocio o funcionalidades reutilizables, sin necesidad de estar directamente ligados a la interfaz gráfica. Además de su enfoque estructural, Angular incorpora herramientas integradas para la realización de pruebas unitarias y de integración, utilizando frameworks como Jasmine y Karma, lo cual favorece el desarrollo confiable y orientado a pruebas desde las primeras fases del proyecto \cite{Angular1MT}.

Por su parte, Tailwind CSS es un framework de estilos que permite aplicar directamente clases predefinidas en el HTML, eliminando la necesidad de escribir archivos CSS personalizados. Este enfoque promueve la rapidez en la maquetación de interfaces, permitiendo ajustes visuales inmediatos sin salir del flujo de trabajo del código. Tailwind favorece un diseño consistente, reutilizable y altamente personalizable, facilitando la implementación de interfaces modernas y adaptables a distintas resoluciones y dispositivos \cite{Tailwind1MT}.

\subsection{Tecnologías backend}

El desarrollo backend moderno se fundamenta en la integración de diversas tecnologías que, en conjunto, permiten estructurar aplicaciones web eficientes, escalables y orientadas a servicios. Tecnologías como JSON, REST, Spring Framework, Java y herramientas de registro de servicios como Eureka no solo cumplen funciones individuales dentro del ecosistema de desarrollo, sino que trabajan de forma articulada para facilitar el procesamiento, la entrega y la organización de los datos en una aplicación.

En este flujo de trabajo, JSON actúa como el formato de intercambio de datos por excelencia, gracias a su ligereza, legibilidad y compatibilidad con múltiples lenguajes. Su estructura basada en pares clave-valor lo hace ideal para representar objetos y transmitir información
entre cliente y servidor de forma sencilla y estructurada \cite{Json1MT}. Este formato cobra especial relevancia dentro de arquitecturas REST, donde los datos se transportan principalmente en formato JSON. REST define un estilo de arquitectura que permite que los distintos sistemas se
comuniquen a través de HTTP mediante operaciones estándar (GET, POST, PUT, DELETE), y que los recursos se identifiquen a través de URLs claras. Al combinar REST con JSON, se obtiene una interfaz de comunicación eficiente y estandarizada para consumir y exponer servicios web \cite{Rest1MT}.

Para implementar esta lógica de negocio entra en juego Spring Framework, una herramienta robusta que proporciona todos los componentes necesarios para desarrollar APIs de manera modular y mantenible. Spring no solo ofrece soporte para la construcción de controladores que gestionen solicitudes HTTP y devuelvan respuestas en formato JSON, sino que también simplifica aspectos internos como la inyección de dependencias, la gestión de datos o la configuración de pruebas. Su compatibilidad nativa con REST y su estructura basada en componentes lo convierten en un puente ideal entre los principios arquitectónicos REST y el desarrollo práctico de servicios web \cite{Spring1MT}.

En arquitecturas distribuidas de microservicios, se hace necesario incorporar herramientas especializadas para el registro y descubrimiento de servicios. Eureka Server, desarrollado inicialmente por Netflix, cumple esta función al actuar como un servidor centralizado donde los microservicios pueden registrarse automáticamente para ser localizados por otros servicios. Esta herramienta elimina la necesidad de definir manualmente las direcciones de cada servicio, gestionando dinámicamente el proceso de descubrimiento y comunicación entre componentes distribuidos \cite{EurekaServer}.

Todo este ecosistema se construye sobre Java, el lenguaje base que aporta características
como portabilidad, orientación a objetos, robustez y seguridad. Java no solo permite imple-
mentar la lógica del negocio y manipular estructuras complejas de datos, sino que también se
adapta perfectamente al uso de frameworks como Spring, asegurando rendimiento y fiabilidad
en aplicaciones backend de cualquier escala \cite{JAVA1MT}.


\subsection{Tecnologías para base de datos}

La gestión eficiente de la persistencia de datos en una aplicación requiere herramientas
que reduzcan la complejidad del acceso a bases de datos relacionales y que, al mismo tiempo,
se integren de manera fluida con el lenguaje de programación utilizado. En este sentido, Java
Persistence API (JPA) e Hibernate forman una combinación poderosa que facilita el trabajo
con bases de datos desde entornos orientados a objetos como Java.

JPA proporciona una especificación estándar para mapear clases Java a tablas de una base de datos, permitiendo manipular datos como objetos sin tener que interactuar directamente con el lenguaje SQL. Esto se traduce en una capa de abstracción que permite al desarrollador realizar operaciones como insertar, actualizar, eliminar o consultar registros, utilizando una sintaxis más cercana al paradigma orientado a objetos \cite{JPA1MT}.

Por su parte, Hibernate es una implementación concreta y extendida de esa especificación.No solo cumple con los lineamientos definidos por JPA, sino que añade funcionalidades avanzadas que automatizan procesos comunes como la sincronización entre objetos y registros, el control de transacciones o el uso de caché para mejorar el rendimiento. Hibernate actúa como el motor que lleva a cabo las operaciones indicadas por JPA, reduciendo la necesidad de código repetitivo y facilitando el mantenimiento del sistema \cite{Hibernate1MT}.

El sistema relacional que sirve de soporte a esta arquitectura de persistencia es PostgreSQL. Este sistema de gestión de bases de datos objeto-relacional (SGBDOR) de código abierto es reconocido por su robustez, su fuerte cumplimiento del estándar SQL y sus avanzadas capacidades transaccionales, incluyendo el control de concurrencia multi-versión (MVCC), lo que lo hace ideal para aplicaciones empresariales que requieren alta integridad y consistencia de datos \cite{googleQuPostgreSQL}.

\section{Lineamientos división TIC }

\subsection{Lineamientos de diseño}

El documento ``Lineamientos de diseño UX-UI para proyectos de desarrollo de software en la División TIC de la Universidad del Cauca'', desarrollado por el área de soporte y desarrollo en su primera versión, define directrices para el diseño en proyectos de desarrollo de software. Este incluye consideraciones sobre tipografía, esquemas cromáticos y enfoques de diseño, ya que los ``proyectos deben mantener la identidad visual del Alma Mater, con la finalidad de reforzar la institucionalidad y el posicionamiento de la marca en diversos entornos digitales'' \cite{TICS}. Entre estas directrices se encuentran especificaciones tipográficas que determinan el uso de las fuentes Titillium Web y Open Sans, así como definiciones cromáticas que establecen diferentes gamas de colores y su aplicación. En las figuras \ref{fig:fig1}, \ref{fig:fig2} y \ref{fig:fig3} se presentan algunos de los colores fundamentales.

\begin{figure}[h!]
    \centering%
    \includegraphics[width=0.6\textwidth, height=0.3\textheight, keepaspectratio]{Cap2/Figuras/Figura1.png}    
    \caption{Color institucional primario}
    \label{fig:fig1}
\end{figure}

\begin{figure}[h!]    
    \centering%
    \includegraphics[width=0.6\textwidth, height=0.3\textheight, keepaspectratio]{Cap2/Figuras/Figura2.png}    
    \caption{Color institucional secundario}
    \label{fig:fig2}
\end{figure}

\begin{figure}[h!]
    \centering%
    \includegraphics[width=0.6\textwidth, height=0.3\textheight, keepaspectratio]{Cap2/Figuras/Figura3.png}    
    \caption{Otros colores destacados}
    \label{fig:fig3}
\end{figure}

\newpage
\subsection{Lineamientos de desarrollo}
La Universidad del Cauca, a través de su división TIC, establece directrices de desarrollo que determinan, entre otros aspectos, las tecnologías a utilizar para el desarrollo de aplicaciones institucionales, las cuales incluyen las siguientes \cite{TICS2}:

\begin{table}[h]
    \centering
    \caption{Tecnologías para proyectos software de la división TIC}
    \label{tab:tecnologias}
    \begin{tabular}{|p{4cm}|p{8cm}|}
        \hline
        \textbf{Tipo} & \textbf{Tecnología} \\
        \hline
        Bases de datos SQL & 
        \begin{itemize}
            \item Oracle.
            \item MySQL.
            \item PostgreSQL.
        \end{itemize} \\
        \hline
        Bases de datos no SQL & 
        \begin{itemize}
            \item MongoDB.
            \item Cassandra.
        \end{itemize} \\
        \hline
        Bases de datos Real Time & 
        \begin{itemize}
            \item Firebase
        \end{itemize} \\
        \hline
        Back-end & 
        \begin{itemize}
            \item Spring Boot
            \item Java
        \end{itemize} \\
        \hline
        Front-end & 
        \begin{itemize}
            \item JavaScript
            \item Vue
            \item React con Next
            \item Angular con TypeScript
        \end{itemize} \\
        \hline
    \end{tabular}
\end{table}

\subsection{Lineamientos de usabilidad}
En el documento ``Lineamientos normativos y pautas de usabilidad para el diseño de interfaces en los proyectos de desarrollo software en la División TIC de la Universidad del Cauca'' se definen criterios de usabilidad, entre los cuales se encuentran \cite{TICS3}:

\begin{itemize}
    \item \textbf{Tipografía legible:} Uso de fuentes con dimensiones y contraste apropiados que garanticen una lectura cómoda para los usuarios.
    \item \textbf{Colores accesibles:} Implementación de esquemas de color que mantengan la identidad visual institucional, asegurando un contraste óptimo entre el texto y el fondo para promover la legibilidad y reducir el cansancio visual.
    \item \textbf{Navegación intuitiva:} Desarrollo de una arquitectura gráfica transparente y de fácil comprensión mediante el uso de encabezados y divisiones apropiadas que orienten al usuario en la búsqueda de información.
    \item \textbf{Contenido accesible:} Presentación de contenido claro y directo, empleando un lenguaje accesible y evitando terminología técnica innecesaria.
\end{itemize}

\mychapter{Capítulo 3. Descripción de los módulos a implementar}

Este capítulo describe el módulo de configuración mediante la descomposición en módulos funcionales, presentados en un lenguaje de alto nivel para facilitar la comprensión de cualquier lector, los cuales permiten la operatividad, personalización y adaptabilidad del sistema a contextos académicos. Se detallan las responsabilidades específicas de cada submódulo y cómo estos influyen en los demás módulos del sistema contable educativo, proporcionando un concepto general de cómo el sistema gira en torno a los parámetros establecidos en la configuración. Estos submódulos cubren gran parte de las dimensiones necesarias para simular entornos empresariales, desde la gestión de cuentas contables e impuestos hasta el control de inventarios, métodos de pago, centros de costo y otros parámetros operativos que se detallarán a continuación.


\begin{figure}[h!]    
    \centering%
    \includegraphics[width=1.0\textwidth, height=0.8\textheight, keepaspectratio]{Cap3/Figuras/Diagrama General-Diagrama Principal.drawio.png}    
    \caption{Submódulos que componen el módulo de configuración}
    \label{fig:DiagramaPrincipal}
\end{figure}

 El módulo de configuración es fundamental para el funcionamiento del
 sistema contable, ya que permite a los administradores establecer y mantener los datos maestros y parámetros que rigen las operaciones de todos los demás módulos. A continuación, se detallan los submódulos identificados.

\section{Gestión de catálogo de cuentas} Tiene como objetivo mantener una estructura contable actualizada y adaptada a las necesidades del sistema, proporcionando funcionalidades esenciales para la creación, modificación, eliminación, importación desde archivos planos, exportación a archivos planos y consulta de cuentas contables, lo que permite una administración integral del catálogo de cuentas.

Este submódulo constituye el núcleo fundamental del sistema contable, al establecer relaciones críticas con los módulos de Inventario (Promedio Ponderado y PEPS) para la generación automática de asientos contables relacionados con el manejo de inventarios; conectarse con el Módulo Comercial para el registro de transacciones de compras y ventas; interactuar con los módulos de Tesorería y Cartera para el registro de movimientos financieros; servir como base para los módulos Contable-Comercial y Contable Cartera, que dependen directamente del catálogo para generar asientos contables; proporcionar la información necesaria para los reportes financieros, tanto de Libros Auxiliares como de Estados Financieros y alimentar el Módulo Educativo Contable, que utiliza esta información para ejercicios de simulación. De esta manera, se consolida como el elemento vertebral que garantiza la coherencia y funcionalidad de todo el ecosistema contable. A continuación se muestran de manera gráfica las relaciones correspondientes del módulo de Gestión de Cuentas Contables.

\begin{figure}[h!]    
    \centering%
    \includegraphics[width=0.9\textwidth, height=0.8\textheight, keepaspectratio]{Cap3/Figuras/Diagrama General-Cuentas contables.png}    
    \caption{Relación del submódulo gestión de cuentas contables.}
    \label{fig:Cuentas contables}
\end{figure}

\newpage

\section{Gestión de tarifas de impuestos} Tiene como objetivo establecer los valores de las tarifas de impuestos asociadas a las transacciones del sistema, ofreciendo funcionalidades que incluyen la creación, modificación, eliminación y consulta de tarifas de impuestos, así como la asociación de cuentas contables auxiliares a cada tarifa correspondiente. Este submódulo se conecta con los módulos de Inventario (Promedio Ponderado y PEPS), donde las tarifas de impuestos pueden influir en el costo de los productos en inventario; resulta esencial para el Módulo Comercial, al facilitar el cálculo de impuestos en facturas de compra y venta y se relaciona con el Módulo Contable-Comercial, donde los asientos contables generados por transacciones comerciales incorporan los impuestos calculados conforme a las tarifas establecidas. De esta forma, constituye un elemento central para el manejo tributario en todas las operaciones del sistema. A continuación se muestran de manera gráfica las relaciones correspondientes del módulo de Gestión de tarifas de impuestos.

\begin{figure}[h!]    
    \centering%
    \includegraphics[width=0.9\textwidth, height=0.8\textheight, keepaspectratio]{Cap3/Figuras/Diagrama General-Impuestos.png}    
    \caption{Relación del submódulo gestión de tarifas de impuestos.}
    \label{fig:Impuestos}
\end{figure}

\newpage

\section{Gestión de terceros} Tiene como objetivo mantener una base de datos actualizada de clientes, proveedores y otras entidades relevantes para las operaciones del sistema, proporcionando funcionalidades que incluyen la creación manual de registros, la creación automática desde archivos PDF (RUT), la modificación, la importación desde archivos planos, la exportación a archivos planos, la consulta y la visualización completa de terceros. Este submódulo se relaciona con el Módulo Comercial, donde resulta esencial para identificar clientes en operaciones de venta y proveedores en procesos de compra; se vincula con el Módulo de Tesorería, permitiendo la gestión de pagos a proveedores; se relaciona con el Módulo de Cartera, posibilitando la gestión de cobros a clientes; interactúa con los Reportes Financieros (tanto de Libros Auxiliares como de Estados Financieros), donde los reportes pueden filtrarse y presentarse por tercero específico y se vincula con el Módulo Educativo Contable, que puede utilizar información de terceros para desarrollar simulaciones y ejercicios prácticos. De esta manera, constituye la base informativa que soporta las relaciones comerciales y financieras del sistema. A continuación se muestran de manera gráfica las relaciones correspondientes del módulo de Gestión de terceros.

\begin{figure}[h!]    
    \centering%
    \includegraphics[width=1.0\textwidth, height=0.8\textheight, keepaspectratio]{Cap3/Figuras/Diagrama General-Terceros.png}    
    \caption{Relación del submódulo gestión de terceros.}
    \label{fig:Terceros}
\end{figure}

\newpage

\section{Gestión de inventario de productos} Tiene como objetivo mantener el catálogo de productos actualizado y correctamente asociado a las cuentas contables e impuestos, permitiendo su gestión completa a través de funcionalidades que incluyen la creación, modificación, eliminación, importación desde archivo plano, exportación a archivo plano y consulta de productos en el inventario. Provee información detallada de los productos a los módulos de Inventario con Promedio Ponderado e Inventario con PEPS para la gestión de inventarios correspondiente, constituye el elemento central del Módulo Comercial donde los productos representan el núcleo de las transacciones de compra y venta, y se relaciona con el Módulo Contable-Comercial donde los movimientos de productos afectan directamente el costo de ventas y la generación de asientos contables, estableciéndose así como el repositorio fundamental que contiene toda la información necesaria para el manejo comercial y contable de los productos dentro del sistema. A continuación se muestran de manera gráfica las relaciones correspondientes del módulo de Gestión de inventario de productos.

\begin{figure}[h!]    
    \centering%
    \includegraphics[width=1.0\textwidth, height=0.8\textheight, keepaspectratio]{Cap3/Figuras/Diagrama General-Inventario productos.png}    
    \caption{Relación del submódulo gestión de inventario de productos.}
    \label{fig:Productos}
\end{figure}

\newpage

\section{Gestión de métodos de pago} Tiene como objetivo establecer los metodos de pago que se usarán como información adicional en los registros contables de comprobantes de egreso y recibos de caja, estableciendo la correcta equivalencia contable de cada método de pago a través de funcionalidades que incluyen la creación, modificación y eliminación de métodos de pago. Este submódulo se conecta con el Módulo de Tesorería, permitiendo la selección del método de pago para los comprobantes de egreso; se integra con el Módulo de Cartera, posibilitando la selección del método de pago para los recibos de caja y se relaciona con los módulos Contable-Comercial y Contable Cartera, donde los asientos contables generados por los pagos y cobros se basan en la parametrización establecida para cada método de pago. De esta forma, constituye el elemento que define y controla la aplicación contable de las diferentes formas de pago utilizadas en las transacciones del sistema contable. A continuación se muestran de manera gráfica las relaciones correspondientes del módulo de Gestión de métodos de pago.

\begin{figure}[h!]    
    \centering%
    \includegraphics[width=1.0\textwidth, height=0.8\textheight, keepaspectratio]{Cap3/Figuras/Diagrama General-Metodos de pago.png}    
    \caption{Relación del submódulo gestión de métodos de pago.}
    \label{fig:MetodosDePago}
\end{figure}

\newpage

\section{Gestión de tipos de documentos} Su objetivo es clasificar y configurar los diferentes tipos de documentos que se manejan en el sistema (factura de venta, factura de compra, recibo de caja, comprobante de egreso, nota de crédito, nota de débito), estableciendo una adecuada clasificación, flujo y registro contable de las operaciones a través de funcionalidades que incluyen la creación, modificación y eliminación de tipos de documentos, así como la definición de prefijos para cada tipo de documento. Este submódulo se conecta con el Módulo Comercial, definiendo los tipos de documentos comerciales como facturas, cotizaciones y órdenes; se integra con los módulos de Tesorería y Cartera, estableciendo los tipos de documentos de tesorería (comprobantes de egreso) y cartera (recibos de caja); se relaciona con los módulos Contable-Comercial y Contable Cartera, donde la generación de asientos contables está ligada al tipo de documento y se vincula con el Módulo de Auditoría, que filtra por tipos de documentos para visualizar los registros de acciones realizadas en el sistema. De esta forma, constituye el elemento normativo que define y controla la estructura documental de las operaciones. A continuación se muestran de manera gráfica las relaciones correspondientes del módulo de Gestión de tipos de documentos.

\begin{figure}[h!]    
    \centering%
    \includegraphics[width=1.0\textwidth, height=1.0\textheight, keepaspectratio]{Cap3/Figuras/Diagrama General-Tipos de documentos.png}    
    \caption{Relación del submódulo gestión de tipos de documentos.}
    \label{fig:TiposDocumento}
\end{figure}

\newpage

\section{Gestión de banco y cuentas bancarias} Su objetivo es registrar y mantener la información de las entidades bancarias y las cuentas bancarias de la empresa para la gestión de flujos de efectivo y conciliaciones, asociando las operaciones financieras con su representación contable mediante funcionalidades que incluyen la creación de bancos, la asociación de cada banco con su respectiva cuenta bancaria y cuenta contable, la modificación de información de bancos y cuentas, y la eliminación de bancos. Este submódulo se conecta con los módulos de Tesorería y Cartera donde los pagos y cobros que involucren movimientos bancarios se registran contra estas cuentas, se integra con los Reportes Financieros - Estados Financieros donde el estado de flujos de efectivo se nutre de la información de las cuentas bancarias y se relaciona con los módulos Contable-Comercial y Contable Cartera donde los asientos contables relacionados con movimientos bancarios se generan con base en esta configuración, constituyendo así el repositorio fundamental que vincula las operaciones bancarias físicas con su registro contable correspondiente en el sistema. A continuación se muestran de manera gráfica las relaciones correspondientes del módulo de Gestión de bancos y cuentas bancarias.

\begin{figure}[h!]    
    \centering%
    \includegraphics[width=1.0\textwidth, height=0.8\textheight, keepaspectratio]{Cap3/Figuras/Diagrama General-Banco y Cuentas Bancarias.png}    
    \caption{Relación del submódulo gestión de bancos y cuentas bancarias.}
    \label{fig:BancoYCuentas}
\end{figure}

\newpage

\section{Gestión de centros de costo} Su objetivo es permitir la segmentación de las operaciones contables y presupuestales por áreas funcionales, unidades organizacionales o proyectos para un análisis gerencial de resultados más detallado, proporcionando funcionalidades que comprenden la creación, modificación y eliminación de centros de costo y la habilitación de la selección de un centro de costo al registrar una operación contable o comercial. Este submódulo se conecta con el Módulo Comercial donde las ventas y compras pueden asignarse a centros de costo, se integra con el Módulo de Tesorería permitiendo que los egresos puedan asignarse a centros de costo, se relaciona con los módulos Contable-Comercial y Contable Cartera donde los asientos contables pueden incluir la dimensión de centro de costo, y se vincula con los Reportes Financieros tanto de Libros Auxiliares como de Estados Financieros donde los reportes pueden filtrarse y presentarse por centro de costo, constituyendo así el elemento dimensional que enriquece el análisis gerencial y la trazabilidad de las operaciones por unidades organizacionales específicas. A continuación se muestran de manera gráfica  las relaciones correspondientes del módulo de Gestión de centros de costo.

\begin{figure}[h!]    
    \centering%
    \includegraphics[width=1.0\textwidth, height=0.8\textheight, keepaspectratio]{Cap3/Figuras/Diagrama General-Centros de Costo.png}    
    \caption{Relación del submódulo gestión de centros de costo.}
    \label{fig:CentrosDeCosto}
\end{figure}

\newpage

\section{Gestión de centro de ayuda} Contribuye a la comprensión y el uso del sistema por parte de los usuarios, permitiendo la gestión de mensajes informativos y explicativos a través de funcionalidades que incluyen la creación, modificación de contenido y eliminación de cuadros de ayuda explicativos. Este módulo establece relaciones transversales con todos los módulos funcionales del sistema, ya que los cuadros de ayuda pueden aparecer en cualquier módulo para proporcionar ayuda contextual, explicaciones de conceptos o advertencias, los cuales se visualizan en un sitio web tipo blog donde se muestran las ayudas creadas, mejorando la experiencia del usuario al ofrecer información relevante y oportuna durante la navegación y operación de las diferentes funcionalidades del sistema, constituyendo así un elemento de soporte que contribuye a la usabilidad y accesibilidad de la plataforma en su conjunto. A continuación se muestran de manera gráfica las relaciones correspondientes del módulo de Gestión de cuadros de ayuda.

\begin{figure}[h!]    
    \centering%
    \includegraphics[width=0.4\textwidth, height=0.5\textheight, keepaspectratio]{Cap3/Figuras/Diagrama General-Cuadros de Ayuda.png}    
    \caption{Relación del submódulo gestión de cuadros de ayuda.}
    \label{fig:CuadrosAyuda}
\end{figure}

\newpage

\section{Gestión de calendario contable} Es el componente temporal que estructura y controla el ciclo de vida operativo del sistema. Su objetivo es definir periodos abiertos y cerrados para que todas las transacciones se registren en el momento correcto, contribuyendo a la integridad, trazabilidad y validez legal de la información financiera. Al permitir que el administrador abra o bloquee meses, evita registros fuera de periodos autorizados, soporta el cierre contable mensual y genera la base auditada para reportes oficializados. Este calendario valida fechas en inventarios, facturación, tesorería y cartera antes de permitir registros; restringe la generación de asientos contables en periodos cerrados y filtra reportes financieros por fechas activas. A continuación se muestran de manera gráfica las relaciones correspondientes del módulo de Gestión de calendario contable.

\begin{figure}[h!]    
    \centering%
    \includegraphics[width=1.0\textwidth, height=0.8\textheight, keepaspectratio]{Cap3/Figuras/Diagrama General-Calendario contable.png}    
    \caption{Relación del submódulo gestión de calendario contable.}
    \label{fig:CalendarioContable}
\end{figure}
\mychapter{Capítulo 4. Ingeniería de requisitos}
\label{cap:IngRequisitos} 
La ingeniería de requisitos es una etapa fundamental en un proyecto de software, ya que permite identificar, analizar y documentar las necesidades de un sistema antes de su implementación. Un levantamiento adecuado de requisitos permite que el resultado cumpla las expectativas del cliente. Es por esto que en este capítulo se describe el proceso llevado a cabo para la especificación de los requerimientos funcionales y no funcionales del módulo de configuración del ``Software contable para uso educativo en el programa de Contaduría Pública de la Universidad del Cauca''.

Durante la ingeniería de requisitos llevada a cabo en el proyecto se realizaron diferentes actividades con el fin de obtener una lista de tareas congruente con los requisitos del software contable educativo. Para hacer posible este proceso fue fundamental la comunicación constante con los principales interesados, docentes del programa de Contaduría Pública, el asesor empresarial Mauro Andrés Sánchez Muñoz y el director de la práctica profesional PhD. Wilson Libardo Pantoja Yepez, quienes muy amablemente reservaron espacios en sus agendas para atender las inquietudes resultantes. Esto permitió una retroalimentación constante y sumado a la flexibilidad de la metodología de trabajo híbrida, fue posible ir ajustando los requisitos en cada encuentro, incluso en etapas posteriores del desarrollo.

También se debe tener en cuenta que el análisis previo descrito en el \autoref{chap:submodulos}, en el cual se identificaron los submódulos que componen el módulo de configuración, permitió tener un contexto más amplio sobre los procesos y actividades involucradas en la gestión contable educativa. Lo anterior claramente benefició la recolección y especificación de requerimientos, dado que fue posible avanzar con mayor seguridad y velocidad en la definición de funcionalidades para cada submódulo.

Sin embargo, a pesar de la disposición de los interesados, se presentaron dificultades relacionadas con la coordinación de horarios y las múltiples ocupaciones de los participantes. En ocasiones, los docentes del programa de Contaduría Pública no disponían de tiempo suficiente para el proyecto debido a sus compromisos académicos, evaluaciones y actividades administrativas. A pesar de ello, se logró ajustar las actividades y el trabajo requerido para el levantamiento de requisitos, superando los inconvenientes mediante la flexibilidad horaria característica del enfoque freelance adoptado.

A continuación, se explica detalladamente la etapa de elicitación de requisitos, en la cual se capturaron, analizaron y validaron las necesidades con los stakeholders.
\section{Elicitación de requisitos}

La fase de elicitación de requisitos se llevó a cabo mediante un proceso sistemático que integra la experiencia práctica acumulada durante el Aprendizaje Basado en Proyectos (PBL) y actividades estructuradas de ingeniería de requisitos. El conocimiento previo del dominio contable, adquirido en las fases iniciales del PBL, proporcionó una base sólida para identificar y priorizar necesidades, mientras que las técnicas formales de elicitación permitieron una especificación completa y validada.


El proceso se organizó en cinco actividades principales:

\begin{itemize}
\item \textbf{Captura:} Se realizaron entrevistas estructuradas con los stakeholders listados en la  \autoref{tab:stakeholders}, se analizó documentación institucional y se levantaron modelos de negocio para comprender los flujos contables educativos.

\item \textbf{Especificación:} Los requisitos se formalizaron mediante historias de usuario, criterios de aceptación detallados y prototipos, lo que facilitó su comprensión y validación temprana.

\item \textbf{Negociación:} Se identificaron y resolvieron discrepancias entre las expectativas de los interesados, los lineamientos institucionales y las restricciones técnicas a través de videoconferencias de discusión y ajuste como se evidencia en la \autoref{fig:evidencia1}.

\item \textbf{Verificación:} Se aplicó revisión por pares como se evidencia en la \autoref{fig:evidenciaHU}, en la cual se presentaron las especificaciones a los directores del proyecto para confirmar su corrección, integridad, claridad y coherencia con los objetivos del sistema.

\item \textbf{Validación:} Los requisitos fueron expuestos y validados con los stakeholders clave, incluyendo docentes del programa de Contaduría Pública y asesores técnicos, mediante demostraciones de prototipos y pruebas funcionales preliminares como se muestra en la \autoref{fig:evidenciaprototipos}, asegurando su alineación con las expectativas y necesidades educativas.
\end{itemize}


\begin{table}[h]
\centering
\caption{Stakeholders del proyecto}
\label{tab:stakeholders}
\begin{tabular}{|p{6cm}|p{8cm}|}
\hline
\textbf{Interesado} & \textbf{Cargo} \\
\hline
Julio Ariel Hurtado Alegria & Asesor de la Universidad del Cauca \\
\hline
Wilson Libardo Pantoja & Asesor de la Universidad del Cauca \\
\hline
Juan Ignacio Oviedo & Asesor de la organización \\
\hline
Viviana Narvaez & Asesor de la organización \\
\hline
Mauro Sanchez & Asesor de la organización \\
\hline
Omar Gomez Gomez & Usuario Final Experto \\
\hline
Brayan Neil Vargas & Usuario Final Experto \\
\hline
Comunidad universitaria & Estudiantes administrativos y docentes \\
\hline
\end{tabular}
\end{table}

\begin{figure}[h!]    
    \centering%
    \includegraphics[width=1.0\textwidth, height=1.0\textheight, keepaspectratio]{Cap4/Figuras/evidencia1.png}    
    \caption{Evidencia sesiones de negociación.}
    \label{fig:evidencia1}
\end{figure}

    
\begin{figure}[h!]    
    \centering%
    \includegraphics[width=1.0\textwidth, height=1.0\textheight, keepaspectratio]{Cap4/Figuras/evidenciaHU.png}    
    \caption{Evidencia sesión de revisión de especificaciones.}
    \label{fig:evidenciaHU}
\end{figure}

\begin{figure}[h!]    
    \centering%
    \includegraphics[width=1.0\textwidth, height=1.0\textheight, keepaspectratio]{Cap4/Figuras/evidenciaprototipos.jpeg}    
    \caption{Evidencia sesión de revisión de prototipos.}
    \label{fig:evidenciaprototipos}
\end{figure}


\section{Especificación de requisitos funcionales}

La \autoref{tab:epicas} refleja las necesidades identificadas y se estructura mediante historias épicas, las cuales a su vez se desglosan en historias de usuario. Las historias épicas agrupan funcionalidades o tareas complejas, mientras que las historias de usuario detallan estas épicas en descripciones manejables que permiten su desarrollo independiente.

\begin{longtable}{|p{2cm}|p{6cm}|p{6cm}|}
\caption{Historias de Épicas}
\label{tab:epicas} \\
\hline
\textbf{ID Épica} & \textbf{Nombre} & \textbf{Descripción} \\
\hline
\endfirsthead

\multicolumn{3}{c}{{\bfseries Tabla \thetable\ Continuación: Historias de Épicas}} \\
\hline
\textbf{ID Épica} & \textbf{Nombre} & \textbf{Descripción} \\
\hline
\endhead
\hline
\endfoot
\endlastfoot
HE-01 & Gestionar el catálogo de cuentas & Yo como administrador del sistema
\newline Quiero gestionar el catálogo de cuentas
\newline Para la correcta administración, actualización y disponibilidad de la información contable necesaria \\
\hline
HE-02 & Gestionar las tarifas de los impuestos & Yo como administrador del sistema
\newline Quiero gestionar las tarifas de los impuestos
\newline Para el cálculo correcto, registro contable y actualización de los impuestos asociados a las transacciones del sistema \\
\hline
HE-03 & Gestionar la información de terceros (clientes, proveedores y otras entidades) & Yo como administrador del sistema
\newline Quiero gestionar la información de terceros (clientes, proveedores y otras entidades)
\newline Para tener disponibilidad de los datos necesarios para las operaciones empresariales \\
\hline
HE-04 & Gestionar el catálogo de productos del inventario & Yo como administrador del sistema
\newline Quiero gestionar el catálogo de productos del inventario
\newline Para tener la información de productos actualizada y correctamente asociada a las cuentas contables e impuestos \\
\hline
HE-05 & Gestionar el centro de ayuda & Yo como administrador del sistema
\newline Quiero gestionar el centro de ayuda
\newline Para facilitar la comprensión y el uso del sistema \\
\hline
HE-06 & Gestionar los métodos de pago & Yo como administrador del sistema
\newline Quiero gestionar los métodos de pago
\newline Para automatizar los registros contables en comprobantes de egreso y recibos de caja \\
\hline
HE-07 & Gestionar los tipos de documento & Yo como administrador del sistema
\newline Quiero gestionar los tipos de documento
\newline Para una adecuada clasificación, flujo y registro contable de las operaciones \\
\hline
HE-08 & Gestionar los bancos y sus cuentas bancarias & Yo como administrador del sistema
\newline Quiero gestionar los bancos y sus cuentas bancarias
\newline Para permitir que las operaciones financieras reales se asocien correctamente con su representación contable \\
\hline
HE-09 & Gestionar los centros de costos & Yo como administrador del sistema
\newline Quiero gestionar los centros de costos
\newline Para segmentar las operaciones contables y presupuestales por áreas funcionales, unidades organizacionales o proyectos \\
\hline
HE-10 & Gestionar el calendario contable interactivo & Yo como administrador del sistema
\newline Quiero gestionar el calendario contable interactivo
\newline Para el registro adecuado de las operaciones en el periodo correspondiente \\
\hline
\end{longtable}



La \autoref{tab:historiasUsuario} muestra fragmentos de las historias de usuario correspondientes a los submódulos de Catálogo de cuentas, Impuestos, Terceros y Métodos de pago especificadas mediante un identificador único, un nombre, una descripción y sus respectivos criterios de aceptación. Para explorar la especificación completa, incluyendo las historias épicas y las historias de usuario detalladas, se debe consultar el siguiente \href{https://docs.google.com/spreadsheets/d/1rFsgFhqhBDgjrOPtcxpHFp3UVVghU8dv/edit?usp=sharing&ouid=114064353546583843217&rtpof=true&sd=true}{\underline{\textcolor{blue}{enlace}}}, en el cual cada criterio de aceptación tiene asociado su respectivo prototipo desarrollado en Figma.

\begin{longtable}{|p{1.5cm}|p{4cm}|p{4cm}|p{6cm}|}
\caption{Historias de Usuario}
\label{tab:historiasUsuario} \\
\hline
\textbf{ID} & \textbf{Nombre} & \textbf{Descripción} & \textbf{Criterios de aceptación} \\
\hline
\endfirsthead
\multicolumn{4}{c}{{\bfseries Tabla \thetable\ Continuación: Historias de Usuario}} \\
\hline
\textbf{ID} & \textbf{Nombre} & \textbf{Descripción} & \textbf{Criterios de aceptación} \\
\hline
\endhead
\hline
\endfoot
\endlastfoot
HE-01-HU07 & Activar/desactivar cuentas contables & Yo como administrador del sistema
\newline Quiero activar/desactivar las cuentas contables en el catálogo de cuentas
\newline Para controlar qué cuentas están disponibles para uso en las operaciones contables & CA-01: Dado que me encuentro en la interfaz configuración/catálogo de cuentas y he seleccionado una cuenta contable Cuando doy clic en activar
\newline Entonces el sistema deberá mostrar un mensaje claro indicando que la cuenta contable fue activada correctamente
\newline CA-02: Dado que me encuentro en la interfaz configuración/catálogo de cuentas y he seleccionado una cuenta contable Cuando doy clic en desactivar 
\newline Entonces el sistema deberá mostrar un mensaje claro indicando que la cuenta contable fue desactivada correctamente \\
\hline
HE-02-HU05 & Activar/desactivar tarifas de impuestos & Yo como administrador del sistema
\newline Quiero activar/desactivar las tarifas de impuestos
\newline Para controlar qué impuestos están disponibles para uso en las operaciones contables & CA-01: Dado que me encuentro en la interfaz configuración/impuestos y he seleccionado un impuesto 
\newline Cuando doy clic en activar 
\newline Entonces el sistema deberá mostrar un mensaje claro indicando que el impuesto fue activado correctamente
\newline CA-02: Dado que me encuentro en la interfaz configuración/impuestos y he seleccionado un impuesto 
\newline Cuando doy clic en desactivar 
\newline Entonces el sistema deberá mostrar un mensaje claro indicando que el impuesto fue desactivado correctamente \\
\hline
HE-03-HU04 & Activar/desactivar información de terceros & Yo como administrador del sistema
\newline Quiero activar/desactivar la información de terceros desde la interfaz del sistema
\newline Para conservar registros vigentes y relevantes & CA-01: Dado que me encuentro en la interfaz configuración / Terceros y he seleccionado un tercero existente de la lista 
\newline Cuando doy clic en Activar/Desactivar 
\newline Entonces el sistema deberá mostrar un mensaje indicando que el tercero fue activado/desactivado correctamente
\newline CA-02: Dado que me encuentro en la interfaz configuración / Terceros / Tipo de Identificación y he seleccionado un tipo de identificación existente de la lista \newline Cuando doy clic en Activar/Desactivar 
\newline Entonces el sistema deberá mostrar un mensaje indicando que el tipo de identificación fue activado/desactivado correctamente
\newline CA-03: Dado que me encuentro en la interfaz configuración / Terceros / Tipo de Tercero y he seleccionado un tipo de tercero existente de la lista
\newline Cuando doy clic en Activar/Desactivar 
\newline Entonces el sistema deberá mostrar un mensaje indicando que el tipo de tercero fue activado/desactivado correctamente \\
\hline
HE-06-HU05 & Activar/desactivar métodos de pago & Yo como administrador del sistema
\newline Quiero activar/desactivar métodos de pago
\newline Para conservar registros vigentes y relevantes & CA-01: Dado que me encuentro en la interfaz configuración / métodos de pago y he seleccionado un método de pago existente de la lista
\newline Cuando doy clic en Activar/Desactivar
\newline Entonces el sistema deberá mostrar un mensaje indicando que el metodo de pago fue activado/desactivado correctamente \\
\hline
\end{longtable}




\section{Definición de requisitos no funcionales}

Una vez definidos los requisitos funcionales y consignados en historias de usuario, es importante considerar también los requisitos no funcionales, los cuales van más allá de las funcionalidades y definen aspectos de calidad del sistema.

La selección y definición de estos requisitos es el resultado de un proceso de consenso colaborativo gestado durante la primera etapa del Aprendizaje Basado en Proyectos (PBL), correspondiente al desarrollo inicial de los módulos básicos. A través de mesas de trabajo conjuntas entre los docentes del programa de Contaduría Pública y los asesores de la División TIC de la Univerdad del Cauca, se identificaron las necesidades fundamentales para la viabilidad del software a largo plazo. En estas sesiones se determinó que, dada la naturaleza evolutiva de la herramienta académica y el crecimiento proyectado, el sistema requería una base estructural robusta. Esta visión estratégica fue el factor determinante para la elección de un estilo arquitectónico basado en microservicios, decisión que fundamenta y da origen a los siguientes atributos de calidad:

\begin{itemize}
\item \textbf{Usabilidad:} En el desarrollo del presente proyecto se tuvieron en cuenta los lineamientos de desarrollo, usabilidad y diseño establecidos por la División TIC de la Universidad del Cauca. Específicamente, los documentos de usabilidad y diseño incluyen parámetros que se reflejan en el uso de tipografía, colores, contrastes y enfoque de diseño adecuados para proporcionar una buena experiencia al usuario final. Por lo tanto, en el proyecto se recogen y aplican estas reglas mediante una interfaz que cumple, entre otras cosas, con las especificaciones de la División TIC y Material Design, sobre las cuales están basados dichos documentos.

\item \textbf{Escalabilidad:} Considerando que el número de usuarios puede crecer con el tiempo y que el volumen de información contable aumenta mes a mes, el sistema fue diseñado pensando en el futuro. La arquitectura de microservicios, seleccionada en las fases iniciales, permite que cuando haya más usuarios trabajando simultáneamente o cuando se procesen más transacciones, el sistema pueda ajustarse sin necesidad de hacer cambios drásticos en su estructura. Esto significa que si hay períodos de mayor actividad, la aplicación puede manejar esta carga adicional sin afectar su funcionamiento normal ni la experiencia de los usuarios.

\item \textbf{Mantenibilidad:}  Con el fin de facilitar la evolución y mejora continua del sistema, se estableció como requisito que el código fuera claro y bien organizado. Cada microservicio está documentado con su propósito y funcionamiento, lo que facilita la colaboración entre desarrolladores. Además, las pruebas automatizadas aseguran el correcto funcionamiento ante cambios reduciendo el riesgo de errores. Esta estructura permite corregir fallos, incorporar nuevas funcionalidades y mantener el sistema actualizado con las normativas contables vigentes sin complicaciones.

\end{itemize}

\section{Prototipos}

La construcción de prototipos fue esencial para una comunicación clara con el asesor empresarial, ya que al permitir visualizar un acercamiento de lo que serían las interfaces del módulo de Configuración se obtenía retroalimentación constante y rápida sobre aspectos pedagógicos y de usabilidad. Esto ha permitido realizar un refinamiento de la especificación de requerimientos en el formato de historias de usuario, este último también ha contribuido en sentido contrario a detallar más cuidadosamente los prototipos, generando así un círculo virtuoso de mejora continua.

Los prototipos se enfocaron en representar las interfaces de los submódulos que componen el módulo de Configuración, cada prototipo consideró los lineamientos de diseño UX-UI establecidos por la División TIC de la Universidad del Cauca, incluyendo la paleta de colores institucional y las fuentes tipográficas especificadas.

Para el diseño de los prototipos se hizo necesario el uso de una herramienta flexible y de fácil manejo, en la que se pudiera plasmar rápidamente el concepto de las interfaces del módulo de Configuración, la herramienta elegida fue Figma por su comodidad, flexibilidad y fácil manejo para el diseño de interfaces web, además de permitir la colaboración en tiempo real con los interesados del proyecto. Por tanto, si se desea explorar en su totalidad los prototipos desarrollados, es preciso dirigirse al proyecto Figma en el siguiente \href{https://www.figma.com/design/b28M2VlaXjz2JrX5TvlfwF/MODULO-CONFIGURACION?node-id=1092-13124}{\underline{\textcolor{blue}{enlace}}}, donde se encuentran compilados todos los diseños de las interfaces. Igualmente, se incluyen a continuación ejemplos seleccionados de los prototipos desarrollados.

\subsection{Vista de catálogo de cuentas}

La \autoref{fig:CatalogoCuentas} muestra el prototipo asociado al catálogo de cuentas, que establece la estructura maestra de clasificación contable donde se definen, organizan y gestionan todas las cuentas que registrarán las transacciones financieras del sistema. Configura el esquema contable maestro del sistema a través de una interfaz que permite gestionar jerárquicamente la estructura de clasificación financiera. Desde este módulo central, el administrador puede crear, importar o exportar cuentas codificadas numéricamente del 1 al 9, organizadas según naturaleza contable (Activo, Pasivos, Patrimonio, Ingresos, Gastos, Costos de transformación y Cuentas de orden), con control de estado activo/inactivo para mantener vigente el plan de cuentas.

\begin{figure}[h!]    
    \centering%
    \includegraphics[width=1.0\textwidth, height=1.0\textheight, keepaspectratio]{Cap4/Figuras/catalogo de cuentas.png}    
    \caption{Prototipo vista catálogo de cuentas.}
    \label{fig:CatalogoCuentas}
\end{figure}

\subsection{Vista de inventario de productos}

La \autoref{fig:InventarioProductos} muestra el prototipo asociado al inventario de productos establece el catálogo maestro de artículos donde se centraliza la gestión de insumos, productos terminados y mercancías a través de una interfaz que permite crear, importar, exportar y administrar registros con atributos clave como referencia, nombre, descripción, costo, cantidad, unidad de medida, categoría, tipo de producto y estado activo.

\begin{figure}[h!]    
    \centering%
    \includegraphics[width=1.0\textwidth, height=1.0\textheight, keepaspectratio]{Cap4/Figuras/inventario de productos.png}    
    \caption{Prototipo vista inventario de productos.}
    \label{fig:InventarioProductos}
\end{figure}

\newpage
\subsection{Vista de terceros}

La \autoref{fig:tercero} muestra el prototipo asociado a terceros,  configura el registro maestro de entidades externas a través de una interfaz que permite crear, importar, exportar y administrar personas naturales y jurídicas, capturando atributos esenciales como identificación, tipo de entidad, correo electrónico y estado activo. Incorpora la funcionalidad específica de procesar archivos PDF del RUT para automatizar la creación de registros, validando duplicados y formatos de datos requeridos. Este módulo opera como repositorio centralizado que estructura y mantiene actualizada la base de datos de clientes, proveedores y otros actores, habilitando búsquedas dinámicas por nombre o numero de identificación.

\begin{figure}[h!]    
    \centering%
    \includegraphics[width=1.0\textwidth, height=1.0\textheight, keepaspectratio]{Cap4/Figuras/terceros.png}    
    \caption{Prototipo vista de terceros.}
    \label{fig:tercero}
\end{figure}


\newpage
\subsection{Vista de impuestos }

La \autoref{fig:impuestos} muestra el prototipo asociado a impuestos encargado de la configuracion maestra de tarifas mediante una interfaz que permite crear, modificar y eliminar registros, asignando descripciones, porcentajes y las cuentas contables específicas para compras y ventas. Centraliza la parametrización de tributos como IVA y retenciones, validando la existencia de las cuentas asociadas.

\begin{figure}[h!]    
    \centering%
    \includegraphics[width=1.0\textwidth, height=1.0\textheight, keepaspectratio]{Cap4/Figuras/impuestos.png}    
    \caption{Prototipo vista de impuestos.}
    \label{fig:impuestos}
\end{figure}
\mychapter{Capítulo 5. Arquitectura del sistema}

\mychapter{Capítulo 6. Patrones de diseño}

Los patrones de diseño son soluciones elegantes y estructuradas para problemas comunes en el desarrollo de software. Su aplicación en el presente proyecto ha permitido estructurar el código de manera más eficiente, mejorando la escalabilidad y el mantenimiento. En este capítulo se exploran algunos patrones de diseño utilizados en el módulo de configuración del software contable educativo.

En el presente capítulo se contextualizan algunos patrones de diseño implementados en este desarrollo. Para ello, se proporciona una breve introducción a cada patrón, se explican las motivaciones de su uso y se incluyen diagramas que ilustran el patrón y su implementación en el proyecto.
\mychapter{Capítulo 7. Evaluación de Funcionalidades}

En este capítulo se describen las pruebas realizadas al módulo de configuración, con el objetivo de evaluar sus funcionalidades y verificar su correcto funcionamiento e integración con el resto del sistema.


\section{Pruebas del Backend}

\subsection{Pruebas unitarias}

Las pruebas unitarias desempeñan un papel fundamental en el desarrollo de software, permitiendo validar el comportamiento correcto de componentes individuales de manera independiente. En este proyecto, se aplicaron pruebas unitarias mediante JUnit 5 y Mockito, que juntas simplifican la elaboración de tests automatizados. JUnit 5 ofrece la estructura necesaria para diseñar, ordenar y llevar a cabo las pruebas, en tanto que Mockito simula las interacciones con dependencias externas a través de mocks, asegurando un aislamiento total de cada elemento en su verificación. Esta sinergia tecnológica garantiza que cada módulo cumpla con las especificaciones requeridas antes de su combinación con otras partes del sistema. Con el fin de lograr pruebas veloces y efectivas, se descartó el empleo de entornos complejos como Spring Boot, centrándose en examinar las piezas de código de manera autónoma. Esta estrategia proporciona una respuesta inmediata durante la fase de desarrollo y ayuda a detectar fallos de forma precoz.


\mychapter{Capítulo 8. Conclusiones, lecciones aprendidas y trabajos
 futuros}

 \section{Conclusiones}

El desarrollo del módulo de Configuración para el software contable educativo ContappUC representa un aporte significativo al programa de Contaduría Pública de la Universidad del Cauca, demostrando cómo la ingeniería de software puede resolver necesidades específicas del ámbito académico contable mediante soluciones tecnológicas y escalables. A continuación, se presentan las conclusiones principales:

\begin{itemize}
\item \textbf{Integración efectiva de teoría y práctica:} El proyecto logró trascender el ámbito teórico al enfrentar los desafíos propios del desarrollo de software empresarial. La implementación exitosa de los diez submódulos que conforman el módulo de Configuración (Catálogo de cuentas, Tarifas de impuestos, Terceros, Inventario de productos, Métodos de pago, Tipos de documentos, Cuentas bancarias, Centros de costo, Centro de ayuda y Calendario contable) validó la capacidad para traducir requerimientos contables en funcionalidades de software. Estos componentes establecen los parámetros maestros fundamentales que soportan la operación de todo el sistema contable educativo.
\item \textbf{Ingeniería de requisitos orientada al contexto educativo:} La aplicación de técnicas de elicitación, como entrevistas estructuradas con docentes del programa de Contaduría Pública y el uso de historias de usuario, junto con el diseño de prototipos en Figma, facilitó una alineación precisa entre las necesidades académicas y la solución final. La fase previa de exploración mediante Aprendizaje Basado en Proyectos (PBL) durante 2024-1 y 2024-2 permitió comprender profundamente el dominio contable antes de iniciar el desarrollo formal, resultando en una especificación de requisitos más completa y contextualizada que se refleja en las 10 épicas y sus correspondientes historias de usuario documentadas.
\item \textbf{Arquitectura flexible y mantenible:} La adopción de dos patrones arquitectónicos complementarios demostró ser una decisión acertada para equilibrar complejidad y mantenibilidad. La arquitectura hexagonal (Ports and Adapters) se aplicó en los módulos desarrollados durante la fase PBL (Catálogo de cuentas, Terceros, Productos e Impuestos), aprovechando el trabajo previo y garantizando la independencia del dominio de negocio. Para los módulos restantes, que corresponden principalmente a operaciones CRUD sin lógica compleja, se implementó una arquitectura en capas que priorizó la simplicidad y la velocidad de desarrollo. Esta dualidad arquitectónica permitió optimizar recursos sin comprometer la calidad del sistema.
\item \textbf{Comunicación asíncrona entre microservicios:} La implementación del patrón publicador-suscriptor mediante RabbitMQ para la actualización de contadores de uso de recursos (como centros de costo, terceros y métodos de pago) permitió desacoplar el módulo de Configuración de otros módulos. Esta arquitectura orientada a eventos resultó en un sistema resiliente capaz de mantener la integridad referencial y prevenir inconsistencias en reportes contables, sin bloquear el flujo principal de las operaciones comerciales y financieras.
\item \textbf{Gestión ágil adaptada al contexto freelance:} La metodología de trabajo híbrida, fundamentada en principios del trabajo freelance y gestionada mediante Kanban, proporcionó la flexibilidad necesaria para un desarrollo individual exitoso en coordinación con otros desarrolladores del ecosistema contable. El enfoque de entregas por fases, permitió una adaptación ágil ante cambios en los requisitos y facilitó entregas continuas de valor al cliente, manteniendo siempre la autonomía característica del modelo freelance.
\item \textbf{Estrategia integral de aseguramiento de calidad:} La implementación de una estrategia multicapa de pruebas garantizó la confiabilidad del sistema en todos sus niveles. Las pruebas unitarias con JUnit 5 y Mockito alcanzó una cobertura superior al 70\%, mientras que las pruebas de integración automatizadas con Postman y Newman validaron el comportamiento correcto de 47,760 aserciones en 20 iteraciones consecutivas sin fallos. Las pruebas de aceptación, ejecutadas de manera virtual con los usuarios finales expertos, validaron que las funcionalidades implementadas cumplen con las expectativas pedagógicas y operativas del programa de Contaduría Pública.
\end{itemize}

  
 \section{Lecciones aprendidas}

\begin{itemize}   
    
    \item \textbf{Comprensión del dominio de negocio:} El estudio exhaustivo del dominio contable fue esencial para el éxito del proyecto. Entender los fundamentos de la estructura del plan de cuentas, la clasificación de transacciones y los principios de registro contable permitió tomar decisiones de diseño más acertadas y establecer una comunicación efectiva con los asesores y docentes expertos del programa de Contaduría Pública.
    
    \item \textbf{Coordinación en ecosistemas de microservicios:} Aunque el desarrollo del módulo fue responsabilidad individual, la arquitectura distribuida demostró que la comunicación continua con otros equipos de desarrollo resulta fundamental. Las reuniones periódicas para establecer contratos de API y compartir patrones de diseño comunes permitieron evitar redundancias en el código y permitieron la integración fluida del módulo de Configuración con los demás componentes del sistema contable.    
    
    \item \textbf{Manejo de operaciones asíncronas en procesamiento masivo:} La implementación de funcionalidades de importación masiva de datos (catálogo de cuentas, terceros y productos) demostró la importancia de diseñar procesos asíncronos robustos. Fue necesario desarrollar estrategias de procesamiento por lotes, mecanismos de validación incremental y gestión controlada de errores para garantizar que la carga de grandes volúmenes de información no bloqueara la aplicación y permitiera identificar y reportar fallos específicos sin comprometer la integridad del sistema.
   
\end{itemize}

\section{Trabajos futuros}

\begin{itemize}
\item \textbf{Integración completa del módulo de Configuración con el resto del sistema contable:}
Actualmente, el registro del uso de los parámetros maestros como centros de costo, cuentas contables, métodos de pago se ha implementado únicamente con el módulo de cartera y los módulos de inventario (PEPS y Promedio Ponderado). Es necesario extender esta integración a los demás módulos del sistema, de modo que cada vez que un parámetro configurado sea utilizado, se actualice su contador de uso. Esto permitirá mantener la integridad referencial, evitar eliminaciones o modificaciones en parámetros con movimientos contables y enriquecer la trazabilidad de las operaciones.

\item \textbf{Refactorización de los módulos de Impuestos y Terceros a arquitectura en capas:}  
Durante las fases iniciales del proyecto, estos módulos se implementaron bajo el patrón hexagonal (Ports and Adapters) como parte del enfoque de Aprendizaje Basado en Proyectos (PBL). Si bien esta decisión permitió aprovechar el trabajo ya desarrollado y mantener la coherencia con los primeros entregables, ambos módulos son esencialmente operaciones CRUD sin lógica de negocio compleja. Su refactorización hacia una arquitectura en capas simplificaría su estructura y reduciría la complejidad innecesaria. Este trabajo no se incluyó en el alcance de la presente práctica debido al tiempo adicional que requería, pero se recomienda como mejora prioritaria para futuras iteraciones.
\end{itemize}



\phantomsection
\addcontentsline{toc}{chapter}{\numberline{}Bibliografía}
\printbibliography

\begin{appendix}
\chapter{Anexo: Nombrar el anexo A de acuerdo con su contenido}\label{AnexoA}
Los Anexos son documentos o elementos que complementan el cuerpo de la tesis o trabajo de investigaci\'{o}n y que se relacionan, directa o indirectamente, con la investigaci\'{o}n, tales como acetatos, cd, normas, etc.\\

\chapter{Anexo: Nombrar el anexo B de acuerdo con su contenido}
A final del documento es opcional incluir \'{\i}ndices o glosarios. \'{E}stos son listas detalladas y especializadas de los t\'{e}rminos, nombres, autores, temas, etc., que aparecen en el mismo. Sirven para facilitar su localizaci\'{o}n en el texto. Los \'{\i}ndices pueden ser alfab\'{e}ticos, cronol\'{o}gicos, num\'{e}ricos, anal\'{\i}ticos, entre otros. Luego de cada palabra, t\'{e}rmino, etc., se pone coma y el n\'{u}mero de la p\'{a}gina donde aparece esta informaci\'{o}n.\\

\chapter{Anexo: Nombrar el anexo C de acuerdo con su contenido}
MANEJO DE LA BIBLIOGRAF\'{I}A: la bibliograf\'{\i}a es la relaci\'{o}n de las fuentes documentales consultadas por el investigador para sustentar sus trabajos. Su inclusi\'{o}n es obligatoria en todo trabajo de investigaci\'{o}n. Cada referencia bibliogr\'{a}fica se inicia contra el margen izquierdo.\\

La NTC 5613 establece los requisitos para la presentaci\'{o}n de referencias bibliogr\'{a}ficas citas y notas de pie de p\'{a}gina. Sin embargo, se tiene la libertad de usar cualquier norma bibliogr\'{a}fica de acuerdo con lo acostumbrado por cada disciplina del conocimiento. En esta medida es necesario que la norma seleccionada se aplique con rigurosidad.\\

Es necesario tener en cuenta que la norma ISO 690:1987 (en Espa\~{n}a, UNE 50-104-94) es el marco internacional que da las pautas m\'{\i}nimas para las citas bibliogr\'{a}ficas de documentos impresos y publicados. A continuaci\'{o}n se lista algunas instituciones que brindan par\'{a}metros para el manejo de las referencias bibliogr\'{a}ficas:\\



\end{appendix}
\end{document}
